%
% closure operators on sets and algebraic lattices
%   Sergiu Rudeanu 
% http://fmi.unibuc.ro/revistadelogica/articole/No1Art113.pdf
%
\section{Closure Operators}

\begin{definition} [closure operators]
If $P$ is a poset, a map 
$\closureOP : P \longrightarrow P$
is called a {\em closure operator},
if for every $ x, y \in P$ the following properties hold:
\begin{enumerate}
\item {\em extensivity} :  $x \leq \closureX$,
\item {\em idempotence} : $ \dclosureX \leq \closureX$ 
(extensibity gives $\closureX \leq \dclosureX$, hence $\dclosureX = \closureX$),
\item {\em monotonicity} : $x \leq y$ implies $\closureX \leq \closureY$.
\end{enumerate}
\end{definition}

\begin{proposition}
The definition above is equivalent to the single axiom:
\begin{prooftree}
\AxiomC{$ x \fCenter \closureY$}
\doubleLine
\UnaryInfC{$\closureX \fCenter \closureY$}
\end{prooftree}
\end{proposition}
\begin{proof}
($\Rightarrow$)
\begin{prooftree}
\AxiomC{$ \closureX \fCenter \closureX$}
\UnaryInfC{$ x \fCenter \closureX$}
\DisplayProof \hskip 48pt
\AxiomC{$ \closureX \fCenter \closureX$}
\UnaryInfC{$ \dclosureX \fCenter \closureX$}
\DisplayProof \hskip 48pt
\AxiomC{$ x \fCenter y$}
\AxiomC{$ y \fCenter \closureY$}
\BinaryInfC{$ x \fCenter \closureY$}
\UnaryInfC{$ \closureX \fCenter \closureY$}
\end{prooftree}
%%%
($\Leftarrow$)
\begin{prooftree}
\AxiomC{$ x \fCenter \closureY $}
\RightLabel{\scriptsize(3)}
\UnaryInfC{$ \closureX \fCenter \dclosureY $}
\AxiomC{$$}
\RightLabel{\scriptsize(2)}
\UnaryInfC{$ \dclosureY \fCenter \closureY $}
\BinaryInfC{$ \closureX \fCenter \closureY $}
\DisplayProof \hskip 48pt
\AxiomC{$$}
\RightLabel{\scriptsize(1)}
\UnaryInfC{$ x \fCenter \closureX $}
\AxiomC{$ \closureX \fCenter \closureY $}
\BinaryInfC{$ x \fCenter \closureY $}
\end{prooftree}
\end{proof}

\begin{definition} [closed elements]
The element satisfying $\closureX  = x$ are said to be {\em closed}. 
We denote by $C$ the set of closed elements: 
$C = \set{x \in P | \closureX = x }$.
\end{definition}

\begin{definition}[principal filter]
For any $x \in P$, we define the set called
 {\em principal filter generated by $x$} as follows: 
$\PfilterX = \set{y \in P  | x \leq y}$.
In other words, $\PfilterX$ is a set of all elements which includes $x$. 
\end{definition}

\begin{definition} [closure system]
If $P$ is a poset, by a {\em closure system} we mean any non-empty
subset $C \subseteq P$ such that for each $x \in P$, the set
$\CcapPfilterX$ has the least element.
\end{definition} 


\begin{proposition} [the set of $C$-closed elements is a closure system]
\label{closureOtoS}
The set of closed elements $C$ of a closure operator $\closureOP$
is a closure system. 
\end{proposition}
\begin{proof}
We show that $\closureX$ is the least element of  $\CcapPfilterX$. 
\begin{enumerate}
\item 
{\em $\closureX \in \PfilterX$}: since we have $x \leq \closureX$ by extensivity. 
%
\item
{\em $\closureX \in C$} : since we have $\dclosureX = \closureX$ by idempotency.
%
\item{\em $\closureX$ is the least element of $\CcapPfilterX$} :
Suppose $y \in \PfilterX$: i.e.  $x \leq y$. 
Then we have $\closureX \leq \closureY$ by monotonicity. 
But $y \in C$ means $\closureY = y$,
hence $\closureX \leq y$  for all $y \in \CcapPfilterX$. 
\end{enumerate}  
Pictorially,
\begin{prooftree}
\AxiomC{$y \in \PfilterX$}
\UnaryInfC{$ x \fCenter y$}
\RightLabel{(by monotonicity)}
\UnaryInfC{$ \closureX \fCenter \closureY$}
\AxiomC{$ y \in C$}
\UnaryInfC{$ \closureY = y$}
\BinaryInfC{$\closureX \fCenter y$}
\end{prooftree}
\end{proof}


\begin{proposition}
\label{xLeastElement}
Let $C$ be a closure system. 
If $x \in C$ then $x$ is the least element of $\CcapPfilterX$. 
\end{proposition}
\begin{proof}
\begin{enumerate}
\item $x \in \PfilterX$ : $x \leq x$ implies $x \in \PfilterX$.
\item  $x \in C$:  by assumption.
\item {\em $x$ is the least element of $\CcapPfilterX$} : 
obvious. 
\end{enumerate}
\end{proof}

\begin{proposition} [A closure system induces a closure operator]
\label{closureStoO}
Let $P$ be a poset and  $C$ a closure system of $P$. 
The assignment 
\[ x \mapsto \closureX 
\phantom{x} \defeq \phantom{x }\mbox{the least element of}\phantom{x} \CcapPfilterX \] 
gives a closure operator. 
\end{proposition}
\begin{proof}
\begin{enumerate}
\item {\em extensivity } : Since $\closureX \in \PfilterX$, it directly follows $x\leq\closureX$.
\item {\em idempotence } : Since $\closureX \in C$ and $\closureX \leq \closureX$, 
it directly follows $\closureX \in \CcapPfilterCX$. 
But $\dclosureX$ is the least element of $\CcapPfilterCX$ by definition,
thus we have $\dclosureX \leq \closureX$. 
\item {\em monotonicity}:
Clearly $x \leq y$ means $\PfilterY \subseteq \PfilterX$.  
Thus $\closureY \in \CcapPfilterY$ implies $\closureY \in \CcapPfilterX$. 
But $\closureX$ is the least element of $\CcapPfilterX$ by definition,
thus we have $\closureX \leq \closureY$. 
\end{enumerate}
\end{proof}

\begin{theorem}
The maps constructed 
in Propositions~\ref{closureOtoS} and~\ref{closureStoO}
establish a bijection between the closure operators and the closure systems of $P$.
\end{theorem}
\begin{proof}
\begin{enumerate}
\item  Let $C$ be a closure system and $\closureOP$ an induced closure operator: 
Since $\closureX = x$ is the least element of $\CcapPfilterX$
(Proposition~\ref{closureOtoS}),
it directly follows $x \in C$. 
Conversely, if $x \in C$,  then $x$ is the least element of $\CcapPfilterX$(Proposition~\ref{xLeastElement}).
%
\item  Let $\closureOP$ be a closure operator and $C$ be a set of closed elements: 
Then $C$ is a closure system and 
$\closureX$ is the least element of $\CcapPfilterX$
(Proposition~\ref{closureOtoS}).
So the induced closure operator  from $C$ is the same as $\closureOP$. 
\end{enumerate}
\end{proof}
%

%
\begin{proposition}
A subset $C$ of a complete lattice $L$ is a closure system 
iff $C$ is closed under arbitrary meet.
\end{proposition}

\begin{proof}
($\Rightarrow$) 
Suppose $C$ is a closure system and let $\closureOP$ be the induced closure operator. 
Let $S$ be a subset of $C$ and set $x = \bigwedge S$. 
We want to show that $x$ is also a closed element of $C$; i.e. $\closureX = x$. 
\begin{enumerate}
\item
$x \leq s$ for all $s$ in $S$ means $\closureX \leq \closure{s} = s$ by monotonicity.
It immediately follows that $\closure{x}$ is a lower bounds for $S$ hence $\closure{x} \leq x$. 
\item 
But $x \leq \closureX$ by extensivity, we conclude $\closureX = x$.
\end{enumerate}
%%
($\Leftarrow$)
Clearly $\closureX = \bigwedge (\CcapPfilterX) $ 
 is the least element if $\closureX \in \CcapPfilterX$.
\begin{enumerate}
\item
{\em $\closureX \in C$} : by assumption. 
\item 
{\em $\closureX \in \PfilterX$; i.e. $x \leq \closureX$}: 
 since $x$ is a lower bound for $\CcapPfilterX$, and $\closureX$ is the greatest one. 
\end{enumerate}
\end{proof}
%
%%  complete lattice
%
The above proposition implies that $(C,\bigwedge)$ is a complete meet-semilattice 
under the same meet operation as $L$. 
Recall that complete meet-semilattice is a complete lattice (the dual of Proposition~\ref{LTC:completeLattice}).

\begin{theorem}
Every closure sysytem $C$ of a complete lattice $(L,\bigvee,\bigwedge)$
is a complete lattice $(C,\closure{\bigvee},\bigwedge)$. 
\end{theorem}
\begin{proof}
Since $\bigvee A$ is the least upper bound in a poset $P$, 
$\principalFilter{\bigvee A}$ is a set of upper bounds for $A$. 
Thus $\closure{\bigvee A}$
(the least element in $C \cap \principalFilter{\bigvee A}$)
is the least upper bound for $A$
in $C$.
\end{proof}

%%  kernel operators

\subsection{kernel operators}
A kernel operator is a dual of kernel operators. Explicitly, 

\begin{definition} [kernel operators]
If $P$ is a poset, a map 
$\kernelOP : P \longrightarrow P$
is called a {\em kernel operator},
if for every $ x, y \in P$ the following properties hold:
\begin{enumerate}
\item {\em extensibity} :  $ \kernelX \leq x$,
\item {\em idempotence} : $ \kernelX \leq \dkernelX$ 
(extensibity gives $\dkernelX \leq \kernelX$, hence $\kernelX = \dkernelX$),
\item {\em monotonicity} : $x \leq y$ implies $\kernelX \leq \kernelY$.
\end{enumerate}
The definition above is equivalent to the single axiom:
\begin{prooftree}
\AxiomC{$ \kernelX \fCenter y $}
\doubleLine
\UnaryInfC{$\kernelX \fCenter \kernelY$}
\end{prooftree}
\end{definition}