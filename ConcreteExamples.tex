\subsection{Concrete Examples of Finite residual lattices}

We describe some small finite examples of residual lattices. 
Hasse diagram is enough to describe the structure of finite lattices. 
We arrange all the possible commutative monoid
of all the elements $L$ in a Cayley table.
Since it is cumbersome to test 
whether the monoid ($\RLotimes$) satisfies the associativity.

\begin{definition} [Light's associativity test]
Choosing some element a in A, and set two new binary operations are defined in $L$. 
\[ x \triangleright  y \defeq  x \RLotimes ( a \RLotimes y) \hskip 72pt
 x \triangleleft y \defeq  (x \RLotimes a) \RLotimes y \]
 The Cayley tables of these operations are constructed and compared. 
If the tables coincide then associativity holds for an element $a \in L$. 
This is repeated for every element of the set $L$. 
 \end{definition}
 
The followings are two obvious cases for Light's associativity test.
\begin {enumerate}
\item 
{\em $a = \LTCbot$}:
since $a \RLotimes \LTCbot = \LTCbot$ for all $a$ (Proposition~\ref{RL:anihilation}), 
$x \triangleright  y  = x \triangleleft  y = \LTCbot$.
\item  
{\em $a = \RLone$}:  $x \triangleright  y  = x \triangleleft  y = x \RLotimes y$.  
\end{enumerate}

Let $(L,\vee,\LTCbot,\wedge,\LTCtop,\RLimp,\RLotimes,\RLone,\RLzero)$ be a small finite
residuated lattice with zero. 

There are some useful facts to determine the concrete structure of residuated lattices. 

\begin{fact}
\begin{enumerate}
\item {\em We can't set $\RLone = \LTCbot$ because of residual law}: 
By residual law  $a \RLotimes \QTargument$ has a right adjoint $a \RLimp \QTargument$.
Thus it preserves all joins including $\LTCbot$;
$a \RLotimes \LTCbot = \LTCbot$ for every $a \in L$(Proposition~\ref{RL:anihilation}.
However $a \RLotimes \RLone = a$ (for all $a \in L$)  by definition.
Hence $\RLone$ can't be equal to $\LTCbot$. 
\item  {\em $\RLzero = \LTCtop$ does not seems to be interesting}:
$ a \RLimp \LTCtop = \LTCbot \RLimp a = \LTCtop$ for each $a \in L$. 
Specifically to set $\RLzero = \LTCtop$ dose not seems to be interesting;
in this case, $\RLneg{a} = \LTCtop$ for all $a \in L$.
This exactly means that
the only $\dRLneg{\QTargument}$-closed element is $\LTCtop$ in $L$. 
\begin{prooftree}
\AxiomC{$ a \RLotimes \LTCtop \fCenter \LTCtop$}
\UnaryInfC{$\LTCtop \fCenter a \RLimp \LTCtop$}
\DisplayProof \hskip 48pt
\AxiomC{$ \LTCbot \RLotimes \LTCtop \fCenter a$}
\UnaryInfC{$\LTCtop \fCenter \LTCbot \RLimp a$}
\end{prooftree}
\item 
$ \RLone \RLimp a = a$ 
\begin{prooftree}
\AxiomC{$\RLone \fCenter \RLone$}
\AxiomC{$a \fCenter a$}
\BinaryInfC{$ \RLone \RLotimes (\RLone \RLimp a) \fCenter a$}
\DisplayProof \hskip 24pt
\AxiomC{$\RLone \RLotimes a \fCenter a$}
\UnaryInfC{$a \fCenter \RLone \RLimp a$}
\end{prooftree}
\end{enumerate}
\end{fact}

\begin{example} [a one-element residual lattice with zero is a Boolean algebra]
Since it is a one-element lattice, $L = \set{\LTCtop}$, $\LTCbot = \LTCtop$ and $\LTCjoin = \LTCmeet$. 
We also have $\RLone = \LTCtop$ (weakening) and $\LTCtop \RLotimes \LTCtop = \LTCtop$ (contraction),
it actually is a one-element Heyting algebra. 
$\LTCtop \RLotimes \LTCtop \leq \LTCtop$ implies $\LTCtop \leq \LTCtop \RLimp \LTCtop$. 
Cleary $\RLzero = \LTCtop$ is the only choise and $\RLneg{\LTCtop} = \LTCtop$(self-dual),
concluding  one-element residual lattices with zero 
$(\set{\LTCtop}, \LTCmeet,\LTCtop,\LTCmeet,\LTCtop,\LTCmeet,\LTCmeet,\LTCtop,\LTCtop)$ 
are (one-element) Boolean Algebras 
$(\set{\LTCtop}, \LTCmeet,\LTCtop,\LTCmeet,\LTCtop,\RLneg{\QTargument})$.
\end{example}

\begin{table}[hbt]
\begin{center}
  \begin{tabular}{c|cc}
  $\RLimp$ & $\LTCbot$ &  $\LTCtop$  \\ \hline
  $\LTCbot$  & $\LTCtop$ &  $\LTCtop$ \\
  $\LTCtop$  & $\LTCbot$ &  $\LTCtop$ \\
  \end{tabular}
\hskip 18pt
  \begin{tabular}{c|c}
  $\RLneg{\QTargument}$ 
      & $\RLzero = \LTCbot$   \\ \hline
  $\LTCbot$  & $\LTCtop$ \\
  $\LTCtop$  & $\LTCbot$  \\
  \end{tabular}
  \caption{Carley Tables for $\RLimp$ and $\RLneg{\QTargument}$ ($\RLzero = \LTCbot$)}
  \end{center}
\end{table}

\begin{example} [two elements residual lattices]
$(\set{\LTCbot,\LTCtop},\vee,\LTCbot,\wedge,\LTCtop,\RLimp,\RLotimes,\RLone)$.
$\RLone$ and $\LTCbot$ are required to be distinct because of residual law: 
we have $\LTCtop \RLotimes \RLone = \LTCtop$  whereas residual law forces 
$\LTCtop \RLotimes \LTCbot = \LTCbot$. 
So the only possibility is to set $\RLone = \LTCtop$;  i.e. weakening law holds.
We also have $\LTCtop \RLotimes \LTCtop = \LTCtop$ and $\LTCbot \RLotimes \LTCbot = \LTCbot$;
i.e. contraction law holds. 
Hence this is a Heyting algebra(Proposition~\ref{RL:WeakeningContraction}).
If we add linear negation, there are two cases. 
\begin{enumerate}
\item {\em $\RLzero = \LTCbot$}: In this case, the linear negation is the standard definition 
of negation ($\RLneg{a} = a \RLimp \LTCbot$) in Heyting algebras. 
Since  $\RLneg{\LTCtop} = \LTCbot$ and $\RLneg{\LTCbot} = \LTCtop$, 
both elements satisfies $a = \dRLneg{a}$. 
That is, this is a well known two elements boolean algebra. 
\item {\em $\RLzero = \LTCtop$}: 
This is a two element lattice with $\RLneg{\LTCtop} = \RLneg{\LTCbot} = \LTCtop$.
\end{enumerate}
\end{example}

\begin{example} [three elements residual lattices]

There are two cases according to the choice of $\RLone$. 
\begin{enumerate}
\item {\em $m = \RLone$}:
\begin{center}
   \begin{tabular}{c|ccc}
  $\RLotimes$ & $\LTCbot$ &  $\RLone$  & $\LTCtop$ \\ \hline
  $\LTCbot$  &  $\LTCbot$ & $\LTCbot$  & $\LTCbot$   \\
  $\RLone$  &     & $\RLone$  & $\LTCtop$  \\
  $\LTCtop$   &     &     & $\LTCtop$  \\
\end{tabular}
\hskip 48pt
\begin{tabular}{c|ccc}
  $\RLimp$   & $\LTCbot$ & $\RLone$ & $\LTCtop$  \\ \hline
  $\LTCbot$  & $\LTCtop$ & $\LTCtop$ & $\LTCtop$ \\
  $\RLone$   & $\LTCbot$ & $\RLone$ & $\LTCtop$ \\
  $\LTCtop$  & $\LTCbot$ & $\LTCbot$ & $\LTCtop$ \\
\end{tabular}
\end{center}
%%
\item {\em $\RLone = \LTCtop$}:  i.e. weakening law holds.
\begin{center}
\begin{tabular}{c|ccc}
  $\RLotimes$ & $\LTCbot$ &  $m$  & $\RLone$ \\ \hline
  $\LTCbot$  &  $\LTCbot$ & $\LTCbot$  & $\LTCbot$   \\
  $m$  &     & ?  & $m$  \\
  $\RLone$   &     &     & $\RLone$  \\
\end{tabular}
\end{center}
$m \leq \RLone$ (weakening) means $m \RLotimes m \leq m$, hence there are two sub-cases;
\begin{enumerate}
%
\item {\em $m \RLotimes m = \LTCbot$}:
\begin{tabular}{c|ccc}
  $\RLotimes$ & $\LTCbot$ &  $m$  & $\RLone$ \\ \hline
  $\LTCbot$  &  $\LTCbot$ & $\LTCbot$  & $\LTCbot$   \\
  $m$  &     & $\LTCbot$  & $m$  \\
  $\RLone$   &     &     & $\RLone$  \\
\end{tabular}
\hskip 48pt
\begin{tabular}{c|ccc}
  $\RLimp$   & $\LTCbot$ & $m$ & $\RLone$  \\ \hline
  $\LTCbot$  & $\RLone$ & $\RLone$ & $\RLone$ \\
  $m$        & $m$ & $\RLone$ & $\RLone$ \\
  $\RLone$  & $\LTCbot$ & $m$ & $\RLone$ \\
\end{tabular}
%
\item {\em $m \RLotimes m = m$}: i.e. contraction law holds.
This is a three element Heyting algebra.
\begin{tabular}{c|ccc}
  $\RLotimes$ & $\LTCbot$ &  $m$  & $\RLone$ \\ \hline
  $\LTCbot$  &  $\LTCbot$ & $\LTCbot$  & $\LTCbot$   \\
  $m$  &     & $m$  & $m$  \\
  $\RLone$   &     &     & $\RLone$  \\
\end{tabular}
\hskip 48pt 
\begin{tabular}{c|ccc}
  $\RLimp$   & $\LTCbot$ & $m$ & $\RLone$  \\ \hline
  $\LTCbot$  & $\RLone$ & $\RLone$ & $\RLone$ \\
  $m$        & $\LTCbot$ & $\RLone$ & $\RLone$ \\
  $\RLone$  & $\LTCbot$ & $m$ & $\RLone$ \\
\end{tabular}
\end{enumerate}
\end{enumerate}
\end{example}

\begin{example} [four elements residual lattice with weakening]
\begin{enumerate}
\item {\em with contraction, i.e., 4-elemnents Heyting Algebra}
There are two cases; one is  total order and the other is not. 
\begin{figure}[ht]
\begin{tikzpicture}[description/.style={fill=white,inner sep=2pt}]
\matrix (m) [matrix of math nodes, row sep=1.5em,
column sep=0.3em, text height=1.5ex, text depth=0.25ex]
{ 
& \top &\\
a & & b  \\
& \bot &\\
};
\path[-] 
(m-2-1) edge (m-1-2)
		edge (m-3-2)
(m-2-3) edge (m-1-2)
		edge (m-3-2) ;
\end{tikzpicture}

\begin{tikzpicture}[description/.style={fill=white,inner sep=2pt}]
\matrix (m) [matrix of math nodes, row sep=1.5em,
column sep=0.3em, text height=1.5ex, text depth=0.25ex]
{ 
& \top  \\
& a \\
& b \\
& \bot \\
};
\path[-] 
(m-3-2) edge (m-4-2)
(m-2-2) edge (m-1-2) edge (m-3-2) ;
\end{tikzpicture}

\caption{Example: four elemnts Heyting Algebras}
\label{RL:EX01}
\end{figure}


\item 
\end{enumerate}
\end{example}

\begin{example} [four elements residual lattice with weakening]
\end{example}