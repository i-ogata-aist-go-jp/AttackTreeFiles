\def\fCenter{\leq}
\subsection{Order Theory}

\begin{definition}[partial order]
Let $A$ be a set. A {\em partial order} on $A$ is a binary relation $\leq$ which is
\begin{enumerate}
\item {\em reflexive}: for all $a \in A$,  $a \leq a$,
\item {\em transitive}:  if $a \leq b$ and $b \leq c$, then $a \leq c$, and
\item {\em antisymmetric}: if $a \leq b$ and $b \leq a$, then $a = b$.
\end{enumerate}
\end{definition}

The reflexive rule is axiom, while the transitive and the antisymmetrc rules are inference rules. In this article, these axioms and inference rules are represented as; 
\begin{center}
\begin{prooftree}
	\AxiomC{} \UnaryInfC{$a \leq a$}
\DisplayProof  \hskip 2cm
	\AxiomC{$a \leq b$} \AxiomC{$b \leq c$}
    \BinaryInfC{$a \leq c$}
\DisplayProof  \hskip 2cm
	\AxiomC{$a \leq b$} \AxiomC{$b \leq a$}
    \BinaryInfC{$a = b$}
\end{prooftree}
\end{center}

\begin{definition} [poset and its opposite]
Let $A$ be a set and $\leq$ be a partial order. 
A {\em poset} (short for partially ordered set) is a double  $(A,\leq)$.
%
Its {\em opposite}, denoted $(A^{\sf op},\geq)$, 
is the poset with the same underlying set, with $a \geq b$ in $A^{\sf op}$ 
iff $a \leq b$ in $A$. 
\end{definition}

\begin{definition}[upper bound]
Let $(A,\leq)$ be a poset, $S$ a subset of $A$. 
We say an element $a$ is an {\em upper bound} for $S$
if $s \leq a$  for all $s \in S$, and write $S \leq a$.
Similarly, $a$ is a lower bound for S, and write $a \leq S$. 
\end{definition}

\begin{definition}[join]
Let $(A,\leq)$ be a poset, $S$ a subset of $A$. 
We say an element $a \in A$ is a {\em join} (or {\em least upper bound})  
for $S$, and write $a = \bigvee S$. The following conditions hold. 
\begin{enumerate}
\item {\em upper bound }:  $S \leq \bigvee S$, i.e. $\bigvee S$ is an upper bound for $S$.
\item {\em least upper bound }: $\bigvee S$ is the least upper bound, 
i.e. $\bigvee S \leq b$  for every upper bound $b$ for $S$. 
Otherwise said, if $S \leq b$ then $\bigvee S \leq b$. 
\end{enumerate}
\end{definition}

The antisymmetry axiom ensures that the join of $S$, if it exists, is unique. 
If $S$ is a two-element set $\set{s,t}$, we write $s \vee t$ for $\bigvee\set{s,t}$. 
If $S$ is the empty set $\emptyset$, we write $\bot$ for $\bigvee\emptyset$. 
$\bot$ is just the least element of $A$,
because every $c \in A$ is an upper bound for $\emptyset$, i.e. $\bot \leq c$. 

\begin{definition}[bounded join-semilattice-ordered poset]
A bounded join-semilattice-ordered poset is a poset $(L,\leq)$ such that;
\begin{enumerate}
\item  each two-element subset {a,b} has a join $a \vee b$, and
\item  the empty set $\emptyset$  has a join, and write $\bot = \bigvee\emptyset$.
\end{enumerate}
\end{definition}

The axioms and the inference rule for join-semilattice-ordered poset are pictorially represented as;
\begin{center}
\begin{prooftree}
    \AxiomC{}  \UnaryInfC{$ S \fCenter \bigvee S$}
  \DisplayProof \hskip .5cm
		\AxiomC{$S \fCenter c$} \UnaryInfC{$\bigvee S \fCenter c$}
  \DisplayProof \hskip .5cm
	\AxiomC{}  \UnaryInfC{$ a \fCenter a \vee b$}
  \DisplayProof \hskip .5cm
    \AxiomC{}  \UnaryInfC{$ b \fCenter a \vee b$} 
  \DisplayProof \hskip .5cm
		\AxiomC{$a \fCenter c$}
    	\AxiomC{$b \fCenter c$}
    \BinaryInfC{$a \vee b \fCenter c$}
  \DisplayProof \hskip .5cm
  		\AxiomC{}  \UnaryInfC{$\bot \fCenter c$}
\end{prooftree}
\end{center}

Dually, in any poset we can consider the notion of {\em meet} (greater lower bound)
and meet-semilattice-ordered poset. 
We write $\bigwedge S, a \wedge b, \top = \bigwedge\emptyset$ 
for the analogues of $\bigvee S, a \vee b, \bot$. 
Specifically, $\top$ is the greatest element of $L$. 
The rules for meet are represented as;
\begin{center}
\begin{prooftree}
  \AxiomC{}  \UnaryInfC{$ \bigwedge S \fCenter S$}
   \DisplayProof \hskip .5cm
		\AxiomC{$c \fCenter S$} \UnaryInfC{$c \fCenter \bigwedge S $}
   \DisplayProof \hskip .5cm	
	\AxiomC{}  \UnaryInfC{$ a \wedge b \leq a$}
   \DisplayProof \hskip .5cm
	\AxiomC{}  \UnaryInfC{$ a \wedge b \leq b$}
   \DisplayProof \hskip .5cm
	\AxiomC{$c \leq a$ \hskip .5cm  $c \leq b$}
    \UnaryInfC{$c \leq a \wedge b$}
   \DisplayProof \hskip .5cm
     \AxiomC{}  \UnaryInfC{$ c \leq \top$}
\end{prooftree}
\end{center}

\subsection{algebraic structure}

An algebraic structure is a set (called carrier set or underlying set) 
with one or more finitary operations defined on it 
that satisfies a list of axioms.

% magma
\begin{definition} [magma]
A magma is a basic kind of algebraic structure. 
It consists of a underlying set $M$ 
accompanied with a binary operation "$\cdotbin$" which is a map: 
$ {\QTargument} { \cdotbin} {\QTargument} : M \times M \longrightarrow M$.  That is, it sends any two elements $a,b \in M$ to another element $a \cdotbin b$ .
%
In set theoretic notation:
\[ \forall a,b \in M    ,      a \cdot b \in M \]
\end{definition}

In this article, we only handle commutative binary operation; 
i.e., $ a \cdot b = b \cdot a$. 

\begin{definition}[commutative monoid]
A commutative {\it monoid} is a triple 
$(M,\cdotbin,\RLone) $  where $\RLone \in M$. 
The following conditions hold:
%
\begin{enumerate} 
\item {\em neutrality law}:  for all  $a \in M$  we have $\RLone \cdotbin a = a$,
\item {\em associative law}:  for every $a,b,c \in M$,
we have $a \cdotbin ( b \cdotbin c) =  (a \cdotbin b) \cdotbin c$,
\item {\em commutative law}:  for every $a,b \in M$ we have $a \cdotbin b = b \cdotbin a$.
\end{enumerate}
\end{definition} 
%
A semilattices is a commutative monoid in which every element is idempotent. 
%
\begin{definition}[semilattice]
A {\it semilattice} is a triple $(M,\cdotbin,\RLone) $.
The following conditions hold:
%
\begin{enumerate} 
\item {\em commutative monoid}:  $(M,\cdotbin,\RLone)$ is a commutative monoid,
\item {\em idempotent law}:  for every $a \in M$  we have $a \cdotbin a = a$.
\end{enumerate}
\end{definition}

\begin{proposition}
A bounded join- and meet-semilattice-ordered poset is a semilattice.
\end{proposition}
\begin{proof}
Let $(L,\leq)$ be a join-semilattice-ordered poset.
\begin{enumerate}
\item The nutrality law holds because $\bot$ is the least element of $L$. 
\item It is obvious that the associative, commutative and idempotent law hold.
\end{enumerate}
Similarly, a bounded meet-semilattice-ordered poset is also a semilattice.
\end{proof}
%
The converse also holds; we can induce a partial order on a semilattice.
%
\begin{theorem} [A semilattice and its induced partial order]
\label{basicTheorem}
Let $(L,\vee,\bot)$   be a semillatice. 
Then there exists a unique partial order on $L$ 
such that $a \vee b$ is the join of $\set{a,b}$, and
$\bot$ is the least element of $L$. 
\end{theorem}
%
\begin{proof}
Clearly, if such a partial order exists, we must have $a \leq b$ iff $a \vee b = b$. 
So we take this as a definition of $\leq$. 
The definition is pictorially represented as;
%
\begin{prooftree}
	\AxiomC{$a \leq b$}
    \doubleLine
    \UnaryInfC{$a \vee b = b$}
\end{prooftree}
%
\begin{enumerate}
\item {\em $\leq$ is a partial order on $L$} ---
We deduce reflexiveness 
from idempotency: i.e. $a \vee a = a$, and 
antisymmetry  from $a = a \vee b = b$. 
To show transitivity, suppose $a \leq b$ and $b \leq c$;
i.e. $a \vee b = b$ and $b \vee c = c$. 
Then we have $a \leq c$ as follows: 
\[ a \vee c = a \vee (b \vee c) = (a \vee b) \vee c = b \vee c = c \mbox{.}\]
%
Pictorially,
\begin{prooftree}
	\AxiomC{$a \vee a = a$} \UnaryInfC{$a \leq a$}
\DisplayProof \hskip 1.5cm
	\AxiomC{$a \leq b$} \UnaryInfC{$a \vee b = b$}
    \AxiomC{$b \leq a$} \UnaryInfC{$b \vee a = a$}
    \BinaryInfC{$a = b$}
\DisplayProof \hskip 1.5cm
	\AxiomC{$a \leq b$} \UnaryInfC{$a \vee b = b$}
    \AxiomC{$b \leq c$} \UnaryInfC{$b \vee c = c$}
    \BinaryInfC{$a \vee c = c$}
    \UnaryInfC{$a \leq c$}
\end{prooftree}
%
\item {\em $a \vee b$ is the join of $\set{a,b}$} ---
%
We deduce $a \vee b$ is an upper bound for both $a$ and $b$ as follows: 
$a \vee (a \vee b) = (a \vee a) \vee b = a \vee b$; i.e.  $a \leq a \vee b$, and 
similarly  $b \leq a \vee b$ by commutative law. 
%
To show it is the least upper bound,
suppose $a \leq c$ and $b \leq c$: i.e. $a \vee c = c$ and $b \vee c = c$. 
Then we have $a \vee b \leq c$ as follows. 
\[  (a \vee b) \vee c = a \vee (b \vee c) = a \vee c = c \mbox{.}\]
\item {\em $\bot$ is the least element} --- 
Neutrality law:  $a \vee \bot = a$ says immediately $\bot \leq a$ for all $a \in L$. 
\end{enumerate}
\end{proof}

The theorem says 
that the two definitions of a semilattice are equivalent. 
One may freely use both style in constructing proofs.

\begin{proposition}\label{joinStability}
In a join-semilattice-ordered poset, 
if $a \leq b$ and $c \leq d$ then $\LTCjoinac \leq \LTCjoinbd$.
\end{proposition}
\begin{proof}
We show two different proofs according to the Theorem~\ref{basicTheorem}.

(1)  A equationally presentable proof.
We have  $\LTCjoinab = b$ and $\LTCjoincd = d$
then
$(\LTCjoinac) \LTCjoin (\LTCjoinbd) 
= (\LTCjoinab) \LTCjoin (\LTCjoincd) 
= \LTCjoinbd $ showing $\LTCjoinac \leq \LTCjoinbd$. 

(2) A proof in join-semilattice-ordered poset. 
\begin{prooftree}
			\AxiomC{$a \fCenter b$}  \AxiomC{$b \fCenter \LTCjoinbd$}
    	\BinaryInfC{$a \fCenter \LTCjoinbd$}
        	\AxiomC{$c \fCenter d$}  \AxiomC{$d \fCenter \LTCjoinbd$}
    	\BinaryInfC{$c \fCenter \LTCjoinbd$}
        \BinaryInfC{$ a \vee c \fCenter \LTCjoinbd$}
\end{prooftree}
\end{proof}

\begin{definition} [lattice]
A {\em lattice} is a quintuple $(L,\LTCjoin,\LTCbot,\LTCmeet,\LTCtop)$.
The following conditions hold.
\begin{enumerate}
\item {\em join-semilattice}: $(L,\LTCjoin,\LTCbot)$ is a join-semilattice,
\item {\em meet-semilattice}: $(L,\LTCmeet,\LTCtop)$ is a meet-semilattice,
\item {\em absorptive law}:  for every $a,b \in L$ we have $a \LTCjoin(a \LTCmeet b) = a$  and  $a \LTCmeet (a \LTCjoin b) = a$. 
\end{enumerate}
\end{definition}

The absorptive laws distinguish a lattice from an arbitrary pair of semilattices. It assure that the partial orders on $L$ induced by the two semilattices are opposite to each other.
%%%

\begin{definition} [lattice homomorphisms]
\begin{enumerate}
\item 
A {\em lattice homomorphism} is a function 
$f: A \longrightarrow B$ such that:
\[  f(\bot) = \bot  \hskip 2cm \mbox{and} \hskip 2cm
f(a \LTCjoin b) = f(a) \LTCjoin f(b) \]
\[  f(\LTCtop) = \LTCtop  \hskip 2cm \mbox{and} \hskip 2cm
f(a \LTCmeet b) = f(a) \LTCmeet f(b) \]
\item
A {\em lattice isomorphism} is a one-to-one and onto lattice homomorphism. 
\end{enumerate}
\end{definition}
%
\begin{definition} [order-preserving functions and order isomorphisms]
Let $(A,\leq_A)$ and $(B,\leq_B)$ are posets. 
\begin{enumerate}
\item An {\em order-preserving function} of posets is a function 
$g: A \longrightarrow B$ such that the order is preserved. 
\item 
An {\em order-isomorphism} of posets is a one-to-one and onto function 
$g: A \longrightarrow B$ such that the order is preserved bi-directionally. 
\end{enumerate}
Pictorially, 
\begin{prooftree}
	\def\fCenter{\leq_A}
	\AxiomC{$a \fCenter b$}
    \def\fCenter{\leq_B}
    \UnaryInfC{$ g(a) \fCenter g(b) $}
    \def\fCenter{\leq}
\DisplayProof \hskip 1.5cm 
	\def\fCenter{\leq_A}
	\AxiomC{$a \fCenter b$}
    \doubleLine
    \def\fCenter{\leq_B}
    \UnaryInfC{$ g(a) \fCenter g(b) $}
    \def\fCenter{\leq}
\end{prooftree}
\end{definition}
%
\begin{proposition} 
If $g$ is an order isomorphism then its inverse $g^{-1}$ is also an order isomorphism. 
\end{proposition}
\begin{proof} \hskip -5cm
\begin{prooftree}
	\def\fCenter{\leq_B}
	\AxiomC{$g(g^{-1}(a)) = a \fCenter b = g(g^{-1}(b)) $}
    \doubleLine
    \def\fCenter{\leq_A}
    \UnaryInfC{$ g^{-1}(a) \fCenter g^{-1}(b) $}
    \def\fCenter{\leq}
\end{prooftree}
\end{proof}
%
\begin{proposition}\label{LTC:OPF}
Every lattice homomorphisms are order-preserving functions.
Moreover, every lattice isomorphisms are order-isomorphisms.
\end{proposition}
\begin{proof}
Let $f$ be a lattice homomorphism, $g$ be a lattice isomorphism.
\begin{prooftree}
	\AxiomC{$ a \fCenter b $}
    \UnaryInfC{$ a \vee b = b$}
    \UnaryInfC{$ f(a \vee b) = f(a) \vee f(b) = f(b) $}
    \UnaryInfC{$f(a) \leq f(b)$}
   \DisplayProof \hskip 2cm 
	\AxiomC{$ a \fCenter b$}
	\doubleLine
	\UnaryInfC{$ a \LTCjoin b = b$}
	\doubleLine
	\UnaryInfC{$ g(a) \LTCjoin g(b) = g(a \LTCjoin b) = g(b)$}
	\doubleLine
	\UnaryInfC{$ g(a) \fCenter g(b)$}
\end{prooftree}
\end{proof}
%
However the converse of the former is not true; 
an order-preserving function is not
a lattice homomorphism in general.
%
\begin{figure}[ht]
\begin{tikzpicture}[description/.style={fill=white,inner sep=2pt}]
\matrix (m) [matrix of math nodes, row sep=1.5em,
column sep=0.3em, text height=1.5ex, text depth=0.25ex]
{ 
& \top & &&&\longrightarrow& \top_B\\
a & & b  &&&\longrightarrow& m \\
& \bot & &&&\longrightarrow& \bot_B \\
};

\path[-] 
(m-2-1) edge (m-1-2)
		edge (m-3-2)
(m-2-3) edge (m-1-2)
		edge (m-3-2)
(m-2-7) edge (m-1-7)
		edge (m-3-7) ;
\end{tikzpicture}

\caption{Example: an order-preserving function is not
a lattice homomorphism}
\label{LCT:CE1}
\end{figure}
%
\begin{example}


Let $(\set{\bot,a,b,\top},\vee,\bot,\wedge,\top)$ and
$(\set{\bot_B,m,\top_B},\vee_B,\bot_B,\wedge_B,\top_B)$ be lattices(See Figure~\ref{LCT:CE1}).
An order preserving function $f(\bot) = \bot_B, f(\top) = \top_B, 
f(a) = f(b) = m$ is not a homomorphism, 
for $f(a \vee b) = f(\top) = \top_B$ is not equal to 
$f(a) \vee_B f(b) = m \vee_B m = m$. 
\end{example}
%
%
%
The converse of the latter is true. 
\begin{proposition} \label{LTC:latticeIsomorphism}
Let $A$ and $B$ are lattices. 
A one-to-one and onto function $f : A \longrightarrow B$ is an order-isomorphism 
iff it is a lattice isomorphism.
\end{proposition}

\begin{proof}
($\Leftarrow$) The latter of Proposition~\ref{LTC:OPF}.

($\Rightarrow$) 
\begin{enumerate}
\item $f(\bot_A) = \bot_B$: 
Since $f$ is onto, there exists some $a \in A$ 
such that $f(a) = \bot_B$. Then,
\begin{prooftree}
\AxiomC{$ f(a) \fCenter f(\bot_A) $}
\UnaryInfC{$ a \fCenter \bot_A$}
\end{prooftree}

\item $f(a \vee b) = f(a) \vee f(b)$: 
There also exists some $c$ 
such that $ f(c) = f(a) \LTCjoin f(b)$. Then,
\begin{prooftree}
\AxiomC{$a \fCenter a \LTCjoin b$}
\UnaryInfC{$f(a) \fCenter f(a \LTCjoin b)$}
\AxiomC{$b \fCenter a \LTCjoin b$}
\UnaryInfC{$f(b) \fCenter f(a \LTCjoin b)$}
\BinaryInfC{$f(a) \LTCjoin f(b) \fCenter f(a \LTCjoin b)$}
%
\AxiomC{$f(a) \fCenter f(c)$}
\UnaryInfC{$a  \fCenter c$}
\AxiomC{$f(b) \fCenter f(c)$}
\UnaryInfC{$b  \fCenter c$}
\BinaryInfC{$a \LTCjoin b \fCenter c$}
\UnaryInfC{$f(a \LTCjoin b) \fCenter f(c)$}
%
\BinaryInfC{$f (a \LTCjoin b) = f(a) \LTCjoin f(b)$}
\end{prooftree}
\end{enumerate}
\end{proof}

%
% distributive lattice
%
Since lattices come with two binary operations, 
it is natural to ask whether one of them distributes over the other, 
i.e. whether one or the other of the following dual laws holds 
for every three elements $a, b, c$ of $L$:

\begin{enumerate}
\item {\em Distributivity of $\LTCjoin$ over $\LTCmeet$}:
$ c \LTCjoin (a \LTCmeet  b) = (c \LTCjoin a) \LTCmeet   (c \LTCjoin b)$
\item {\em Distributivity of $\LTCmeet$   over $\LTCjoin$}:
$c \LTCmeet  (a\LTCjoin b) = (c\LTCmeet  a) \LTCjoin  (c\LTCmeet  b)$
\end{enumerate}

\begin{definition}[distributive lattice]\label{distributiveLattice}
A lattice $(L,\LTCjoin,\LTCmeet,\LTCbot,\LTCtop)$ is a distributive lattice 
if it satisfies the avobe two {\em distributive laws}.
\end{definition}

The following definition and lemma shows that one is enough 
to define distributivity; the other (its dual) is automatically holds. 

\begin{lemma}[dual of distributive law holds]
If the distributive law holds in a lattice,  then so does its dual.
\end{lemma}
\begin{proof}
\begin{eqnarray*}
   (c \LTCjoin a) \LTCmeet (c \LTCjoin b)
= (c \LTCmeet c) \LTCjoin (c \LTCmeet b) \LTCjoin  (a \LTCmeet c) \LTCjoin (a \LTCmeet b) 
&=&  c \LTCjoin (a \LTCmeet b)  \\
   (c \LTCmeet a) \LTCjoin (c \LTCmeet b)
=  (c \LTCjoin c) \LTCmeet (c \LTCjoin b) \LTCmeet  (a \LTCjoin c) \LTCmeet (a \LTCjoin b) 
&=&  c \LTCmeet (a \LTCjoin b)
\end{eqnarray*}
\end{proof}

\subsection{Boolean Algebras} 

\begin{definition}[boolean algebra]
A {\em  Boolean algebra} is a sextuple 
$(L,\LTCjoin,\LTCbot,\LTCmeet,\LTCtop,\lnegP{\mbox{\textendash}})$.
The following conditions hold:
\begin{enumerate}
\item {\em distributive lattice}: 
$(L,\LTCjoin,\LTCbot,\LTCmeet,\LTCtop)$ is a distributive lattice,
\item  {\em complement law}: for each element $a$, 
there exists at most one  $\lneg{a}$  satisfying  
$a \LTCjoin \lneg{a} = \LTCtop$ and $a \LTCmeet \lneg{a} = \LTCbot$.  
\end{enumerate}
\end{definition}

\section{Heyting  Algebras}

\begin{definition}[heyting algebra]
A {\em  Heyting algebra} is a sextuple 
$(L,\LTCjoin,\LTCbot,\LTCmeet,\LTCtop,\RLimp)$.
The following conditions hold:
\begin{enumerate}
\item {\em lattice}: 
$(L,\LTCjoin,\LTCbot,\LTCmeet,\LTCtop)$ is a lattice,
\item  {\em residual law}: for each pair of elements $(a,b)$, 
there exists an element $a \LTCimp b$  such that  
$c \fCenter a \LTCimp b$ iff $a \LTCmeet c \fCenter b$.  
The inference rule can also be used as upside down.  Pictorially,
\hskip -7cm
\begin{prooftree}
	\AxiomC{$c \LTCmeet a \fCenter b$}
    \UnaryInfC{$c \fCenter a \RLimp b$}
 \DisplayProof \hskip 96pt   
    \AxiomC{$c \fCenter a \RLimp b$}
    \UnaryInfC{$c \LTCmeet a \fCenter b$}
\end{prooftree}
\end{enumerate}
\end{definition}

% complete join-semilattice
%
\begin{definition} [complete join-semilattices and lattices]
A poset $(L,\leq)$ is called a  {\em complete join-semilattice} 
if every subset $A$ of $L$ has a join.  
Taking $A=L$ shows that every complete join-semilattice has a greatest element
$\bigvee L = \top$.
A complete meet-semilattice is a dual of complete join semi-lattice. 
A poset $(L,\leq)$ is called a  {\em complete lattice}  if every subset $S$ of $L$ has a join and a meet. 
\end{definition}
%
% complete join-semilattice is a complete lattice
%
\begin{proposition} [complete join-semilattice is a complete lattice] \label{LTC:completeLattice}
For any subset $A$ of $L$, consider the set  $X$ of lower bounds for $A$. 
Let
\[ X  =  \set{ x \in L | x \leq A } \]
and set
\[ \bigwedge A = \bigvee X \]
\end{proposition}
\begin{proof}

\begin{enumerate}
\item  {$\bigvee X  \leq A$}: 
$X \leq a \Rightarrow \bigvee X \leq a$(for all $a \in A$),
hence $\bigvee X \leq A$. 
%
\item  {$y \leq A \Rightarrow y \leq \bigvee X$}: 
Clearly $y \in X \Rightarrow y \leq \bigvee X$. 
\end{enumerate}

\end{proof}

\subsection{Commutative Residuated Lattices}

A book dedicated to residuated lattices
\cite{zbMATH05611027}\cite{isbnplus9780444521415}.

\newcommand{\RLleft}{a \RLotimes c \fCenter b}
\newcommand{\RLright}{c \fCenter a \RLimp b}

\begin{definition}[commutative residuated lattice]
A commutative {\em  residuated lattice} is a octaple 
$(L,\vee,\bot,\wedge,\top,\RLimp,\RLotimes,\RLone)$.
The following conditions hold:
\begin{enumerate}
\item {\em lattice}: 
$(L,\vee,\bot,\wedge,\top)$ is a lattice,
\item {\em commutative monoid}: 
$(L,\RLotimes,\RLone)$ is a commutative monoid,
\item  {\em residual law}: for each pair of elements $(a,b)$, 
there exists an element $a \RLimp b$  such that residual law holds:
\hskip -7cm
\begin{prooftree}
	\AxiomC{$\RLleft$}
    \doubleLine
    \UnaryInfC{$\RLright$}
\end{prooftree}
\end{enumerate}
\end{definition}

%%%
\begin{proposition} [stability of residual operators]
\label{RL:stability}
If $a \leq b$ and $c \leq d$, 
then 
$ a \RLotimes c \fCenter b \RLotimes d$
and
$ b \RLimp c \fCenter a \RLimp d$.
\end{proposition}

\begin{proof}
Pictorially; 
\begin{prooftree}
	\AxiomC{$a \fCenter b $}
    		\AxiomC{$c \fCenter d $} 
			\AxiomC{$ \RLotimesbd \fCenter \RLotimesbd $}	
    		\UnaryInfC{$ d \fCenter b \RLimp  (\RLotimesbd)$}
    	\BinaryInfC {$c \fCenter b \RLimp (\RLotimesbd)$}
    	\UnaryInfC{$ b \fCenter  c \RLimp (\RLotimesbd)$}
    \BinaryInfC{$a \fCenter  c \RLimp (\RLotimesbd)$}
    \UnaryInfC{ $\RLotimesac\fCenter \RLotimesbd$}
    \DisplayProof \hskip .1cm
	\AxiomC{$a \fCenter b$}
        	\AxiomC{$\RLimpbc \fCenter \RLimpbc$}
    		\UnaryInfC{$b \RLotimes (\RLimpbc) \fCenter c$}
    		\AxiomC{$c \fCenter d$}
    	\BinaryInfC{$b \RLotimes (\RLimpbc) \fCenter d$}
    	\UnaryInfC{$b \fCenter (\RLimpbc) \RLimp d$}
    \BinaryInfC{$a \fCenter (\RLimpbc) \RLimp d$}
    \UnaryInfC{$\RLimpbc \fCenter \RLimpad$}
\end{prooftree}
\end{proof}

\begin{fact}
$ \LTCtop \RLotimes \LTCtop = \LTCtop$, 
since $ \RLone \leq \LTCtop$ and 
$\LTCtop \RLotimes \RLone = \LTCtop \leq \LTCtop \RLotimes \LTCtop$. 
\end{fact}

The residual law exactly says that
the operation $a \RLimp \QTargument$ is the right adjoint to 
$\QTargument \RLotimes a$.

\[  a  \RLotimes  \QTargument  \dashv   a \RLimp \QTargument  \]

It is known that adjointness can also be defined 
by using {\em unit} and {\em counit}.
%
%  unit and counit  
%
\newcommand{\RLunit}{c \fCenter a \RLimp (a \RLotimes c)}
\newcommand{\RLcounit}{a \RLotimes (a \RLimp b) \fCenter b}

\begin{proposition}
The following conditions on a residuated lattice $L$ are equivalent:
\begin{enumerate}
\item  The residual law holds.
\item  The unit ($\RLunit$) and counit ($\RLcounit$) axiom holds. \\
\end{enumerate}
\end{proposition}
\begin{proof}
%
(1) $\Rightarrow$ (2):
\begin{prooftree}
	\AxiomC{$a \RLotimes c \fCenter a \RLotimes c$}
    \UnaryInfC{$ \RLunit$}
\DisplayProof  \hskip  2cm
    \AxiomC{$a \RLimp b \fCenter a \RLimp b$}
    \UnaryInfC{$ \RLcounit $}
\end{prooftree}
%
(2) $\Rightarrow$ (1):
\begin{prooftree}
    \AxiomC{$ c \fCenter a \RLimp (a \RLotimes c)$}
    \AxiomC{$a \RLotimes c \fCenter b$}
	\UnaryInfC{$a \RLimp (a \RLotimes c) \fCenter a \RLimp b$}
    \BinaryInfC{$c \fCenter a \RLimp b$}
\DisplayProof
    \AxiomC{$c \fCenter a \RLimp b$}
	\UnaryInfC{$ a \RLotimes c \fCenter  a \RLotimes (a \RLimp b)$}
    \AxiomC{$ a \RLotimes (a \RLimp b) \fCenter b$}
    \BinaryInfC{$ a \RLotimes c \fCenter b$}
\end{prooftree}
\end{proof}

\begin{proposition} [A residuated lattice is distributive]
\label{RLdistibutive}
A function $ c \RLotimes (\mbox{\textendash})$ preserves joins.
Similarly, a function $ c \LLimp \QTargument$ preserves meets.
\[ c \LLotimes  (a \LTCjoin b) =  (c \LLotimes a) \LTCjoin (c \LLotimes b)
\hskip 2cm 
 c \LLimp (a \LTCmeet b) =  (c \LLimp a) \LTCmeet (c \LLimp b) \]
\end{proposition}

\begin{proof}
\begin{prooftree}
		\AxiomC{$c \fCenter c$}
		\AxiomC{$a \fCenter \LTCjoinab$}
	\BinaryInfC{$\RLotimesca \fCenter c \RLotimes (\LTCjoinab)$}
		\AxiomC{$c \fCenter c$}
		\AxiomC{$b \fCenter \LTCjoinab$}
	\BinaryInfC{$\RLotimescb \fCenter c \RLotimes (\LTCjoinab)$}
	\BinaryInfC{$\RLojcacb \fCenter c \RLotimes (\LTCjoinab)$}	
\end{prooftree}
\begin{prooftree}
	\AxiomC{$ \RLotimesca \fCenter \RLojcacb$}
		\UnaryInfC{$ a \fCenter c \RLimp (\RLojcacb)$}
	\AxiomC{$ \RLotimescb \fCenter \RLojcacb$}
		\UnaryInfC{$ b \fCenter c \RLimp (\RLojcacb)$}
	\BinaryInfC{$ \LTCjoinab \fCenter c \RLimp (\RLojcacb)$}
\UnaryInfC{$ c \RLotimes (\LTCjoinab) \fCenter \RLojcacb$}
\end{prooftree}
%%
\begin{prooftree}
		\AxiomC{$c \fCenter c$}  \AxiomC{$ \LTCmeetab \fCenter a$}
    \BinaryInfC{$c \RLimp (\LTCmeetab) \fCenter c \RLimp a$}
    	\AxiomC{$c \fCenter c$}  \AxiomC{$ \LTCmeetab \fCenter b$}
    \BinaryInfC{$c \RLimp (\LTCmeetab) \fCenter c \RLimp b$}
   	\BinaryInfC{$c \RLimp (\LTCmeetab) \fCenter \RLimcacb$}
\end{prooftree}
\begin{prooftree}
		\AxiomC{$\RLimcacb \fCenter \RLimpca$}
	\UnaryInfC{$c \RLotimes (\RLimcacb) \fCenter a$}
		\AxiomC{$\RLimcacb \fCenter \RLimpcb$}
	\UnaryInfC{$c \RLotimes (\RLimcacb) \fCenter b$}
	\BinaryInfC{$c \RLotimes (\RLimcacb) \fCenter \LTCmeetab$}
	\UnaryInfC{$\RLimcacb \fCenter c \RLimp (\LTCmeetab)$}
\end{prooftree}
\end{proof}

It follows directly that
the operation $a \RLimp \QTargument$ and $\QTargument \RLotimes a$
are order preserving maps. 

\newcommand{\RLbigwedge}{\bigvee ( c \RLotimes A) }
\newcommand{\RLbigimp}{\bigwedge ( c \RLimp A)}

\begin{proposition} [Adjoint Functor Theorem] \label{RL:infiniteDistributivity}
Let $L$ be a residuated lattice. 
Let $ c \RLotimes A = \set { c \otimes a | \forall a \in A}$. 
\begin{enumerate}
\item $c \RLotimes \QTargument$ preserves all joins which exist in $L$. 
%
\[  c \RLotimes \bigvee A  = \RLbigwedge  \]
%
\item $c \RLimp \QTargument$ preserves all meets which exist in $L$. 
\end{enumerate}
%
\[   c \RLimp  \bigwedge A = \RLbigimp  \]
%
\end{proposition}

\begin{proof}
\begin{enumerate}
\item
Let $A$ be a subset of $L$ (i.e. $A \subseteq L$) such that $\bigvee A$ exists.

\begin{prooftree}
	\AxiomC{$c \fCenter c$}  \AxiomC{$ A \fCenter \bigvee A$}
    \BinaryInfC{$ c \RLotimes A \fCenter c \RLotimes \bigvee A$}
    \UnaryInfC{$ \bigvee (c \RLotimes A) \fCenter c \RLotimes \bigvee A$}
\DisplayProof \hskip 3cm 
    \AxiomC{$ c \RLotimes A \fCenter \bigvee(c \RLotimes A) $}
    \UnaryInfC{$  A \fCenter   c  \RLimp \bigvee(c \RLotimes A)$ }
    \UnaryInfC {$ \bigvee A \fCenter  c  \RLimp \bigvee(c \RLotimes A) $}
    \UnaryInfC {$ c \RLotimes \bigvee A \fCenter \bigvee(c \RLotimes A) $}
\end{prooftree}

\item Let $A \subseteq L$ such that $\bigwedge A$ exists.

\begin{prooftree}
	\AxiomC{$c \fCenter c$}  \AxiomC{$ \bigwedge A \fCenter A$}
    \BinaryInfC{$c \RLimp \bigwedge A \fCenter c \RLimp A$}
    \UnaryInfC{$  c \RLimp \bigwedge A \fCenter \bigwedge (c \RLimp A)$}
\DisplayProof \hskip 3cm 
    \AxiomC{$ \bigwedge (c \RLimp A) \fCenter c \RLimp A $}
    \UnaryInfC{$ c \RLotimes \bigwedge (c \RLimp A) \fCenter A$ }
    \UnaryInfC {$ c \RLotimes \bigwedge (c \RLimp A) \fCenter  \bigwedge A $}
    \UnaryInfC {$ \bigwedge (c \RLimp A) \fCenter c \RLimp \bigwedge A $}
\end{prooftree}

\end{enumerate}

\end{proof}

\begin{fact} [anihilative law] \label{RL:anihilation}
For every $a \in L$, $a \QTotimes \bot = \bot$
\end{fact}
\begin{proof}
$a \QTotimes \bigvee\emptyset = \bigvee (a \QTotimes \emptyset)
= \bigvee\emptyset$
\end{proof}
%%

\subsection {Examples of commutative residual lattices}

\subsubsection{Commutative Semirings}

\begin{definition}[commutative semiring]
A commutative (i.e. $a \cdotbin b = b \cdotbin a$)
{\it semiring} is a quintuple $(R,+,0,\cdotbin,1) $.
The following conditions hold:
%
\begin{enumerate} 
\item {\em additive commutative monoid}:
$(R,+,0)$ is a commutative monoid with identity element $0$,
\item {\em multiplicative commutative monoid}:
$(R,\cdotbin,1)$ is a commutative monoid with identity element $1$, 
\item {\em anihilative law}:  $a \cdotbin 0 = 0$,
\item {\em distributive law}:
$\cdotbin$ distibutes over $+$;
\[ c \cdotbin (a + b) = (c+a) \cdotbin (c+b) \]
\end{enumerate}
\end{definition}

\begin{remark}
A commutative residuated lattice 
is an idempotent (i.e. $a + a = 0$) commutative semiring.
\end{remark}

\subsubsection{Standard Kleene Algebras}

\begin{definition} [Kleene algebra]
A {\em Kleene algebra} is an idempotent (and thus partially ordered) semiring 
endowed with a closure operator ${*}$.
\end{definition}
%
\begin{remark} [standard kleene algebras]
A commutative residuated lattice is a standard kleene algebra 
in the following sense.
We define an additional unary operator ${*}$; 
for each $a \in L$, 
let $S = \set{\RLone,a,a^2,\ldots}$ 
where $a^n = \underbrace{a \RLotimes \ldots \RLotimes a}_{n}$
and define $a^{*} \defeq \bigvee S$.
%
The following conditions hold,
\begin{center}
\begin{prooftree}
	\AxiomC{}
	\UnaryInfC{$\RLone \vee ( a \RLotimes a^{*} ) = a^{*}$}
\DisplayProof \hskip 2cm
	\AxiomC{$a \QTotimes b \leq b$}
    \UnaryInfC{$a^{*} \RLotimes b \leq b$}
\end{prooftree}
\end{center}
\end{remark}




\subsubsection{Unital Commutative Quantales}
A quantale is a join-semilattice $Q$
equipped with a multiplication which is distributive over arbitrary joins. 
 
%%
In this article, we only consider unital (i.e. having an identity element) commutative quantales. 
%%
\begin{definition}[unital commutative quantales]
A (unital commutative) {\em quantale} is a quadruple 
$(Q,\bigvee,\QTotimes,\RLone)$.
The following conditions hold:
\begin{enumerate}
\item {\em complete join-semilattice}:
$(Q,\bigvee)$ is a complete join-semilattice,
\item {\em commutative monoid}:
$(Q,{\QTotimes},\RLone)$ is a commutative monoid,
\item {\em infinite distributive law}: 
for any $a \in Q$, $a \QTotimes \QTargument$ preserves all joins in $Q$. 
\[ a \QTotimes \bigvee X = \bigvee (a \QTotimes X) \]
\end{enumerate}
\end{definition}
%%
\begin{proposition} [$a \QTotimes \QTargument$ is an order preserving map]
If $c \leq d$, then $ a \QTotimes c \leq a \QTotimes d$.
\end{proposition}
\begin{proof}
Recall that $c \leq d$ iff  $c \vee d = d$.
The claim is $ (a \QTotimes c) \vee (a \QTotimes d) = a \QTotimes d$. 
But this claim can be proved easily by distributivity: 
 $ a \QTotimes d
= a \QTotimes (c \vee d)
= (a \QTotimes c) \vee (a \QTotimes d)$.
\end{proof}

\begin{definition} [Implication]
For each $a,b \in Q$ set
\[ X  \defeq \set{x \in Q |  a \RLotimes x \leq b} \]
and set
\[ a \RLimp b \defeq \bigvee X \]
to produce an operation (implication) 
$ {\QTargument} { \RLimp } {\QTargument} : Q \times Q \RLimp Q$.
\end{definition}

\begin{proposition} [counit]
$a \QTotimes (a \QTmultimap b)  \leq b$ holds.
\end{proposition}
\begin{proof} \hskip -7cm
\begin{prooftree}
\AxiomC{$a \QTotimes X \fCenter b$}
\UnaryInfC{$\bigvee (a \QTotimes X) \fCenter b$}
\UnaryInfC{$a \QTotimes \bigvee X \fCenter b$}
\end{prooftree}
\end{proof}
%
\begin{proposition} [Adjoint Functor Theorem]
If $Q$ has all joins and $a \QTotimes \QTargument$ preserves them, 
it has a right adjoint $a \QTmultimap \QTargument$. 
Otherwise said, for any $a,b,x \in Q$, the residual law holds:
\begin{prooftree}
\AxiomC{$a \QTotimes x \fCenter b$}
\doubleLine
\UnaryInfC{$x \fCenter a \QTmultimap b$}
\end{prooftree}
\end{proposition}
\begin{proof}
\begin{enumerate}
\item
$a \QTotimes x \leq b$ means $x \in X$, hence $x \leq \bigvee X$.
\item
\begin{prooftree}
\AxiomC{$x \fCenter  a \RLimp b$}
\UnaryInfC{$ a \QTotimes x \fCenter a \QTotimes (a \RLimp b)$}
\AxiomC{$a \QTotimes (a \QTotimes b) \fCenter b$}
\BinaryInfC{$a \QTotimes x \fCenter b$}
\end{prooftree}
\end{enumerate}
$(\Rightarrow)$ 
Since $a \QTmultimap b = \bigvee X$ is an upper bound for $X$, 
it is obvious that any $x \in X$ is smaller than $\bigvee X$. 
$(\Leftarrow)$  
We already know that  $a \QTotimes (a \QTmultimap b)  \leq b$.
It directly follows from stability that 
$x \leq a \QTmultimap b$ implies $a \QTotimes x \leq b$. 
\end{proof}

Since $(Q,\bigvee)$ is a complete join-semilattice, 
it is a complete lattice(Proposition~\ref{LTC:completeLattice}).
The proposition says 
that every unital commutative quantale is a complete lattice 
in which residual law holds;
i.e. a unital commutative quantale is a (complete) commutative residuated lattice.

The converse is true; a complete commutative residuated lattice
is a (unital, commutative) quantale, 
since it satisfies infinite distributive law
(Proposition~\ref{RL:infiniteDistributivity}).

\subsection{Negations in residuated lattice}
In this subsection, we consider {\em negations} in residuated lattices.
First, we set a designated element $\RLzero$ called {\em zero}. 
%%

\begin{definition} [linear negation] \label{linearNegation}
Let $L$ be a residuated lattice. 
We specify an arbitrary designated element $\RLzero \in L$ and set 
\[ \RLneg a \defeq a \QTmultimap \RLzero  \] 
to produce an operation, called {\em linear negation} 
$ \RLneg{\QTargument}  : L  \rightarrow L$.
\end{definition}
%
\begin{proposition} \label{RL:orderReversing}
A {\em linear negation} 
$ \RLneg{\QTargument}  : L  \rightarrow L$ is an order-reversing function. 
\end{proposition}
\begin{proof}
This is clear from stability(Proposition~\ref{RL:stability}).
\begin{prooftree}
\AxiomC{$ a \fCenter b$}
\AxiomC{$ \RLzero \fCenter \RLzero$}
\BinaryInfC{$ b \RLimp \RLzero \fCenter a \RLimp \RLzero$}
\UnaryInfC{$ \RLnegB \fCenter \RLnegA$}
\end{prooftree}
\end{proof}
%%
\begin{proposition} \label{DeMorganEZ}
One side of De Morgan law holds. 
\[ 
	\RLneg{\RLone} = \RLzero
\hskip 48pt
	\RLneg{\LTCbot} = \LTCtop
\hskip 48pt
	\RLnegP{a \LTCjoin b} =  \RLnegA \LTCmeet \RLnegB
\]
\end{proposition}
\begin{proof}
It directly follows from residual law 
that $\RLneg a$ is the unique largest element $x$ satisfying 
$a \QTotimes x \leq \RLzero$. 
\begin{prooftree}
\AxiomC{$\RLnegA \leq \RLnegA$}
\UnaryInfC{$a \QTotimes \RLnegA \leq \RLzero$}
\DisplayProof \hskip 3cm 
\AxiomC{$a \QTotimes x \leq \RLzero$}
\UnaryInfC{$x \leq \RLnegA$}
\end{prooftree}
Put $a = \RLone$ and $a = \LTCbot$ in the above. Specifically, 
\begin{prooftree} 
	\AxiomC{$\RLneg{\RLone} \fCenter \RLneg{\RLone}$}
	\UnaryInfC{$ \RLone \RLotimes \RLneg{\RLone} \fCenter \RLzero$}
    \UnaryInfC{$ \RLneg{\RLone} \fCenter \RLzero$}
\DisplayProof \hskip 24pt
	\AxiomC{$\RLzero \fCenter \RLzero$}
    \UnaryInfC{$\RLone \RLotimes \RLzero \fCenter \RLzero$}
	\UnaryInfC{$\RLzero \fCenter \RLneg\RLone$}
\DisplayProof \hskip 48pt	
	\AxiomC{$\LTCbot \fCenter \RLzero$}
	\UnaryInfC{$\LTCbot \QTotimes \LTCtop \fCenter \RLzero$}
	\UnaryInfC{$\LTCtop \fCenter \RLneg{\LTCbot}$}
\end{prooftree}

\begin{prooftree}
\AxiomC{$a \fCenter a \LTCjoin b$} 
 \UnaryInfC {$\RLnegP{a \LTCjoin b} \fCenter \RLnegA $} 
\AxiomC{$b \fCenter a \LTCjoin b$} 
 \UnaryInfC {$\RLnegP{a \LTCjoin b} \fCenter \RLnegB $} 
  \BinaryInfC  {$\RLnegP{a \LTCjoin b} \fCenter \RLnegA \LTCmeet \RLnegB $}
%
\DisplayProof \hskip 48pt
%
\AxiomC {$\RLnegA \LTCmeet \RLnegB  \fCenter  \RLnegA$}
 \UnaryInfC{$a \fCenter \DeMorganMeet$}
\AxiomC {$\RLnegA \LTCmeet \RLnegB  \fCenter  \RLnegB$}
 \UnaryInfC{$b \fCenter \DeMorganMeet$}
 \BinaryInfC  {$a \LTCjoin b\fCenter \DeMorganMeet$}
 \UnaryInfC {$ \RLnegA \LTCmeet \RLnegB \fCenter \RLnegP{a \LTCjoin b} $}
\end{prooftree}
\end{proof}
%%
%%
\begin{definition}[residual lattice with double negation elimination]
An {\em residual lattice with double negation elimination} 
is a nonuple
$(L,\vee,\bot,\wedge,\top,\RLimp,\RLotimes,\RLone,\RLzero)$.
The following conditions hold:
\begin{enumerate}
\item {\em residuated lattice}:
$(L,\vee,\bot,\wedge,\top,\RLimp,\RLotimes,\RLone)$ is a residuated lattice,
\item {\em multiplicative zero}: 
$\RLzero \in L$ is a distinguished object,
\item {\em double negation elimination law}:
 $\dRLneg a = a$  for all $a \in L$(Definition~\ref{linearNegation}).
\end{enumerate}
\end{definition}
%
\begin{proposition}
In case double negation elimination law holds (i.e.  $a = \dRLnegA$), 
$\RLneg{\QTargument}$ is an order isomorphism.
Then $\RLneg{\QTargument}$ is a lattice homomorphism
from $(A,\leq)$ to its opposite $(A^{\sf op}, \geq)$(Proposition~\ref{LTC:latticeIsomorphism}).
\end{proposition}
\begin{proof}
We want to prove 
$\RLneg{\QTargument}$ is an order-reversing 
bijection of $L$ to itself. 
\begin{enumerate}
\item {\em order-reversing}: Proposition~\ref{RL:orderReversing}),
\item {\em one to one}: $\RLnegA = \RLnegB$ implies $a = \dRLnegA = \dRLnegB = b$,
\item {\em onto}: $\RLnegP{\RLnegB} = b$ ; 
i.e. $\RLneg{\QTargument}$  itself is an inverse function. 
\end{enumerate}
\end{proof}
%
The above proposition says 
that $\RLneg{\QTargument}$ carries joins into meets and vice versa; 
that is De Morgan Law holds. However, we can prove the De Morgan Law more directly.
%
\begin{proposition} [De Morgan Law]
The other side of De Morgan Law holds (Proposition~\ref{DeMorganEZ}) 
in residual lattices with double negation elimination.

\[
	\RLneg{\RLzero} = \RLone
\hskip  48pt 
	\RLneg{\LTCtop} = \LTCbot
\hskip 48pt
  \RLnegP{a \LTCmeet b} =  \RLnegA \LTCjoin \RLnegB \]
\end{proposition}

\begin{proof} 
\[ \RLneg{\RLzero} = \dRLneg{\RLone} = \RLone 
\hskip 48pt
 \RLneg{\LTCtop} = \dRLneg{\LTCbot} = \LTCbot \]
\[ \RLneg{(a \LTCmeet b)} = \RLneg{(\dRLneg{a} \LTCmeet \dRLneg{b})}
= \dRLneg {(\RLnegA \LTCjoin \RLnegB)} = \RLnegA \LTCjoin \RLnegB
\]
\end{proof}
%
% closure operators on residual lattices 
%
\begin{definition} [closure operators for residual lattices]
\label{RL:closureDef}
If $L$ is a residual lattice, 
a map $\closure{\QTargument} : L \longrightarrow L$
is called a {\em closure operator},
if for every $a,b\in L$ the following properties hold:
\begin{enumerate}
\item {\em extensivity} :  $a \leq \closureA$,
\item {\em idempotence} : $ \dclosureA \leq \closureA$ 
\hskip 48pt
(extensibity gives $\closureA \leq \dclosureA$, hence $\dclosureA = \closureA$),
\item {\em monotonicity} : $a \leq b$ implies $\closureA \leq \closureB$,
\item {\em monoid}:
$ \closureA  \RLotimes \closureB \leq \closure{a \RLotimes b}$.
\end{enumerate}
\end{definition}

\newcommand{\cLTCmeetAB} {\closureA \LTCmeet \closureB}
\newcommand{\cRLotimesAB}{\closureA \RLotimes \closureB}
\newcommand{\cRLimpAB}{\closureA \RLimp \closureB}

\begin{proposition}
The monoid condition is equal to the following condition. 
\[ \closure{\cRLotimesAB} = \closure{a \RLotimes b} \] 
\end{proposition}

\begin{proof} 
It is clear from  extensivity that $\cRLotimesAB \leq \closure{\cRLotimesAB}$. 
The other direction is;
\begin{prooftree}
\AxiomC{$ \cRLotimesAB \fCenter \closure{a \RLotimes b} $}
\UnaryInfC{$ \closure{\cRLotimesAB} \fCenter \closure{a \RLotimes b} $}
\DisplayProof \hskip 24pt
\AxiomC{$a \fCenter \closureA$}
\AxiomC{$b \fCenter \closureB$}
\BinaryInfC{$ a \RLotimes b \fCenter \cRLotimesAB$}
\UnaryInfC{$ \closure{a \RLotimes b} \fCenter \closure{\cRLotimesAB}$}
\end{prooftree}
\end{proof}

\begin{definition} [closed elements]
The element satisfying $\closureA  = a$ are said to be {\em closed}. 
We denote by $C$ the set of closed elements: 
$C = \set{a \in L | \closureA = a}$.
\end{definition}

\begin{lemma}
$C$ is closed uner $\LTCmeet$ and $\RLimp$. 
\end{lemma}

\begin{proof}
\hskip -7cm
\begin{prooftree}
\AxiomC{$ \cLTCmeetAB \fCenter \closureA$}
\UnaryInfC{$ \closure{\cLTCmeetAB} \fCenter \closureA$}
\AxiomC{$ \cLTCmeetAB \fCenter \closureB$}
\UnaryInfC{$ \closure{\cLTCmeetAB} \fCenter \closureB$}
\BinaryInfC{$ \closure{\cLTCmeetAB} \fCenter \cLTCmeetAB$}
%
\DisplayProof \hskip 48pt
%
\AxiomC{$\cRLimpAB \fCenter \cRLimpAB$}
\UnaryInfC{$\closureA \RLotimes (\cRLimpAB) \fCenter \closureB$}
\UnaryInfC{$\closure{\closureA \RLotimes (\cRLimpAB)} \fCenter \closureB$}
\UnaryInfC{$\closure{\closureA \RLotimes \closure{\cRLimpAB}} \fCenter \closureB$}
\UnaryInfC{$\closureA \RLotimes \closure{\cRLimpAB} \fCenter \closureB$}
\UnaryInfC{$\closure{\cRLimpAB} \fCenter \cRLimpAB$}
\end{prooftree}
\end{proof}

\begin{proposition}
Let $(L,\LTCjoin,\bot,\LTCmeet,\top,\RLimp,\RLcdot,\RLone)$ 
be a commutative residual lattice
with a closure operation $\closure{\QTargument}$. 
Put
\[  a \RLoplus b = \closure{a \LTCjoin b}   \hskip 3cm 
 a \RLotimes b = \closure{a \RLcdot b} \]
Then $(C,\RLoplus,\closure{\bot},\wedge,\top,\RLimp,\RLotimes,\closure{\RLone})$ 
is also a residuated lattice. 
\end{proposition}

\begin{proof}
\begin{enumerate}
\item {\em  $(C,\RLoplus,\closure{\bot},\LTCmeet,\LTCtop)$ is a lattice}:
Since $C$ is closed under $\LTCmeet$, $\LTCmeet$ is a meet. 
We show $\RLoplus$ is a join of $C$ under the same partial order as $L$. 
\begin{prooftree}
\AxiomC{$ \closureA \fCenter \closureA \LTCjoin \closureB$}
\UnaryInfC{$ \closureA \fCenter \closure{\closureA \LTCjoin \closureB}$}
\UnaryInfC{$ \closureA \fCenter  \closureA \RLoplus \closureB$}
\DisplayProof \hskip 12pt
\AxiomC{$ \closureB \fCenter \closureA \LTCjoin \closureB$}
\UnaryInfC{$ \closureB \fCenter \closure{\closureA \LTCjoin \closureB}$}
\UnaryInfC{$ \closureB \fCenter  \closureA \RLoplus \closureB$}
\DisplayProof \hskip 48pt
\AxiomC{$\closureA \fCenter \closureC$}
\AxiomC{$\closureB \fCenter \closureC$}
\BinaryInfC{$\closureA \LTCjoin \closureB \fCenter \closureC$}
\UnaryInfC{$\closure{\closureA \LTCjoin \closureB} \fCenter \closureC$}
\UnaryInfC{$\closureA \RLoplus \closureB \fCenter \closureC$}
\end{prooftree}

\item {\em $(C,\RLotimes,\closure{\RLone})$ is a commutative monoid}: 
\[ \closureA \RLotimes \closure{\RLone}
= \closure{\closureA \RLcdot \closure{\RLone}}
= \closure{a \RLcdot \RLone}
= \closureA
\]
\[ 
\closureA \RLotimes (\closureB \RLotimes \closureC)
=  \closure{\closureA \RLcdot \closure{\closureB \RLcdot \closureC}}
=  \closure{a \RLcdot (b \RLcdot c)}
=  \closure{(a \RLcdot b) \RLcdot c}
=  \closure{ \closure{\closureA \RLcdot \closureB} \RLcdot \closureC}
=  (\closureA \RLotimes \closureB) \RLotimes \closureC
\]
%
\item {\em residual law holds}:  

\begin{prooftree}
\AxiomC{$ \closureA \RLcdot \closureC \fCenter \closure{\closureA \RLcdot \closureC}$}
\AxiomC{$ \closureA \RLotimes \closureC \fCenter \closureB$}
\UnaryInfC{$\closure{\closureA \RLcdot \closureC} \fCenter \closureB$}
\BinaryInfC{$\closureA \RLcdot \closureC \fCenter \closureB$}
\UnaryInfC{$ \closureC \fCenter \closureA \RLimp \closureB$}
\DisplayProof \hskip 24pt
\AxiomC{$ \closureC \fCenter \closureA \RLimp \closureB$}
\UnaryInfC{$ \closureA \RLcdot \closureC \fCenter  \closureB$}
\UnaryInfC{$ \closure{\closureA \RLcdot \closureC} \fCenter  \closureB$}
\UnaryInfC{$ \closureA \RLotimes \closureC \fCenter  \closureB$}
\end{prooftree}
\end{enumerate}
\end{proof}

\begin{proposition}
$\dlneg{\QTargument}$ is a closure operator(Definition~\ref{RL:closureDef})
on a residual lattice. That is, the following conditions hold:
\begin{enumerate}
\item {\em extensivity} :  $a \leq \dRLnegA$,
\item {\em idempotence} : $ \dRLneg{(\dRLnegA)} \leq \dRLnegA$ ,
\item {\em monotonicity} : $a \leq b$ implies $\dRLnegA \leq \dRLnegB$,
\item {\em monoid}:
$ \dRLnegA  \RLotimes \dRLnegB \leq \dRLneg{(a \RLotimes b)}$.
\end{enumerate}
\end{proposition}

\begin{proof}
\hskip -7cm
\begin{prooftree}
\AxiomC{$\RLnegA \fCenter \RLnegA$}
\UnaryInfC{$a \RLotimes \RLnegA \fCenter \RLzero$}
\UnaryInfC{$\RLnegA \RLotimes a \fCenter \RLzero$}
\UnaryInfC{$a \fCenter \dRLnegA$}
\DisplayProof \hskip 48pt
\AxiomC{$\RLnegA \fCenter \dRLnegP{\RLnegA}$}
\UnaryInfC{$\dRLnegP{\dRLnegA} \fCenter \dRLnegA$}
\DisplayProof \hskip 48pt
\AxiomC{$a \fCenter b$}
\UnaryInfC{$\RLnegB \fCenter \RLnegA$}
\UnaryInfC{$\dRLnegA \fCenter \dRLnegB$}
% \DisplayProof \hskip 48pt
\end{prooftree}

\begin{prooftree}
\AxiomC{$ \RLotimesab \fCenter \RLotimesab$}
\UnaryInfC{$\RLnegP{\RLotimesab} \RLotimes \RLotimesab \fCenter \RLzero$}
\UnaryInfC{$\RLnegP{\RLotimesab} \RLotimes b \fCenter \RLnegA$}
\AxiomC{$\RLzero \fCenter \RLzero$}
\BinaryInfC{$\dRLnegA \RLotimes \RLnegP{\RLotimesab} \RLotimes b \fCenter \RLzero$}
\UnaryInfC{$\dRLnegA \RLotimes \RLnegP{\RLotimesab} \fCenter \RLnegB$}
\AxiomC{$\RLzero \fCenter \RLzero$}
\BinaryInfC{$\dRLnegB \RLotimes \dRLnegA \RLotimes \RLnegP{\RLotimesab} 
\fCenter \RLzero$}
\UnaryInfC{$\dRLnegA \RLotimes \dRLnegB \fCenter \dRLnegP{\RLotimesab} $}
\end{prooftree}
\end{proof}

\begin{proposition}
$\RLnegA$ is a closed element, i.e.  $\RLnegA = \dRLnegP{\RLnegA}$.
\end{proposition}

\begin{prooftree}
\AxiomC{$ \RLnegA \RLotimes a \fCenter \RLzero$}
\UnaryInfC{$ a \fCenter \RLnegP{\RLnegA}$}
\UnaryInfC{$\RLnegP{\RLnegP{\RLnegA}} \fCenter \RLnegA$}
\end{prooftree}










