\maketitle

\input{absIntro}

\input{TimesPlusAT}

\input{Quantale}

\def\fCenter{\subseteq}

\section{Phase Space}

\begin{definition}[power set] 
Given a set $M$,
the power set of  $M$ is the set of all subsets of $M$
including the $\emptyset$ (empty set) and $M$ itself.
\end{definition}
The power set of $M$ is denoted by $\PSpowerset{M}$.
%%
Recall that 
$(\PSpowerset{M},\cup,\emptyset,\cap,M,{\PSargument}^{\complement}) $ 
is always a complete boolean algebra, 
where $\cup$(set union), $\emptyset$(empty set),
$\cap$(set intersection) and ${\PSargument}^{\complement}$(set complement) 
are the usual set-theoretic operations. It is ordered by $\subseteq$(set inclusion).
%
\begin{definition}
Let $(M,\PScdot,\PSidentity)$ be a commutative monoid. A binary function
%
$ {\PSargument} { \PScdot} {\PSargument} : M \times M \longrightarrow M$ 
%
is naturally extended to binary operations on $\PSpowerset{M}$:
 \begin{eqnarray*}
 Z \otimes X  &=&  \bigcup \set{ z \PScdot x | z \in Z,  x \in X} \\
 X \multimap Y  &=&  \bigcup \set{ z  | \forall x \in X (z \PScdot x \in Y)} 
 \end{eqnarray*}
 for any subset $X$,$Y$ of $M$.
That is, 
$ Z \otimes X$ contains all the possible interactions 
between one element of $z \in Z$ and one element of $x \in X$. 
$ X \multimap Y$ contains all the possible elements $z \in Z$
such that $\forall x \in X (z \PScdot x \in Y)$.
\end{definition}
%%
\begin{proposition}
Let $(M,\cdot,\PSidentity) $ be a commutative monoid.
$(\PSpowerset{M},\cup,\emptyset,\cap,M,\multimap,\otimes,\set{\PSidentity}) $ 
is an IL-algebra.
\end{proposition}
\begin{proof}
\begin{enumerate}
%
\item {\em lattice}: $(\PSpowerset{M},\cup,\emptyset,\cap,M)$ 
is a lattice. 
%
\item {\em commutative monoid}:
The extension of $\PScdot$ to $\PSpowerset{M}$  induces  a commutative monoid $(\PSpowerset{M},\otimes,\set{\PSidentity}) $. 
%
\item {\em residual law}: 
for each pair of elements $(X,Y)$, 
there exists an element $X \QTmultimap Y$  
such that residual law holds: \hskip -7cm
\begin{prooftree}
	\AxiomC{$ Z \otimes X \fCenter Y$}
	\doubleLine
	\UnaryInfC{$ Z \fCenter X \multimap Y$}
\end{prooftree}
\end{enumerate}
\end{proof}

%%%

\begin{definition}[phase space]
A phase space is a quadruple $(\PSpowerset{M},\cdot,\set{\PSidentity},\PSzero)$ 
where
\begin{enumerate}
\item {\em commutative monoid}:$(\PSpowerset{M},{\otimes},\set{\PSidentity}) $ 
is a commutative monoid.
\item {\em multiplicative zero}: ${\PSzero}$ 
is a designated subset of $M$ ($ \PSzero \subseteq M$) 
called {\it multiplicative zero}. 
\end{enumerate}
\end{definition}
%%
The elements of $\PSpowerset{M}$ are called phases. 

%
\begin{definition}[relation]
A relation is any subset of a Cartesian product.  A subset of  $X \times Y$, called a binary relation from $X$ to $Y$,
is a collection of ordered pairs $(x,y)$ with first components from $X$ and second components from $Y$.
\end{definition}
%%
\begin{definition}[linear negation]
We write $\bot$ for the relation  $\set{ (x,y) | x \cdot y \in \PSzero}$.
We define the linear negation operator  $\lneg{\PSargument} : X \longrightarrow Y$ as  $y \in \lneg{X}  \mbox{ iff for every }  x \in X \mbox{  we have  }  (x,y) \in \bot $.
\end{definition}
%
%\begin{definition}[linear negation]
%We define the linear negation operator  
%$\lneg{\PSargument} : X \longrightarrow Y$ as follows:
%
% \[ y \in \lneg{X}  \mbox{ iff for every }  x \in X \mbox{  we have  }  (x,y) \in \bot \]
%
%\end{definition}
An easy consequence of the above definition is that 
$\lneg{\PSemptyset} = M$,
$\lneg{\set{\PSidentity}} = \PSzero$
  and $X \otimes \lneg{X} \subseteq \PSzero$. 
%
\begin{proposition} For any subset $X,Y,Z$ of $M$, we have the following property. 
\begin{enumerate}
\item $Z \subseteq X$  implies  $Z \otimes \lneg{X} \subseteq \PSzero$. 
\item $X \otimes Y \subseteq \PSzero$  iff $Y \subseteq \lneg{X}$.
\end{enumerate}
\end{proposition}
\begin{proof}
(1) For every $z \in Z \subseteq X$ and $y \in \lneg{X}$ we have $(z,y) \in \bot$. 
(2) From assumption, for every $x \in X$  and $y \in Y$  we have $(x,y) \in \bot$. Hence $Y \subseteq \lneg{X}$ by definition. Conversely, if $ y \in Y \subseteq \lneg{X}$, we have $(x,y) \in \bot$ for every $x \in X$ and $y \in Y$. 
\end{proof}
%
%
\begin{proposition} [closure operator]
For any subset $X,Y$ of $M$, $\dlneg{\PSargument}$ is the closure operator on $X,Y$. 
Explicitly, $\dlneg{\PSargument} : X \longrightarrow X$ is a closure operator if the following conditions hold:
%
\begin{enumerate}
\item $X \subseteq \dlneg{X}$.
\item If $X \subseteq Y$ then $\dlneg{X} \subseteq \dlneg{Y}$.
\item $\dlneg{(\dlneg{X})} \subseteq \dlneg{X}$.
\item $\dlneg{X} \otimes \dlneg{Y} \subseteq \dlneg{(X \otimes Y)}$.
\end{enumerate}
\end{proposition}
%
\begin{proof}
\begin{prooftree}
\AxiomC {$ X \fCenter X$}
\RightLabel{(1)} \UnaryInfC  {$X \otimes \lneg{X} \fCenter \PSzero$}
\RightLabel{(2)} \UnaryInfC  {$X \fCenter \dlneg{X}$}
\DisplayProof \hskip 24pt
%
\AxiomC {$ X \fCenter Y $}
\RightLabel{(1)} \UnaryInfC  {$X \otimes \lneg{Y} \fCenter \PSzero$}
\RightLabel{(2)} \UnaryInfC  {$\lneg{Y} \fCenter \lneg{X}$}
\RightLabel{(1)(2)} \UnaryInfC  {$\dlneg{X} \fCenter \dlneg{Y}$}
\DisplayProof \hskip 24pt
%
\AxiomC {$\lneg{X} \fCenter \dlneg{(\lneg{X})}$}
\RightLabel{(1)(2)} \UnaryInfC  {$\dlneg{(\dlneg{X})} \fCenter \dlneg{X}$}
\end{prooftree}
%
\begin{prooftree}
\AxiomC {$ X \otimes Y \fCenter X \otimes Y$}
\RightLabel{(1)}
 \UnaryInfC {$ X \otimes Y \otimes \lneg{( X \otimes Y)} \fCenter \PSzero$}
\RightLabel{(2)(1)(2)(1)}
\UnaryInfC {$ \dlneg{X} \otimes \dlneg{Y}  \otimes \lneg{( X \otimes Y)} 
                \fCenter \PSzero$}
\RightLabel{(2)}
\UnaryInfC  {$\dlneg{X} \otimes \dlneg{Y} \fCenter \dlneg{(X \otimes Y)}$}
\end{prooftree}
\end{proof}

\begin{definition} [fact] 
A fact in a phase space is a fixed point for the closure operator. Explicitly, for any subset $X$ of $M$, $X$ is a fact if $X = \dlneg{X}$. 
\end{definition}

\begin{proposition}  \label{propOfFact}
We have the following property about facts:
\begin{enumerate}
\item $\lneg{X}$ is a fact. 
\item $\dlneg{X}$ is the smallest fact that includes $X$. 
\end{enumerate}
\end{proposition}
%
\begin{proof}
(1) $\lneg{X}$ is a fact because of $\lneg{X} = \dlneg{(\lneg{X})}$. 
(2) Suppose $Y$ is any fact that includes $X$  (i.e., $Y = \dlneg{Y}$ and $X \subseteq Y$) .
Then $X \subseteq Y$ implies $\dlneg{X} \subseteq \dlneg{Y} = Y$.
It directly follows that $\dlneg{X}$ is the smallest among those facts. 

\end{proof}
%
\begin{proposition}
The set of facts of a phase space is a complete lattice. 
\end{proposition}
\begin{proof}
We define phase space operators, 
$\PSlub$ (least upper bound, lub) and $\PSglb$ (greatest lower bound, glb),  
for $(X_i)_{i \in I} \subseteq \PSpowerset{M}$, a family of subsets of $M$,  as follows:
%
\[
\PSlub = \dlnegP{\bigcup_{ i \in I}  X_i} 
\hskip 24pt \mbox{and} \hskip 24pt
\PSglb = \bigcap_{ i \in I}  X_i \mbox{.}
\]
%
$\PSlub$ is obviously a fact, it suffices to prove it is indeed a lub.
%%
$\PSlub$ is an upper bound 
because $X_i \subseteq \bigcup X_i$ implies $X_i = \dlneg{X_i} \subseteq \PSlub$.
%%
If $Y$ is an upper bound, i.e., $X_i \subseteq Y$ for all $i$, 
then $\cup X_i \subseteq Y$,  $\PSlub \subseteq \dlneg{Y} = Y $, 
so $\PSlub$ is indeed a lub within the set of facts. \\
%%
$\PSglb$ is obviously a glb, it suffices to prove it is indeed a fact. 
$\PSglb \subseteq X_i$ implies $\dlnegP{\PSglb} \subseteq \dlneg{X_i} = X_i$ for each $i$, 
hence $\dlnegP{\PSglb} \subseteq \PSglb$. \\
\end{proof}

%%%
\subsection{phase space semantics}

\begin{proposition} [phase space units]
According to the corresponding logical units of linear logic, we designate four facts as follows:
\begin{itemize}
\item $\PSzero$ (multiplicative zero) and  $\PSunit = \lnegP{\PSzero} $ (multiplicative unit).
\item $ \PStrue = M$ (additive true) and $\PSfalse = \lneg{M}$ (additive false). 
\end{itemize}
\end{proposition}
\begin{proof}
$\PStrue = M = \lneg{\PSemptyset}$ and $\PSzero = \lneg{\set{\PSidentity}}$ are facts from proposition~\ref{propOfFact}.
\end{proof}
%
Notice also that $\PSfalse = \dlneg{\PSemptyset}$ is the smallest fact 
out of all facts in a phase space.
$\PSunit = \dlneg{\set{\PSidentity}}$ is the smallest fact containing $\PSidentity$. 

\begin{definition} [phase space operators]
We define the phase space operators as follows:
\begin{eqnarray*}
X \with Y =  X \cap Y,
 &\phantom{XXXXX}&  X \oplus  Y  = \dlnegP{X \cup Y}, \\ 
X \parr  Y  = \lnegP{\lneg{X} \otimes \lneg{Y}} ,
 &\phantom{XXXXX}& X \otimes Y = \dlnegP{X \otimes Y},\\
\wn X = \lnegP{\lneg{X} \cap \PSexpStandard},
 &\phantom{XXXXX}& \oc X = \dlnegP{X \cap \PSexpStandard},
\end{eqnarray*}
where $\PSexpStandard = \set{ x \in \PSunit | x = x^2 }$. 
\end{definition}
%%%
\begin{proposition}
$\otimes$ distributes over $\bigoplus$. 
\end{proposition}
\begin{proof}
$ Y \cdot X_i \subseteq Y \cdot (\PSlub)$ implies 
$Y \otimes X_i \subseteq \PSdist$ for each $i$, 
and therefore $\bigcup (Y \otimes X_i) \subseteq \PSdist$. 
This implies $ \PSsrc  \subseteq \PSdist $.
Conversely,  $\PSdist = \dlnegP{\dlneg{Y} \cdot \dlnegP{\bigcup X_i}}
= \dlnegP{Y \cdot \bigcup X_i} = \dlnegP{ \bigcup (Y \cdot X_i)} =  \PSpre$. 
But $\PSpre \subseteq \PSsrc$, this completes the proof. 
\end{proof}
%%
\begin{definition} [phase space model]
Given a formula $A$ of linear logic and an assignation that associate a fact $\alpha^{M}$ to any variable $\alpha$. 
A phase space model is a phase space together with a fact  for each (propositional) variables. 
We define a interpretation $\interpret{\PSargument}$ of a formula  $A$ in a phase space$(M,\cdot,\PSidentity,\PSzero)$  by structural induction as follows:
\begin{itemize}
\item $\interpretP{X \otimes Y} = \interpret{X} \otimes \interpret{Y}$.
\item $\interpretP{X \oplus  Y}   = \interpret{X} \oplus \interpret{Y}$.
\end{itemize}
$\interpret{\PSargument}$ is lifted to contexts (multiset of formulas)  as  
%
$\interpretP{A_1,\ldots,A_n} 
   = \interpretP{A_1} \otimes \ldots \otimes \interpretP{A_n}$.
%
\end{definition}


\begin{theorem}[Soundness]
Let $\Gamma \LLvdash B$ be a provable sequent in linear logic.
Then $ \interpret{\Gamma} \subseteq \interpret{B}$ for any assignment for variables. 
\end{theorem}
\begin{proof}
By induction  on proofs.
\end{proof}




\input{PSexample3}
\input{PSexample}
\begin{table}[htb]
\begin{center}
  \begin{tabular}{c|ccc|c|cccc} 
   intrusion senario & $A$ & $B$ & $C$ & $\LLbplusc$ & $\LLsrcForm$ &
   $\LLatimesb$ & $\LLatimesc$ & $\LLdistForm$  \\ \hline
    cost & 30 & 20 & 10  & 10  & 40 & 50 & 40 & 40 \\
  \end{tabular}
\end{center}
\caption{quantified valuation for chepest  --- an example}
\end{table}


\newcommand{\QVaTimesC}{A \otimes C}
\newcommand{\QVbTimesC}{B \otimes C}

\begin{table}[htb]
\begin{center}
  \begin{tabular}{c|ccc|c|cccc} 
   intrusion senario & $A$ & $B$ & $C$ & $\LLbplusc$ & $\LLsrcForm$ &
   $\LLatimesb$ & $\LLatimesc$ & $\LLdistForm$  \\ \hline
   probability & 0.9 & 0.8 & 0.7 & 0.94 & 0.846 & 0.72 & 0.63 & 0.8964  \\
   expense & 30 & 20 & 10  & 30  & 60 & 50 & 40 & 90 \\
  \end{tabular}
\end{center}
\caption{quantified valuation for costs and probability --- an example}
\end{table}

\subsection{distributive attribute domain}

A

\subsection{lowest-cost-based analysis}

\begin{eqnarray*}
\valuationC{A \otimes B} &=& \valuationC{A} + \valuationC{B} \\
\valuationC{A \oplus B} &=& \min \set{\valuationC{A},\valuationC{B}} 
\end{eqnarray*}

The valuation for $\oplus$ ($min{}$) obviously breaks the soundness,
because  $\oplus$ is required to be an least upper bound from the inference rule. 
Specifically, it must be an upperbound for $\interpret{A}$,
 i.e., $\interpret{A} \leq \interpret{A} \oplus \interpret{B}$.


\subsection{attack-cost-gain-based analysis}

The actual decision-making process of attackers is more complicated and that the interactions between different parameters
play an important role. 

For example, if the success probability and possible monetary gain are considered as parameters, their product (i.e. the expected gain) also has a meaning and can be taken into account when making decisions about attacks.

The model of probability and expense, for instance, can be modeled by the following valuations: 

\begin{eqnarray*}
\valuationP{A \otimes B} &=& \valuationP{A} \times \valuationP{B} \\
\valuationP{A \oplus B} 
  &=& \valuationP{A} + \valuationP{B} - (\valuationP{A} \times \valuationP{B}) 
\end{eqnarray*}

\begin{eqnarray*}
\valuationE{A \otimes B} &=& \valuationE{A} + \valuationE{B} \\
\valuationE{A \oplus B} &=& \valuationE{A} + \valuationE{B} 
\end{eqnarray*}

When the gain $\valuationG{A}$  for an attack scenario $A$  is specified,  the expected gain can be calculated by
$\valuationG{A} \times \valuationP{A}$. 

The rational attacker do not attack if the attack scenario is unprofitable and chose the most profitable attack scenarios. 

The attacke scenario ( attack suite / intrusion scenario )  shows ways of attacking. 




\subsection{impact(risk * damage)-based analysis}

\subsection{capabilities-based analysis}

\url{http://www.amenaza.com/AT-whatAre.php}

Performing the leaf level attack operations usually requires the adversary to expend resources (time, money, skill, etc.) 
Different attack scenarios will have different sets of resources costs. Since adversaries differ in their characteristics, some scenarios will be better suited to one adversary and other scenarios to another. The analysis of which scenarios best fit a given adversary is known as capabilities-based analysis.



\section{further work}

It should be better if we allow sharing of leaf nodes ( variables ).  For this, we have to introduce exponentials in linear logic. 

\url{http://www.amenaza.com/downloads/docs/AttackTreeThreatRiskAnalysis.pdf}

\appendix 

\section{Lattices}
	\subsection{Lattices as algebraic structures}
		\input{LatticeA}
	\subsection{Lattices as partially ordered sets}
		\input{LatticeB}

\section{Proof Calculi}
	\subsection{propositional language}
		\input{PropositionalLogic}
	\subsection{Hilbert-style and Gentzen-style proof calculus} 
		\input{ProofCalculi}
	\subsection{Hilbert Proof Systems}
		\input{HilbertProofSystems}
	\subsection{Redundancies in the Axioms}
		\input{Redundancy}
	\subsection{Heyting Algebra Axioms}
		\input{HAaxioms}
	\subsection{Deducibility from assumption in the Hilbert systems}
		%
% Rules
%
\begin{table}
\begin{prooftree}
\RightLabel{Ax}  \AxiomC{}  \UnaryInfC {$A \fCenter A$}
\end{prooftree}
\begin{prooftree}
\AxiomC	{$\Gamma \fCenter A$} 
\AxiomC{$\Delta\fCenter A \RLimp B$}  
\RightLabel{M.P.}
\BinaryInfC{$\Gamma,\Delta \fCenter B$}
	\DisplayProof  \hskip 32pt
\AxiomC{$\Gamma \fCenter A$} 
\AxiomC{$\Gamma \fCenter B$}  
\RightLabel{Conjunction}
\BinaryInfC{$\Gamma \fCenter A \with B$}
\end{prooftree}
\caption{Inference Rules: Deduction from Assumption}
\end{table}

\newcommand{\DTwea} { \alwaysNoLine \AxiomC{$\Pi_1$}  
	\UnaryInfC{$\Gamma \fCenter B$}}

\newcommand{\DTcon} { \alwaysNoLine \AxiomC{$\Pi_1$}  
	\UnaryInfC{$\Gamma,A,A  \fCenter B$}}


\newcommand{\DTmpaONE}{ 	\alwaysNoLine 		
	\AxiomC{$\Pi_1$} 		\UnaryInfC{$ \Gamma,A \fCenter B$} 	}
\newcommand{\DTmpaTWO}{  \alwaysNoLine 
	\AxiomC{$\Pi_2$}{ \UnaryInfC{$ \Delta \fCenter \LTCimpBC$}}}

\newcommand{\DTmpbONE}{ 	\alwaysNoLine 		
	\AxiomC{$\Pi_1$} 		\UnaryInfC{$ \Gamma \fCenter B$} 	}
\newcommand{\DTmpbTWO}{  \alwaysNoLine 
	\AxiomC{$\Pi_2$}{ \UnaryInfC{$ \Delta,A  \fCenter \LTCimpBC$}}}

\newcommand{\DTadjONE} { \alwaysNoLine \AxiomC{$\Pi_1$}  
	\UnaryInfC{$\Gamma, C \fCenter A$}}
\newcommand{\DTadjTWO} { \alwaysNoLine \AxiomC{$\Pi_2$}
	\UnaryInfC{$\Gamma, C \fCenter B$}}

\newcommand{\HBTleadsto}
	{\hskip 6pt \raisebox{-16pt}{\LARGE $\leadsto$}  \hskip 6pt }
\newcommand{\HBTleadstoAx}
	{\hskip 24pt \raisebox{0pt}{\LARGE $\leadsto$}  \hskip 24pt }
\newcommand{\HBTleadstoWea}
	{\hskip 24pt \raisebox{-6pt}{\LARGE $\leadsto$}  \hskip 24pt }
\newcommand{\HBTleadstoCon}
	{\hskip 24pt \raisebox{-12pt}{\LARGE $\leadsto$}  \hskip 24pt }

%%%

We now advance to a theory of Deduction from assumption. 

\subsubsection{Internal and External deductions}

In sequent calculi, we can distinguish several notions of deduction from hypotheses. 

\begin{enumerate}
\item $B$ is an internal consequence of $\Gamma$  iff 
the sequent $ \Gamma \LLvdash B$ is a conditional tautology. 
%
\item $B$ is an external consequence of $\Gamma$  iff  
the sequent $\LLvdash B$ is derivable from hypotheses: 
$\LLvdash A_1, \ldots, \LLvdash A_n$.  We write 
\begin{prooftree}
\AxiomC{$\LLvdash A_1$}
\AxiomC{$\ldots$}
\AxiomC{$\LLvdash A_n$}
\TrinaryInfC{$\LLvdash B$}
\end{prooftree}
\end{enumerate}
%
Our important motivation stems from the search for the notation of deduction where $B$ is said to follow an assumption $\Gamma$ in an explicit manner.  Suppose $\Gamma$ is a context( i.e., multiset of formulas). 
%
In this case, we say $B$ is an internal consequence of $\Gamma$ and write $\Gamma \fCenter B$. 
%
Thus, informally, ``a sequent $\Gamma \fCenter B$ is derivable'' means that $B$ is provable assuming all the formulas in $\Gamma$. 

Another notion of deduction from hypotheses is an external consequence;  a sequent $S$ is externally deducible (i.e., derivable)  from a multiset $X$ of sequents if $S$ can be derived with the element of $X$.



\begin{lemma}[Deduction Theorem]
The deduction theorem holds, i.e.,
\[
 \Gamma, A \fCenter B  \hskip 48pt \mbox{implies} \hskip 48pt
  \Gamma \fCenter \LTCimpAB
\]
\end{lemma}
%
\begin{proof}
By induction on the length of derivations. 
\flushleft{\textbf{\large base case}}
\begin{prooftree}
\RightLabel{Ax}  \AxiomC{}  \UnaryInfC {$A \fCenter A$}
	\DisplayProof \HBTleadstoAx
\RightLabel{Axiom I}  \AxiomC{}  \UnaryInfC {$\fCenter \LTCimpNP{A}{A}$}
\end{prooftree}
%
\flushleft{\textbf{\large induction step}}
% Weakening
\begin{prooftree}
	\DTwea
		\alwaysSingleLine
			\RightLabel{Weakening}
			\UnaryInfC{$\Gamma,A \fCenter B$}
\DisplayProof 	\HBTleadstoWea
	\DTwea
		\alwaysSingleLine
			\RightLabel{K}
			\UnaryInfC{$\Gamma \fCenter \LTCimpAB$}
\end{prooftree}
% Contraction
\begin{prooftree}
	\DTcon
		\alwaysSingleLine
			\RightLabel{Contraction}
			\UnaryInfC{$\Gamma,A \fCenter B$}
\DisplayProof 	\HBTleadstoCon
	\DTcon
		\alwaysSingleLine
			\RightLabel{I.H.}
			\UnaryInfC{$\Gamma \fCenter \LTCimpNP{A}{(\LTCimpAB)}$}
			\RightLabel{W}
			\UnaryInfC{$\Gamma \fCenter \LTCimpAB$}
\end{prooftree}
% Modus Ponens A
\begin{prooftree}
	\DTmpaONE
	\DTmpaTWO
		\alwaysSingleLine
		\RightLabel{M.P.} 
		\BinaryInfC{$\Gamma, \Delta, A \fCenter C$}
\DisplayProof 	\HBTleadsto
	\DTmpaONE
		\alwaysSingleLine
		\RightLabel{I.H.}
		\UnaryInfC{$ \Gamma \fCenter \LTCimpAB$}
%
	\DTmpaTWO
		\alwaysSingleLine
			\RightLabel{B} 
			\BinaryInfC{$\Gamma, \Delta  \fCenter \LTCimpAC$}
\end{prooftree}
% Modus Ponens B
\begin{prooftree}
	\DTmpbONE
	\DTmpbTWO
		\alwaysSingleLine
		\RightLabel{M.P.} 
		\BinaryInfC{$\Gamma, \Delta, A \fCenter C$}
\DisplayProof  \HBTleadsto
	\DTmpbONE
%
	\DTmpbTWO
		\alwaysSingleLine
		\RightLabel{I.H.} 	\UnaryInfC{$ \Gamma \fCenter \LTCimpABC$}
		\RightLabel{C} 		\UnaryInfC{$\Gamma, \Delta  \fCenter \LTCimpBAC$}
	\RightLabel{M.P.} 
	\BinaryInfC{$\Gamma, \Delta  \fCenter \LTCimpAC$}
\end{prooftree}
% Adjunction
\begin{prooftree}
	\DTadjONE    	\DTadjTWO
		\alwaysSingleLine
		\RightLabel{Adj.} 
		\BinaryInfC{$\Gamma,C  \fCenter \LTCmeetAB$}
\DisplayProof
\hskip 12pt \mbox{\LARGE $\leadsto$}  \hskip 12pt 
	\DTadjONE \alwaysSingleLine \RightLabel{I.H.} 
		\UnaryInfC {$\Gamma \fCenter \LTCimpCA$}
	\DTadjTWO \alwaysSingleLine \RightLabel{I.H.}
		\UnaryInfC {$\Gamma \fCenter \LTCimpCB$}
	\RightLabel{Adj.} 		
	\BinaryInfC{$\Gamma, \Delta  \fCenter \LTCmeetNP{(\LTCimpCA)}{(\LTCimpCB)}$}
			\RightLabel{$\LTCmeet$ intro.} 
			\UnaryInfC{$\Gamma, \Delta  \fCenter \LTCimpNP{C}{(\LTCmeetAB)}$}
\end{prooftree}
\end{proof}

\subsubsection{Cut}


	\subsection{Classical Logic}
		\input{LTCcomplement}	
	\subsection{Lindenbaum-Tarski-Algebra}
		The Lindenbaum–Tarski algebra of a theory in propositional logic is an algebraic (and order-theoretic) structure built out of its formula?s (modulo provable equivalence) and connectives. It thus carries an algebraic structure that corresponds to the logic in question and is generally the free such structure generated by the axioms of the theory.

Examples

The Lindenbaum–Tarski algebra of a theory in intuitionistic propositional logic is a Heyting algebra.

The Lindenbaum–Tarski algebra of a theory in classical propositional logic is a Boolean algebra.

%  Linear logics are, in fact, relevant logics 
% that lack contraction and the distribution of conjunction over disjunction


%  def  "provably equivalent" 

If two formulas $A$ and $B$ have the property that each implies the other, i.e., $A \vdash B$ and $B \vdash A$, 
then then $A$ and $B$ are said to be {\it congruent} 
and denoted by $A \cong B$. 

%% lemma   provably equivalent is  A \cong B  is a congruent relation 

\begin{enumerate}
\item {\it determination} \cite{Lincoln90simple,Lincoln94ll,Lincoln94tcs}
\item {\it reflexive}
\item {\it symmetric}
\item {\it transitive}
\item {\it structure}
\end{enumerate}


\newcommand{\LTCapprox}{\approx}

To discuss about  ``equality''

Lindenbaum-Tarski algebra 

\subsection{Lindenbaum–Tarski algebra}

The most basic equalities on formulas are

equivalence class corresponding a formula $P$ in the 

% What is Lindenbaum-Tarski algebra 

The Lindenbaum–Tarski algebra of a proof system consists of the equivalence classes of formulas. 

% Equivalence class

An equivalence class is defined as a subset of the form $\set{x \in P | x \LTCapprox A}$  where $A$ is an element (i.e., formula) of $P$ and the notation $x \LTCapprox y$  is used to mean that there is an equivalence relation between $x$ and $y$. 

% congruence 

A congruence is an equivalence relation on an algebraic object that is compatible with the algebraic structure, in the sense that the operations are well-defined on the equivalence classes. 
Every congruence relation has a corresponding quotient structure, whose elements are the equivalence classes.

The Lindenbaum–Tarski algebra is thus the quotient algebra obtained by factoring the algebra of formulas by this congruence relation.

We write $ A / \LTCapprox$  for the equivalence class corresponding to a prime formula  $A$ in the Lindenbaum-Tarski algebra. 
For example, $\LTCtop / \LTCapprox$  is an equivalence class of tautology. 

% equivalence relation 

We define an equivalence relation $\LTCapprox$  on the set of formulas  as follows:
\[  A \LTCapprox  B  \phantom{XXXXX}\mbox{if and only if}   \phantom{XXXXX}
\LTCvdash A \LTCimp  B \phantom{XXXXX}\mbox{and} \phantom{XXXXX}
  \LTCvdash B \LTCimp  A \]

% well-definedness

The operations in a Lindenbaum–Tarski algebra  are inherited from those in the underlying proof system.  They  must be well-defined on the equivalence classes  --- 
i.e., $A \LTCapprox B$ and $C \LTCapprox D$ implies:
\begin{enumerate}
\item $\LTCneg A \LTCapprox \LTCneg B$
\item  $A \LTCmeet B \LTCapprox C \LTCmeet D$ 
\item  $A \LTCjoin B \LTCapprox C \LTCjoin D$ 
\item  $A \LTCimp B \LTCapprox C \LTCimp D$.
\end{enumerate}

%  implication is a partial order on P

Then  $\LTCimp$ is a partial order $\leq$ on $P$ as follows:
\begin{enumerate}
\item The axiom I implies reflexivity  --- $A \leq A$.
\item  The axiom B implies transitivity --- if  $A \leq B$  and $B \leq C$, then $A \leq C$. 
\item  The congruence relation $\LTCapprox$ implies antisymmetry --- if $A \leq B$ and $B \leq A$, then $A \LTCapprox B$. 
\end{enumerate}


\subsection{Equality}


\newcommand{\LLdistLeaf}[1]{
\LLaxiom{A} \LLaxiom{#1}
\RightLabel{$\otimes$R}
\BinaryInfC{ $A,#1\fCenter A \otimes #1 $ }
\RightLabel{$\oplus$R}
\UnaryInfC{$A,#1\fCenter\LLdistForm$} 
}

\newcommand{\LLsrcLeaf}[1]{
\LLaxiom{A}  \LLaxiom{#1} 
\RightLabel{$\oplus$R}  \UnaryInfC{$#1 \fCenter B \oplus C$} 
\RightLabel{$\otimes$R}  \BinaryInfC{$A , #1 \fCenter \LLsrcForm$}
\RightLabel{$\otimes$L}  \UnaryInfC{$A \otimes #1 \fCenter \LLsrcForm$}
}

\begin{table}[t]  
\begin{prooftree}
\LLdistLeaf{B}  \LLdistLeaf{C}
\RightLabel{$\oplus$L} \BinaryInfC{ $A, B\oplus C\fCenter  (\LLdistForm)$}
\RightLabel{$\otimes$L} \UnaryInfC{ $\LLsrcForm \fCenter \LLdistForm$}
\end{prooftree}
%
\begin{prooftree}
\LLsrcLeaf{B}  \LLsrcLeaf{C}
\RightLabel{$\oplus$L} \BinaryInfC{ $\LLdistForm \fCenter \LLsrcForm$}
\end{prooftree}
\caption{Distributibity in Linear Logic}
\label{table:DL}
\end{table}

%%%
\subsection{Distributivity in Linear Logic}
%%%

\begin{proposition} 
Distributibity of $\otimes$ over $\otimes$ holds:

\[  \LLsrcForm =  \LLdistForm  \]
\end{proposition}
\begin{proof}
The proof is shown in Table~\ref{table:DL}
\end{proof}

% Distributibity of $\parr$ over $\with$.    \[  A \parr (B \with C) = (A\parr B)\with(A\parr C)  \]


We define XX  as a binary relation between formulas (in X) by

xxxxx


IF  A modulo B,  we say that A and B are equivalent modulo xxx.


XXX  is a congruence relation of free algebras. 

We have to demonstrate it is comptible under operators. 

L-T algebra for this logic. 

We simply define it to be the quotient algebra of the free algebra over the congruence relation . 










\section{What is classical logic / linear logic ? }
		\newcommand{\CLexcludedMiddle}{A \vee \neg A}
\newcommand{\CLaxiom}[1]{\AxiomC {$#1 \fCenter #1$}}
\newcommand{\CLconj}{\wedge}
\newcommand{\CLdisj}{\vee}
\newcommand{\CLimp}{\to}

\def\fCenter{\LLvdash}

\newcommand{\CLnegRule}{
\begin{center}
\begin{prooftree}
\RightLabel{L$\neg$}
\AxiomC {$\Gamma \fCenter A,\Delta$}
\UnaryInfC  {$\Gamma,\neg{A} \fCenter \Delta$}
	\DisplayProof \hskip 96pt
\RightLabel{R$\neg$}
\AxiomC {$\Gamma,A \fCenter \Delta$}
\UnaryInfC  {$\Gamma \fCenter \neg{A},\Delta$}
\end{prooftree}
\end{center}
}

\begin{table}
\begin{center}
\begin{prooftree}
\RightLabel{Ax}\AxiomC{} \UnaryInfC {$A \fCenter A$}
	\DisplayProof \hskip 48pt
\RightLabel{Cut}
\AxiomC {$\Gamma_0 \fCenter A,\Delta_0$}
\AxiomC {$\Gamma_1,A \fCenter \Delta_1$}
\BinaryInfC  {$\Gamma_0,\Gamma_1 \fCenter \Delta_0,\Delta_1$}
\end{prooftree}
\end{center}
%%

\begin{center}
\begin{prooftree}
\AxiomC {$\Gamma,A,A\fCenter\Delta$} \RightLabel{LC} \UnaryInfC {$\Gamma,A\fCenter\Delta$}
	\DisplayProof \hskip 18pt
\AxiomC {$\Gamma\fCenter A,A,\Delta$}
\RightLabel{RC} \UnaryInfC {$\Gamma\fCenter A,\Delta$}
	\DisplayProof \hskip 18pt\AxiomC {$\Gamma\fCenter\Delta$} \RightLabel{LW} \UnaryInfC {$\Gamma,A\fCenter\Delta$}
	\DisplayProof \hskip 18pt
\AxiomC {$\Gamma\fCenter\Delta$} 
\RightLabel{RW} \UnaryInfC {$\Gamma\fCenter A,\Delta$}
\end{prooftree}
\end{center}
%%
\begin{center}
\begin{prooftree}
\AxiomC {$\Gamma\fCenter\Delta$} \RightLabel{L$\top$} 
\UnaryInfC {$\Gamma,\top\fCenter\Delta$}
	\DisplayProof \hskip 18pt
\RightLabel{R$\top$} \AxiomC{} \UnaryInfC {$\Gamma\fCenter \top,\Delta$}
	\DisplayProof \hskip 18pt
\RightLabel{L$\bot$} \AxiomC{}\UnaryInfC {$\Gamma,\bot\fCenter\Delta$} 
	\DisplayProof \hskip 18pt
\AxiomC {$\Gamma\fCenter\Delta$} \RightLabel{R$\bot$} 
\UnaryInfC {$\Gamma\fCenter \bot,\Delta$}
\end{prooftree}
\end{center}
%%
\CLnegRule
%%
\begin{center}
\begin{prooftree}
\RightLabel{L$\CLimp$}
\AxiomC {$\Gamma_0 \fCenter A,\Delta_0$}
\AxiomC {$\Gamma_1,B \fCenter \Delta_1$}
\BinaryInfC  {$\Gamma_0,\Gamma_1,A\CLimp B \fCenter \Delta_0,\Delta_1$}
%
\DisplayProof \hskip 48pt
%
\RightLabel{R$\CLimp$}
\AxiomC {$\Gamma,A \fCenter B,\Delta$}
\UnaryInfC  {$\Gamma \fCenter A\CLimp B,\Delta$}
\end{prooftree}
\end{center}
%%
%%
\begin{center}
\begin{prooftree}
\RightLabel{L$\CLconj$ $(i = 1,2)$}
\AxiomC {$\Gamma,A_i \fCenter \Delta$}
\UnaryInfC  {$\Gamma,A_1 \CLconj A_2 \fCenter \Delta$}
%
\DisplayProof \hskip 64pt
%
\RightLabel{R$\CLconj$}
\AxiomC {$\Gamma \fCenter A,\Delta$}
\AxiomC {$\Gamma \fCenter B,\Delta$}
\BinaryInfC  {$\Gamma \fCenter A\CLconj B,\Delta$}
\end{prooftree}
\end{center}
%%
\begin{center}
\begin{prooftree}
\RightLabel{L$\CLdisj$}
\AxiomC {$\Gamma,A \fCenter \Delta$}
\AxiomC {$\Gamma,B \fCenter \Delta$}
\BinaryInfC  {$\Gamma,A\CLdisj B, \fCenter \Delta$}
%
\DisplayProof
\hskip 64pt
%
\RightLabel{R$\CLdisj$ $(i = 1,2)$}
\AxiomC {$\Gamma \fCenter A_i,\Delta$}
\UnaryInfC  {$\Gamma \fCenter A_1 \CLdisj A_2,\Delta$}
%
\end{prooftree}
\end{center}
%%
\caption{Classical Logic}
\end{table}

\begin{table}
\begin{center}
\begin{prooftree}
\CLaxiom{A}  
\RightLabel{Weakening} \UnaryInfC  {$A \fCenter B,A$}
\UnaryInfC  {$ \fCenter A \to B,A$}
\CLaxiom{A}
\BinaryInfC{$ (A \to B) \to A \fCenter A,A$}
\RightLabel{Contraction}  \UnaryInfC{$ (A \to B) \to A \fCenter A$}
\UnaryInfC{$  \fCenter ((A \to B) \to A) \to A$}
%
\DisplayProof
\hskip 48pt
%
\CLaxiom{A}
\UnaryInfC  {$A \fCenter \CLexcludedMiddle$}
\UnaryInfC  { $\fCenter \neg A,\CLexcludedMiddle$}
\UnaryInfC  {$ \fCenter \CLexcludedMiddle, \CLexcludedMiddle$}
\UnaryInfC  {$ \fCenter \CLexcludedMiddle$}
\end{prooftree}
\end{center}
\caption{pierce's law  and excluded middle}
\end{table}

\begin{table}
\begin{center}\begin{prooftree}
\AxiomC{[$A$] \scriptsize{1}} \AxiomC{[$\neg A$] \scriptsize{2}}
\BinaryInfC {$\bot$}
\UnaryInfC{$B$}
\RightLabel{$\to$ (1)} \UnaryInfC {$A \to B$}
\AxiomC {[$(A \to B) \to A$] \scriptsize{3}}
\BinaryInfC {$A$}
\AxiomC{[$\neg A$] \scriptsize{2}}
\BinaryInfC{$\bot$}
\RightLabel{$\to$ (2)} \UnaryInfC{$\neg\neg A$}
\UnaryInfC{$A$}
\RightLabel{$\to$ (3)} \UnaryInfC {$((A \to B) \to A) \to A$}
\end{prooftree}\end{center}
\caption{pierce's law in natural deduction style}
\end{table}


		\newcommand{\LLaddUnitRule}{
\begin{center}
\begin{prooftree}
\RightLabel{L$\LLzero$}
\AxiomC{}
\UnaryInfC{${\Gamma,\LLzero} \fCenter {\Delta}$}
	\DisplayProof \hskip 32pt
\RightLabel{R$\LLtop$}
\AxiomC{}
\UnaryInfC{${\Gamma} \fCenter {\LLtop,\Delta}$}
\end{prooftree}
\end{center}
}

\newcommand{\LLwithRule}{
\begin{center}
\begin{prooftree}
\RightLabel{L$\with$$(i = 1,2)$}
\AxiomC {$\Gamma,A_i \fCenter \Delta$}
\UnaryInfC  {$\Gamma,A_1 \with A_2 \fCenter \Delta$}
	\DisplayProof \hskip 32pt
\RightLabel{R$\with$}
\AxiomC {$\Gamma \fCenter A,\Delta$}
\AxiomC {$\Gamma \fCenter B,\Delta$}
\BinaryInfC  {$\Gamma \fCenter A \with B, \Delta$}
\end{prooftree}
\end{center}
}

\newcommand{\LLoplusRule}{
\begin{center}
\begin{prooftree}
\RightLabel{L$\oplus$}
\AxiomC {$\Gamma,A \fCenter \Delta$}
\AxiomC {$\Gamma,B \fCenter \Delta$}
\BinaryInfC  {$\Gamma,A\oplus B \fCenter \Delta$}
	\DisplayProof \hskip 32pt
\RightLabel{R$\oplus$$(i = 1,2)$}
\AxiomC {$\Gamma \fCenter A_i,\Delta$}
\UnaryInfC  {$\Gamma \fCenter A_1 \oplus A_2,\Delta$}
%
\end{prooftree}
\end{center}
}

\newcommand{\LLmultUnitRule}{
\begin{center}
\begin{prooftree}
\RightLabel{L$\LLone$}
\AxiomC{$\Gamma\Rightarrow\Delta$}
\UnaryInfC{$\Gamma,\LLone\Rightarrow\Delta$}
	\DisplayProof \hskip 16pt
\RightLabel{R$\LLone$}
\AxiomC{}
\UnaryInfC{$ \fCenter \LLone $}
	\DisplayProof \hskip 16pt
\RightLabel{L$\LLbot$}
\AxiomC{}
\UnaryInfC{$\LLbot \fCenter$}
	\DisplayProof \hskip 16pt
\RightLabel{R$\LLbot$}
\AxiomC{$\Gamma\Rightarrow\Delta$}
\UnaryInfC{$\Gamma\Rightarrow\LLbot,\Delta$}
\end{prooftree}
\end{center}
}

\newcommand{\LLotimesRule}{
\begin{center}
\begin{prooftree}
\RightLabel{L$\LLotimes$}
\AxiomC {$\Gamma,A,B \fCenter \Delta$}
\UnaryInfC  {$\Gamma,A \LLotimes B \fCenter \Delta$}
	\DisplayProof \hskip 64pt
\RightLabel{R$\LLotimes$}
\AxiomC {$\Gamma_0 \fCenter A,\Delta_0$}
\AxiomC {$\Gamma_1 \fCenter B,\Delta_1$}
\BinaryInfC  {$\Gamma_0,\Gamma_1 \fCenter A\LLotimes B,\Delta_0,\Delta_1$}
\end{prooftree}
\end{center}
}

\newcommand{\LLimpRule} {
\begin{center}
\begin{prooftree}
\RightLabel{L$\LLimp$}
\AxiomC {$\Gamma_0 \fCenter A,\Delta_0$}
\AxiomC {$\Gamma_1,B \fCenter \Delta_1$}
\BinaryInfC  {$\Gamma_0,\Gamma_1,A\LLimp B \fCenter \Delta_0,\Delta_1$}
	\DisplayProof \hskip 32pt
\RightLabel{R$\LLimp$}
\AxiomC {$\Gamma,A \fCenter B,\Delta$}
\UnaryInfC  {$\Gamma \fCenter A\LLimp B,\Delta$}
\end{prooftree}
\end{center}
}

\newcommand{\LLnegRule}{
\begin{center}
\begin{prooftree}
\RightLabel{L$\lneg{(\phantom{l})}$}
\AxiomC {$\Gamma \fCenter A,\Delta$}
\UnaryInfC  {$\Gamma,\lneg{A} \fCenter \Delta$}
	\DisplayProof \hskip 96pt
\RightLabel{R$\lneg{(\phantom{l})}$}
\AxiomC {$\Gamma,A \fCenter \Delta$}
\UnaryInfC  {$\Gamma \fCenter \lneg{A},\Delta$}
\end{prooftree}
\end{center}
}

\begin{table}
\begin{center}
\begin{prooftree}
\RightLabel{Ax}\AxiomC{} \UnaryInfC {$A \fCenter A$}
	\DisplayProof \hskip 48pt
\RightLabel{Cut}
\AxiomC {$\Gamma_0 \fCenter A,\Delta_0$}
\AxiomC {$\Gamma_1,A \fCenter \Delta_1$}
\BinaryInfC  {$\Gamma_0,\Gamma_1 \fCenter \Delta_0,\Delta_1$}
\end{prooftree}
\end{center}
%%
\LLaddUnitRule
\LLwithRule
\LLoplusRule
%%
\LLmultUnitRule
\LLotimesRule
\LLimpRule
\LLnegRule

\caption{Multiplicative Additive Linear Logic}
\end{table}


\bibliography{../BiB/LinearLogicPfenning,../BiB/AttackTree,../BiB/algebra} 
