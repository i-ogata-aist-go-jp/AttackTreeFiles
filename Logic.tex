\def\fCenter{\entails}

In this subsection, we recall necessary definitions and
notations. 
%
% Formulas
%
\begin{definition}[formulas]
{\em Formulas} ranged over by $A,B,C,\ldots$ are built up by the
constructor from a set of propositional variables. 
%%%
The formulas are inductively defined as follows:
\begin{enumerate}
\item Each propositional variable ranged over $P,Q,R,\ldots$ is a formula.
\item Each constant $\LLbot$, $\LLtop$, $\LLone$, $\LLzero$ is a formula. 
\item If $A$ is a formula, then $\lneg{A}$ is a formula.
\item If $A$ and $B$ are formulas, 
and $\cdot$  is any binary connective, then $(A \cdot B)$ is a formula.
\end{enumerate}
\end{definition}

Here $\cdot$  are  
$\LLoplus$, $\with$, $\LLotimes$ or $\parr$. 
%%%
Here is a definition by Backus-Naur Form;
\begin{bnf*}
	\bnfprod{propositional variable}  
    	{P,Q,R,\ldots} \\
    \bnfprod{constant}  
    	{\LLbot,\LLtop,\LLone,\LLzero} \\
    \bnfprod{formula} {
		\bnfpn{propositional variable} \bnfor
		\bnfpn{constant} \bnfor
		\lneg{\bnfpn{formula}} \bnfor
		\bnfpn{formula} \bnfsp \cdot \bnfsp \bnfpn{formula}
	}
\end{bnf*}

%
% Context
%
\begin{definition}[context]
{\em Contexts} ranged over by $\Gamma,\Delta$ are {\bf multisets} of  formulas. 
Comma means taking union as multisets. Thus, the set $\Gamma_0
\cup \Gamma_1$ is denoted by ``$\Gamma_0,\Gamma_1$''.
\end{definition}
%
% Sequents
%
\begin{definition}[sequents]
A Gentzen-style {\em Sequents} are of the form:
\[{\Gamma} \Rightarrow {D} \]
where $\Rightarrow$ is the {\bf entailment sign} of the calculus.
\end{definition}
%
% Multiplicative
% 
The use of multisets of formulas in the interpretation of the sequent 
eliminates the need for an explicit permutation rule. 
This corresponds to shifting associativity and commutativity 
outside the sequent calculus.

The use of multisets also implies 
that we have to handle multiplicative and additive rules differently. 
It also means that we have to include contraction rules explicitly. 
This corresponds to including idempotency inside the sequent calculus. 
The name multiplicative and additive is from Girard's linear logic\cite{Girard87tcs}.
It is the most common nowadays. 
%
In classical logic, one can use structural rules of weakening and contraction for free.
Hence these distinct formulations of the sequent calculi are interderivable, 
and the two styles of the binary connectives are provably equivalent. 
Because of this, the distinction is often overlooked
in the previous literature. For example, in Gentzen's original
definition of {\sf LK}, the conjunction and the
disjunction is defined as additive, while the implication is defined
as multiplicative. In multiplicative style, divided contexts in the
conclusion is the union of divided contexts in the premises.

Surprisingly, some small changes in the rules of {\sf LK} ({\sf CLL}) suffice 
to turn it into a proof system for {\sf LJ} ({\sf ILL} respectively). 
One has to restrict to sequents with exactly one formula on the right-hand side, 
and modify the rules to maintain this invariant. 



\begin{table}
\begin{center}
\begin{prooftree}
\RightLabel{Ax}\AxiomC{} \UnaryInfC {$A \fCenter A$}
	\DisplayProof \hskip 48pt
\RightLabel{Cut}
\AxiomC {$\Gamma_0 \fCenter A$}
\AxiomC {$\Gamma_1,A \fCenter D$}
\BinaryInfC  {$\Gamma_0,\Gamma_1 \fCenter D$}
\end{prooftree}
\end{center}
%%
\LLaddUnitRule
\LLwithRule
\LLoplusRule
%%
\LLmultUnitRule
\LLotimesRule
\LLimpRule
\LLnegRule

\caption{Multiplicative Additive Linear Logic}
\end{table}
