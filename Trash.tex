\newcommand{\LTCnegA}{\LTCnegNP{A}}
\newcommand{\LTCdnegA}{\LTCnegNP{\LTCnegNP{A}}}
\newcommand{\LTCnegB}{\LTCnegNP{B}}
\newcommand{\LTCimpAA}{\LTCimpNP{A}{A}}
\newcommand{\LTCimpAB}{\LTCimpNP{A}{B}}
\newcommand{\LTCimpAC}{\LTCimpNP{A}{C}}
\newcommand{\LTCimpAD}{\LTCimpNP{A}{D}}
\newcommand{\LTCimpBA}{\LTCimpNP{B}{A}}
\newcommand{\LTCimpBB}{\LTCimpNP{B}{B}}
\newcommand{\LTCimpBC}{\LTCimpNP{B}{C}}
\newcommand{\LTCimpBD}{\LTCimpNP{B}{D}}
\newcommand{\LTCimpCA}{\LTCimpNP{C}{A}}
\newcommand{\LTCimpCB}{\LTCimpNP{C}{B}}
\newcommand{\LTCimpCD}{\LTCimpNP{C}{D}}
\newcommand{\LTCimpABC}{\LTCimpNP{A}{(\LTCimpBC)}}
\newcommand{\LTCimpBAC}{\LTCimpNP{B}{(\LTCimpAC)}}
\newcommand{\LTCimpAAA}{\LTCimpNP{A}{(\LTCimpAA)}}
\newcommand{\LTCimpABA}{\LTCimpNP{A}{(\LTCimpBA)}}
\newcommand{\LTCimpBAB}{\LTCimpNP{B}{(\LTCimpAB)}}
\newcommand{\LTCimpOneTwoThree}[3] {\LTCimpNP{#1} {\LTCimpP{#2}{#3}}}
\newcommand{\LTCimpOneTwoThreeP}[3]{(\LTCimpOneTwoThree{#1}{#2}{#3})}
\newcommand{\LTCmeetAB}{\LTCmeetNP{A}{B}}
\newcommand{\LTCmeetAC}{\LTCmeetNP{A}{C}}
\newcommand{\LTCmeetBD}{\LTCmeetNP{B}{D}}
\newcommand{\LTCmeetCD}{\LTCmeetNP{C}{D}}
\newcommand{\LTCjoinAB}{\LTCjoinNP{A}{B}}
\newcommand{\LTCotimesAB}{\LTCotimesNP{A}{B}}
\newcommand{\LTCmeetimpOneTwoThree}[3]{\LTCimpNP{\LTCmeetP{#1}{#2}}	{#3}}
\newcommand{\LTCmeetimpABC}{\LTCmeetimpOneTwoThree{A}{B}{C}}
\newcommand{\LTCnegNP}[1]{\LTCneg #1}

\newcommand{\LTCimpNP}[2]{#1 \LTCimp #2}
\newcommand{\LTCimpP}[2]{(\LTCimpNP{#1}{#2})}
\newcommand{\LTCmeetNP}[2]{#1 \LTCmeet #2}
\newcommand{\LTCmeetP}[2]{(\LTCmeetNP{#1}{#2})}
\newcommand{\LTCjoinNP}[2]{#1 \LTCjoin #2}
\newcommand{\LTCjoinP}[2]{(\LTCjoinNP{#1}{#2})}
\newcommand{\LTCotimesNP}[2]{#1 \LTCotimes #2}
\newcommand{\LTCotimesP}[2]{(\LTCotimesNP{#1}{#2})}



\newcommand{\LTCeqNP}[2]{#1 \LTCeq #2}
\newcommand{\LTCeqP}[2]{(\LTCeqNP{#1}{#2})}



\begin{proposition}\label{joinEndomorphism}
Let $L$ be a join-semilattice. 
A function $c \vee \QTargument : L \longrightarrow L $
is a homomorphism (thus order-preserving). 
\end{proposition}
\begin{proof}
\[c \vee (a \vee b) =  (c \vee c) \vee (a \vee b) = (c \vee a) \vee (c \vee b) \]
\end{proof}

%% IL-algebra  Adjoing Functor Theorem

\begin{proposition} [IL-algebra is distributive]
$\QTotimes$ distributes over $\vee$. That is,
for any family $(b_i)_{i \in I}$ of elements of $L$ and any $a \in L$, 
\[ a \QTotimes \QTlub b_i = \QTlub (a \QTotimes b_i) \]
holds.
\end{proposition}
\begin{proof}
$(\QTlub b_i) \QTotimes a \leq c $ iff
$\QTlub b_i \leq a \QTmultimap c$  iff
$ b_i \leq a \QTmultimap c$  for each $i \in I$ iff
$ b_i \QTotimes a \leq c$  for each $i \in I$ iff
$\QTlub (b_i \QTotimes a) \leq c$. 
Then consider the cases when 
$c = (\QTlub b_i) \QTotimes a$  and
$c = \QTlub (b_i \QTotimes a)$. 
\end{proof}

The propositon exactly says that every IL-algebra is distibutive. It directly follows a complete IL-algebra (i.e. complete as a lattice) is a quantale. 

\begin{proposition} [ Adjoint functor theorem (2) ]
If $\QTleftAdjoint$ has a right adjoint $\QTrightAdjoint$, 
it preserves all joins which exist in $Q$. 
\end{proposition}
\begin{proof}
We have to check (1) $a \QTotimes \bigvee X$ is an upper bound, 
and (2) it is the least one. \\
(1) 
Since $\QTleftAdjoint$ is order preserving, 
$ a \QTotimes \bigvee X$ is clearly an upper bound for
$\set{a \QTotimes x | x \in X}$. \\
(2) If $b$ is any upper bound for $X$, 
then we have $ a \QTotimes x \leq b$ for all $x \in X$,
it directly follows $x \leq a \QTmultimap b$ for all $x \in X$, 
$\bigvee X \leq a \QTmultimap b$ and $a \QTotimes \bigvee X \leq b$. 
\end{proof}
%


\begin{definition} [IL-Algebra] \label{IL-Algebra}
An {\em IL-Algebra} is a lattice $I$ 
equipped with two additional binary operations:
\begin{enumerate}
\item 
$\LLotimes$ from $ {\QTargument} { \LLotimes} {\QTargument} : I \times I \longrightarrow I$ and
\item
$\LLimp$ from $ {\QTargument} { \LLimp} {\QTargument} : I \times I \longrightarrow I$.
\end{enumerate}
Then $\LLotimes$ (called {\em multiplicative and}) and 
$\LLimp$ (called {\em multiplicative implication}) makes $I$ into an {\em IL-algebra}  
if, for each pair of elements $(a,b)$, there exists an element $a \LLimp b$
such that $\LLresiduateLeft$ iff $\LLresiduateRight$.
%
This bi-directional inference rule, namely the law of (multiplicative) residuation, is pictorially represented as
%
\begin{prooftree}
	\AxiomC{$\LLresiduateLeft$}
    \doubleLine
    \UnaryInfC{$\LLresiduateRight$}
\end{prooftree}
\end{definition}

\begin{proposition}
\label{closureLeast}
Let $\closureOP$ be a closure operator on a poset $P$.
For each $x \in P$, $\closure{x}$ is the least element of $\CcapPfilterX$
(i.e. 
$\closure{x}$ is the least closed element which includes $x$). 
\end{proposition}
\begin{proof}
\begin{enumerate}
\item $\closure{x} \in C$ : 
	$\dclosure{x} = \closure{x} $ by idempotence,
\item $\closure{x} \in \PfilterX$ : $x \leq \closure{x}$  by extensivity,
\item $\closure{x}$ is the least one: 
if $y \in \PfilterX$ ($x \leq y$),
then $\closure{x} \leq \closure{y}$ by monotonicity.
If $y \in C$  ($\closure{y} = y$), so $\closure{x} \leq y$.
\end{enumerate} 
\end{proof}

Explicitly, for any subset $A$ of $L$, consider the set  $X$ of upper bounds for $A$. 
Let
\[ X  =  \set{ x \in L |  \forall a \in A (a \leq x) } \]
and set
\[ \bigoplus A = \bigwedge X \mbox{.} \]

There is another construction. 
