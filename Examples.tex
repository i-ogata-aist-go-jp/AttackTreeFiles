%%

\subsection{A concrete example --- de morgan lattice}


\begin{table}[hbt]
\begin{center}
   \begin{tabular}{c|ccc}
  $\oplus$ & F &  M  & T \\ \hline
  F   &  F & M  & T   \\
  M  &     & M  & T   \\
  T   &     &     & T   \\
  \end{tabular}
%
\hskip 18pt\begin{tabular}{c|ccc}
  $\with$ & F  & M & T \\ \hline
        F   & F & F & F \\
        M   &   & M & M \\
        T   &   &     & T \\
  \end{tabular}
%
\hskip 18pt\begin{tabular}{c|ccc}
  $\otimes$ & F  & M & T \\ \hline
        F   & F & F & F \\
        M   &   & M & T \\
        T   &   &     & T \\
  \end{tabular}
%
\hskip 18pt
 \begin{tabular}{c|c}
 $\lnegP{\_}$ &  \\ \hline
F & T \\
M & M \\
T & F \\
 \end{tabular}
\end{center}
\caption{multiplication tables for phase space operators}
\label{table:interpretation5}
\end{table}


\subsection{A concrete example}

\newcommand{\PSsetT}{\set{0,a,a^2,a^3,\PSidentity}}
\newcommand{\PSsetH}{\set{0,a,a^2,a^3}}
\newcommand{\PSsetM}{\set{0,a^2,a^3}}
\newcommand{\PSsetL}{\set{0,a^3}}
\newcommand{\PSsetF}{\set{0}}

\begin{table}[hbt]
\begin{center}
  \begin{tabular}{c|ccccc}
  $\cdot$ & 0 & $a$ & $a^2$ & $a^3$ & $\PSidentity$ \\ \hline
0 & 0 & 0 & 0 & 0 & 0 \\
$a$ & & $a^2$ & $a^3$ & 0 & $a$ \\
$a^2$ & &  & 0 & 0 & $a^2$ \\
$a^3$ & &  &   & 0 & $a^3$ \\
$\PSidentity$ & & & & & $\PSidentity$ \\
\end{tabular}
\end{center}
\caption{five-elements commutative monoid subject to $x \cdot 0 = 0$ and $a^4 = 0$}
\label{multTable}
\end{table}

\begin{table}[t]
\begin{center}
   \begin{tabular}{c|ccccc}
  $\oplus$ & F & L & M & H & T \\ \hline
  F   & F & L & M & H & T   \\
  L   &  & L & M & H & T   \\
 M   &  & & M  & H & T   \\
 H   &  & &  & H & T   \\
 T   &   &   &      &    & T \\
  \end{tabular}
\hskip 36pt\begin{tabular}{c|ccccc}
  $\otimes$ & F & L & M & H & T \\ \hline
        F       & F & F & F & F & F \\
        L       &   & F & F & F & L \\
        M      &   &    &  F & L & M \\
        H      &   &   &      & M & H \\
        T      &   &   &      &    & T \\
  \end{tabular}
\hskip 36pt
 \begin{tabular}{c|c}
 $\lnegP{\_}$ &  \\ \hline
F & T \\
L & H \\
M & M \\
H & L \\
T & F \\
 \end{tabular}
\end{center}
\caption{multiplication tables for phase space operators}
\label{interpretationTable}
\end{table}

Some of the earliest descriptions of attack trees are found in papers and articles by Bruce Schneier. He gave a various leaf-node values assignment methods. 
Among them, the simplest of these values are boolean:  possible / impossible for example. 
This means that the any attack  is categorized either critical or safe in a model based on classical logic. 
%
However, in a practical world  where zero risk is impossible, 
we assign a discrete risk score  to a threat.
Each threat is given a score of  False,  Low/Medium/High risk or True with True means the threat is critical and an attack to the treat is fully effective. 
%
In our framework, we could give a five-valued model to express the above risk score method.
Consider a phase space  $(M , \cdot , \PSidentity,\PSzero)$.
$M = \PSsetT$ is a five-element monoid and ${\cdot}$ is defined in the obvious way subject to $a^4 = 0$.
We also give an explicit multiplication table in Table~\ref{multTable}.
%
We define  $\PSzero = \set{0}$. 
%
Now, we seek for all the facts in the phase space.

\begin{proposition} 
We have exactly  5 facts in the phase space in question as follows:
%
\begin{eqnarray*}
\mbox{True} &=& \PSunit = \lneg{\PSsetF}  = M = \PStrue = \lneg{\mbox{False}} \\
\mbox{High} &=& \dlneg{\set{a}} = \lneg{\PSsetL} = \PSsetH  = \lneg{\mbox{Low}} \\
\mbox{Medium} &=& \dlneg{\set{a^2}} = \lneg{\PSsetM} = \PSsetM = \lneg{\mbox{Medium}} \\
\mbox{Low} &=& \dlneg{\set{a^3}} = \lneg{\PSsetH} = \PSsetL = \lneg{\mbox{High}} \\
\mbox{False} &=& \PSfalse = \lneg{M} =  \PSsetF = \PSzero  = \lneg{\mbox{True}} \\
\end{eqnarray*}
%
Note that the five facts are totally ordered by strict inclusion: 
 $ \mbox{False} \subset \mbox{Low} \subset \mbox{Medium} \subset \mbox{High} \subset \mbox{True}$. 
\end{proposition}

\begin{proof}
The fact which we designate ``True'' and ``False''  is the collapsed units appeared 
in example~\ref{example:collapsedUnits}). Thus the set of these two facts, 
$(\set{\PSEfalse,\PSEtrue},\oplus,\otimes,\lnegP{\cdot},\PSEfalse,\PSEtrue)$,
 is a two-valued Boolean algebra. 
%
``High'' is the smallest fact that includes $a$, 
but there is no larger fact because $\mbox{High} = \mbox{True} \setminus \set{\PSidentity}$. Identical argument for the rest: 
 $\mbox{Medium} = \mbox{High} \setminus \set{a}$ and
 $\mbox{Low} = \mbox{Medium} \setminus \set{a^2}$. Therefore we exactly have 5 facts described above in this phase space. 
\end{proof}
%
We  give  explicit multiplication tables for the set of five facts
in Table~\ref{interpretationTable}.
%

\begin{remark}
 If  both $A \LLvdash B$ and $B \LLvdash A$ are valid sequent, then $A$ and $B$ are said to be equivalent. 
Thanks to the soundness theorem,  the interpretation of equivalent formulas are equal in phase space for any assignment for variables. 
For example the following equality holds regardless of assignment values $\PSintA$,$\PSintB$ and $\PSintC$. 
%
$\interpretP{ \alpha  \otimes (\beta  \oplus \gamma)}
 = \interpretP{ (\alpha \otimes \beta) \oplus ( \alpha \otimes \gamma)}$. 
\end{remark}






\subsection{A concrete example --- three elements monoid}


\newcommand{\PSt}{\set{0,a,\PSidentity}}
\newcommand{\PSm}{\set{0,\PSidentity}}
\newcommand{\PSf}{\set{0}}

\begin{table}[hbt]
\begin{center}
 \begin{tabular}{c|ccc}
  $\cdot$ & 0 & $\PSidentity$ & $a$ \\ \hline
0 & 0 & 0 & 0 \\
$\PSidentity$ &  & $\PSidentity$ & $a$ \\
$a$ & & & $a$ \\
  \end{tabular}
\hskip 36pt\begin{tabular}{c|ccc}
  $\cdot$ & 0 & $\PSidentity$ & $a$ \\ \hline
0 & 0 & 0 & 0 \\
$\PSidentity$ &  & $\PSidentity$ & $a$ \\
$a$ & & & $a$ \\
  \end{tabular}
\hskip 36pt\begin{tabular}{c|ccc}
  $\cdot$ & 0 & $\PSidentity$ & $a$ \\ \hline
0 & 0 & 0 & 0 \\
$\PSidentity$ &  & $\PSidentity$ & $a$ \\
$a$ & & & $a$ \\
\end{tabular}
\end{center}
\caption{three-elements commutative monoid}
\label{table:mult3}
\end{table}

\begin{table}[hbt]
\begin{center}
   \begin{tabular}{c|ccc}
  $\oplus$ & F &  M  & T \\ \hline
  F   &  F & M  & T   \\
  M  &     & M  & T   \\
  T   &     &     & T   \\
  \end{tabular}
\hskip 36pt\begin{tabular}{c|ccc}
  $\otimes$ & F  & M & T \\ \hline
        F   & F & F & F \\
        M   &   & M & T \\
        T   &   &     & T \\
  \end{tabular}
\hskip 36pt
 \begin{tabular}{c|c}
 $\lnegP{\_}$ &  \\ \hline
F & T \\
M & M \\
T & F \\
 \end{tabular}
\end{center}
\caption{multiplication tables for phase space operators}
\label{table:interpretation3}
\end{table}








\subsection{basic examples}
%%%
\begin{example} [Boolean algebra] 
%
Let $(M, \cdot, \PSidentity, \PSzero = \PSemptyset)$
be a phase space.  
Then $M$ and $\PSemptyset$ are the only facts and we get a two-valued Boolean algebra
( $\PSunit = \lnegP{\PSzero} = \lneg{\PSemptyset} = M = \PStrue$ and
$\PSfalse = \lnegP{\PStrue} = \lneg{M} = \PSemptyset$). 
For any non-empty $X \subset M$, $X$ is not a fact($\dlneg{X} = \lneg{\PSemptyset} = M$).
The set of all facts of this phase space, 
$(\set{\PSemptyset,M},\oplus,\otimes,\lneg{\PSargument},\PSemptyset,M)$,  is a two-valued Boolean algebra. 
\end{example}
%%
\begin{example}
Let $(\set{a,\PSidentity}, \cdot, \PSidentity,\set{a})$ be a phase space subject to $a \cdot a = a$. 
Then
$\PSunit = \lneg{\set{a}} = M = \PStrue$ and $\PSfalse = \lneg{M} = \set{a} = \PSzero$ are the only facts. 
Neither $\set{\PSidentity}$ nor $\PSemptyset$ is a fact. 
\end{example}
%%
%%
\begin{example} \label{example:collapsedUnits}
Consider a commutative monoid that has an element $0$ 
such that for every  $a \in M$  we have $0 \cdot a = 0$.
%
Let $(M, \cdot, \PSidentity, \PSzero = \set{0})$
be a phase space. Again, $M$ and $\set{0}$ are the ``collapsed'' phase space units (
$\PSunit =  \lneg{\set{0}} = M = \PStrue$
and 
$\PSfalse = \lneg{M} = \set{0} = \PSzero$). 
Neither $\set{\PSidentity}$ nor $\PSemptyset$ is a fact, 
because $\dlneg{\set{\PSidentity}} = \lnegP{\PSzero} = \PSunit = M$ and 
$\dlneg{\PSemptyset} = \lnegP{\PStrue} = \PSfalse = \set{0}$. However there may be other facts in this phase space. 
\end{example}
%%
\begin{example}
Let $(\set{a,\PSidentity}, \cdot, \PSidentity)$ be a commutative monoid subject to $a \cdot a = \PSidentity$ (i.e., it is a cyclic group of order 2). 
\\
(1)  Let $(\set{a,\PSidentity}, \cdot, \PSidentity, \set{a} )$
 be a phase space.  Then 
$\PStrue = \set{a,\PSidentity}$,
$\PSunit = \lneg{\set{a}} = \set{\PSidentity}$,
$\PSzero = \set{a} = \lneg{\set{\PSidentity}}$,
$\PSfalse = \lneg{\set{a,\PSidentity}} = \PSemptyset$.
\\
(2)  Let $(\set{a,\PSidentity}, \cdot, \PSidentity, \set{\PSidentity} )$
 be a phase space. Then we also have four facts as follows: 
$\PStrue = \set{a,\PSidentity}$, 
$\PSunit = \lneg{\set{\PSidentity}} = \set{\PSidentity} = \PSzero$,
$\set{a} = \lneg{\set{a}}$ ,
$\PSfalse = \lneg{\set{a,\PSidentity}} = \PSemptyset$.
%
\end{example}


\begin{example}
Let $M = \PSsetofintegers$ (the set of integers) with addition and $\PSzero = \set{0}$. 
That is, let $(\PSsetofintegers,+,0,\set{0})$ be a phase space. 
In this case, the set of facts consists of $\PSsetofintegers$, $\PSemptyset$ and all singleton sets: 
$\PStrue = \PSsetofintegers$, 
$\PSunit = \lneg{\set{0}} = \set{0} = \PSzero$,
$\PSfalse = \lneg{\PSsetofintegers} = \PSemptyset$ and
$\set{a} = \lneg{\set{-a}}  \mbox{ for any } a \in \PSsetofintegers$.
%
There is no other facts, $\dlneg{\set{0,1}} = \lneg{\PSemptyset} = \PSsetofintegers$ for instance.  
If we identify singleton sets with its only respective element, 
then $\otimes$ are identified with addition, i.e., $\set{1} \otimes \set{2} = \dlneg{\set{1+2}} = \set{3}$.  
Moreover $\PSfalse \oplus \PSfalse = \PSfalse$  and 
$X \oplus Y = \PStrue \mbox{ (otherwise) }$.  
For instance, 
$\dlnegP{\set{0} \cup \set{1}} = \dlneg{\set{0,1}} = \PSsetofintegers$. 
\end{example}






%%
\subsection{Examples based on Frames}
A complete Heyting algebra is a complete lattice 
where $\QTleftAdjoint$ has a right adjoint $\QTrightAdjoint$. 
Hence Adjoint functor Theorem says that frames and complete Heyting algebras amount to the same thing.
%
In that case, soundness theorem says that the set of inference rules for intuitionistic linear logic is also valid as inference rules for intuitionistic logic. 
%%
\begin{proposition}
The set of integers ($\PSsetofintegers$) 
in their usual order form a distributive lattice, 
under the operations of $\min{\QTargumentT}$ (meet) 
and $\max{\QTargumentT}$ (join). 
\end{proposition}
\begin{proof}
Without loss of generosity, we can assume $b \leq c$, 
then it follows $\min(b,c) = b$ and $\max(b,c) = c$. 
\begin{eqnarray*}
\min(a,\max(b,c)) &= \min(a,b) =& \max(\min(a,b),\min(a,c)) \\
\max(a,\min(b,c)) &= \max(a,b) =& \min(\max(a,b),\max(a,c)) 
\end{eqnarray*}
In general, if the distributivity law holds in a lattice, then so its dual:
$ (a \vee b) \wedge (a \vee c) 
= (a \wedge a) \vee (a \wedge b) \vee (a \wedge c) \vee ( b \wedge c) 
= (a \vee ( a \wedge b)) \vee  (a \vee ( a \wedge c)) \vee ( b \wedge c)
=  a \vee ( b \wedge c)$.
\end{proof}
%
Any non-empty finite lattice is trivially complete,
and,  in a real world,  it suffices to consider finite values. 
Therefor we consider the finite subset of integers which forms a complete distributive lattice. 
%

\begin{example}
$(\set{0,1},\min{},0,\max{})$ is a complete Boolean algebra. 
We interpret 0 as ``impossible'' and 1 as ``possible''.
The valuation could be understood as ``is the attack possible''. 
\end{example}
%%
\begin{example}
$(\set{0,1,2,3,4},\min{},0,\max{})$ is a complete Heyting algebra.
We interpret integers as a degree of skill levels ( $0$ is the highest skill level). 
The valuation could be understood as ``minimum skill level required to perform attack''. 
\end{example}
%%
\subsection{Examples based on Quantales }
Next, we consider a quantale, 
in which the monoid  is ``additive'' operation on $\PSsetofnnint$.
%%
\begin{example}
$(\set{0, \ldots, 100},+,0,\max{})$ is a quantale. 
We interpret integers as damages to the information system. 
The valuation could be understood as ``maximal damage''. 
\end{example}
%%
\begin{example}
Similarly, $(\set{0, \ldots, 10000},+,0,\min{})$ is a quantale with reverse order. 
We interpret integers as costs  required to perform attack. 
The valuation could be understood as ``cost of the cheapest attack''.
\end{example}

