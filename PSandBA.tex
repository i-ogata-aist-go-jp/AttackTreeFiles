
\subsection{Phase Space}
\def\fCenter{\subseteq}

\begin{definition}[power set] 
Given a set $M$,
the power set of  $M$ is the set of all subsets of $M$
including the $\emptyset$ (empty set) and $M$ itself.
The power set of $M$ is denoted by $\PSpowerset{M}$.
\end{definition}
%%
\begin{fact}
$(\PSpowerset{M},\cup,\emptyset,\cap,M,{\PSargument}^{\complement}) $ 
is always a complete boolean algebra, 
where $\cup$(set union), $\emptyset$(empty set),
$\cap$(set intersection) and ${\PSargument}^{\complement}$(set complement) 
are the usual set-theoretic operations.
It is ordered by $\subseteq$(set inclusion).
\end{fact}
%
\begin{definition}
Let $(M,\PScdot,\PSidentity)$ be a commutative monoid. 
A binary function
%
$ {\PSargument} { \PScdot} {\PSargument} : M \times M \longrightarrow M$ 
%
is naturally extended to binary operations on $\PSpowerset{M}$:
 \begin{eqnarray*}
 X \otimes Z  &=&  \bigcup \set{ x \PScdot z | x \in X, z \in Z} \\
 X \multimap Y  &=&  \bigcup \set{ z  | \forall x \in X (x \PScdot z \in Y)} 
 \end{eqnarray*}
 for any subset $X$,$Y$,$Z$ of $M$.
\end{definition}

That is, 
\begin{enumerate}
\item 
$ X \otimes Z$ contains all the possible interactions 
between one element of $x \in X$ and one element of $z \in Z$. 
\item 
$ X \multimap Y$ contains all the possible elements $z \in Z$
such that $\forall x \in X (x \PScdot z \in Y)$.
\end{enumerate}

\begin{fact}
Let $(M,\cdot,\PSidentity) $ be a commutative monoid.
Then 
$(\PSpowerset{M},\RLotimes,\set{\PSidentity})$ 
is also a commutative monoid.
\end{fact}
\begin{proof}
\begin{enumerate} 
\item {\em neutrality law}:  for every  $X \subseteq M$  
we have $\set{\PSidentity} \RLotimes X
= \bigcup \set{  \PSidentity  \PScdot x  | x \in X} 
= X$,
\item {\em associative law}:  for every $X,Y,Z \subseteq M$,
we have 
$X \RLotimes ( Y \RLotimes Z)  \\
= \bigcup \set{ x \PScdot (y \PScdot z) | x \in X, y \in Y, z \in Z} 
= \bigcup \set{ (x \PScdot y) \PScdot z | x \in X, y \in Y, z \in Z}  \\
= (X \RLotimes Y) \RLotimes Z$,
\item {\em commutative law}: for every $X,Y \subseteq M$,
we have 
$X \RLotimes Y  
= \bigcup \set{ x \PScdot y  | x \in X, y \in Y} 
= \bigcup \set{ y \PScdot x  | x \in X, y \in Y} 
= Y \RLotimes X$.
\end{enumerate}
\end{proof}
%%
%%
\begin{proposition}
Let $(M,\cdot,\PSidentity) $ be a commutative monoid.
Then 
$(\PSpowerset{M},\cup,\emptyset,\cap,M,\multimap,\otimes,\set{\PSidentity})$ 
is a residuated lattice.
\end{proposition}
\begin{proof}
\begin{enumerate}
\item {\em lattice}: 
$(\PSpowerset{M},\cup,\emptyset,\cap,M)$ 
is a lattice. In fact, it is a complete distributive lattice.
\item {\em commutative monoid}:
The extension of $\PScdot$ to $\PSpowerset{M}$  induces  a commutative monoid $(\PSpowerset{M},\otimes,\set{\PSidentity}) $. 
\item {\em residual law}: 
for each pair of elements $(X,Y)$, 
there exists an element $X \QTmultimap Y$  
such that residual law holds: \hskip -7cm
\begin{prooftree}
	\AxiomC{$ X \otimes Z \fCenter Y$}
	\doubleLine
	\UnaryInfC{$ Z \fCenter X \multimap Y$}
\end{prooftree}
\end{enumerate}
\end{proof}



\def\fCenter{\leq}

\subsubsection{Heyting Algebras}

Consider a special kind of commutative residual lattices, 
in which the commutative monoid and the meet operation coincide 
(i.e. $a \QTotimes b = a \wedge b$, $\RLone = \LTCtop$). 
Such lattices are called {\em Heyting Algebras}.

\newcommand{\LTCresiduateLeft}{a \LTCmeet c \fCenter b}
\newcommand{\LTCresiduateRight}{c \fCenter a \LTCimp b}

\begin{definition} [Heyting Algebra] \label{HeytingAlgebra}
An {\it Heyting Algebra} is a sextuple $(L,\LTCjoin,\bot,\LTCmeet,\top,\LTCimp)$.
The following conditions hold:
\begin{enumerate}
\item {\em lattice}: $(L,\LTCjoin,\LTCbot,\LTCmeet,\LTCtop)$ is a lattice,
\item {\em residual law}:  for each pair of elements $(a,b)$, 
there exists an element $\LTCimpab$ such that the residual law holds:
\begin{prooftree}
	\AxiomC{$\LTCresiduateLeft$}
    \doubleLine
    \UnaryInfC{$\LTCresiduateRight$}
\end{prooftree}
\end{enumerate}
\end{definition}

Next proposition says the conditions on residual lattices. 
%%
\begin{definition} [weakening and contraction]
Let $A$ be a subset of a residual lattice $L$. 
\begin{enumerate}
\item $A$ satisfies {\it weakening (or strictly two-sided) } 
iff $\forall a \in A (a \leq \RLone)$.
\item $A$ satisfies {\it contraction (or idempotent)} 
iff $\forall a \in A ( a \otimes a = a)$. 
\end{enumerate}
\end{definition}
%
\begin{proposition} \label{RL:WeakeningContraction}
A residual lattice $L$ satisfies both weakening and contraction 
iff $L$ is a Heyting Algebra. 
\end{proposition}
\begin{proof}
$(\Rightarrow)$  We want to prove $\LTCtop = \bigwedge\emptyset = \RLone$ 
and  $a \wedge b = a \QTotimes b$. 
\begin{enumerate}
\item {$\RLone = \LTCtop$}: 
The weakening condition says $\RLone$ is the unique greatest element.
%
\item 
\begin{enumerate}
\item If $L$ satisfies weakening 
then $a \RLotimes b \leq a \LTCmeet b$.
\item If $L$ satisfies contraction
then $a \LTCmeet b \leq a \RLotimes b $.
\end{enumerate}
\begin{prooftree}
 \AxiomC{$a \fCenter a$}
 \AxiomC{$b \fCenter \RLone$}
\BinaryInfC{$a \RLotimes b \fCenter a \RLotimes \RLone$}
\UnaryInfC{$a \RLotimes b \fCenter a$}
%
 \AxiomC{$a \fCenter \RLone$}
 \AxiomC{$b \fCenter b$}
\BinaryInfC{$a \RLotimes b \fCenter \RLone \RLotimes b$}
\UnaryInfC{$a \RLotimes b \fCenter b$}
%
\BinaryInfC{$a \RLotimes b \fCenter a \LTCmeet b$}
\DisplayProof \hskip 48pt
 \AxiomC{$a \LTCmeet b \fCenter a$}
 \AxiomC{$a \LTCmeet b \fCenter b$}
\BinaryInfC{$(a \LTCmeet b) \RLotimes (a \LTCmeet b) \fCenter a \RLotimes b$}
\UnaryInfC{$a \LTCmeet b \fCenter a \RLotimes b$}
\end{prooftree}
\end{enumerate}
$(\Leftarrow)$ 
The converse is obvious. 
\end{proof}