% Names for tuples of specific lengths  1 single  2 double  3 triple/treble  4 quadruple
% 5 quintuple pentadruple   6 sextuple hexatruple  7 septuple  8 octuple  
%  9 nonuple  10 decuple  11 undecuple hendecuple  12 duodecuple  13 tredecuple 100 centuple

\subsection{IL- and CL-algebras}
%%
\begin{definition}[IL-algebra]
An  {\it IL-algebra} is an octuple
$(R,\LLoplus,\with,\LLzero,\LLtop,\LLotimes,\LLimp,\LLone)$.
The following conditions hold for any $a,b,c \in R$:
\begin{description}
\item [lattice]
$(R,\oplus,\with,\LLzero,\LLtop)$ is a lattice, where
$(R,\oplus,\LLzero)$ is a join semilattice and 
$(R,\with,\LLtop)$ is a  meet semilattice,
\item [commutative monoid] 
$(R,{\LLotimes},\LLone)$ is a commutative monoid, 
% \item [distributive law]
% for any $a,b,c \in R$, 
% $ a \LLotimes (b \LLoplus c)  =   (a \LLotimes b) \LLoplus (a \LLotimes c)$ holds.
\item [distributibity]
$c \LLotimes (a \LLoplus b) = (c \LLotimes a) \LLoplus (c \LLotimes b)$ and
%
% stability
% if $a \leq a'$,$b \leq b'$, then $a \LLotimes b \leq a' \LLotimes b'$
%and $a' \LLimp b \leq a \LLimp b'$ for any $a,a',b,b' \in R$,
%
% \item [adjunction]  $c \LLotimes a \leq b$ iff $ c \leq a \LLimp b$.
%
\end{description}
\end{definition}
%%
\begin{proposition}  [stability of $\LLotimes$] In any IL-algebra,
if $a \leq a'$ and $b \leq b'$, then $a \LLotimes b \leq a' \LLotimes b'$.
\end{proposition}
\begin{proof}
Recall that $c \leq c'$ iff  $c \LLoplus c' = c'$.
So the assumption is $a \LLoplus a' = a'$ and $b \LLoplus b' = b'$,
whereas the claim is 
$ (a \QTotimes b) \LLoplus (a' \QTotimes b') = a' \QTotimes b'$. 
But this claim can be proved easily by distributivity: 
 $ (a \QTotimes b) \LLoplus (a' \QTotimes b')
\leq (a \QTotimes b) \LLoplus (a \QTotimes b') \LLoplus (a' \QTotimes b) \LLoplus (a' \QTotimes b')
= (a \LLoplus a') \QTotimes (b \LLoplus b')
= a' \QTotimes b'$. 
\end{proof}
%%
\begin{definition} [implication]
Consider the join of all elements of $c$ satisfying $c \LLotimes a \leq b$.
Let $X = \set{ x \in R | x \LLotimes a \leq b}$,
and we set $ a \LLimp b \defeq \bigvee X$ to produce an operation, namely {\em implication}, 
$ {\QTargument} { \LLimp } {\QTargument} : Q \times Q \rightarrow Q$.
\end{definition}
%%
\begin{proposition}
In any IL-algebra,
$ (a \LLimp b)  \LLotimes a  \leq b$.
\end{proposition}
\begin{proof}
All elements $x_i \in X$ 
satisfy the equation $ a \QTotimes x_i  \leq b$.
Thus it directly follows $\bigvee a \QTotimes x_i \leq b$, then
$a \QTotimes \bigvee X \leq b$ by distributivity. 
\end{proof}
%%
\begin{proposition}  [an instance of the adjoint functor theorem] In any IL-algebra,
$c \LLotimes a \leq b$ iff $ c \leq a \LLimp b$.
\end{proposition}
\begin{proof}
$(\Rightarrow)$  
Since $a \LLimp b = \bigvee X$ is an upper bound for $X$, 
it is obvious that any $x \in X$ is smaller than $\bigvee X$. \\
%%
$(\Leftarrow)$  
We already know that  $ (a \LLimp b) \QTotimes a  \leq b$.
It directly follows $x \leq a \LLimp b$ implies $x \LLotimes a \leq b$ 
by stability of $\LLotimes$. 
\end{proof}
%%
\begin{proposition} In any IL-algebra, 
$a \LLotimes \LLzero = \LLzero$ and $  \LLzero \LLimp \LLzero = \LLtop$. 
\end{proposition}
\begin{proof}
$ \LLzero \leq a \LLimp \LLzero$ iff $\LLzero \LLotimes a \leq \LLzero$
iff $a \leq \LLzero \LLimp \LLzero$. 
\end{proof}
%%
\begin{definition}[CL-algebra]
An  {\it CL-algebra} is a nonuple 
$(R,\LLoplus,\with,\LLzero,\LLtop,\LLotimes,\LLimp,\LLone,\LLbot)$, 
which is an IL-algebra with an additional constant $\LLbot$. 
We set $\lneg{a} \defeq a \LLimp \LLbot$. 
The following condition hold for any $a \in R$:
\begin{description} 
\item [double negation eliminaion] $ \dlneg{a} = a$.
\end{description}
\end{definition}
%
% \begin{proposition}
%The distributive law; 
%$a \LLotimes (b \LLoplus c)  =   (a \LLotimes b) \LLoplus (a \LLotimes c)$ %holds.
%\end{proposition}
%%
%\begin{proof}
%($\geq$) Since $ \LLleftAdjoint :  Q \rightarrow Q$ is an order-preserving map, 
%we have 
%$\LLatimesbS \leq \LLsrcFormS$ and
%$\LLatimescS \leq \LLsrcFormS$, hence
%$\LLdistFormS \leq \LLsrcFormS$. \\
%($\leq$)
%$a \LLotimes b \leq \LLdistFormS$ implies $b \leq a \LLimp (\LLdistFormS)$
%and similarly $c \leq a \LLimp (\LLdistFormS)$. Thus
%$b \oplus c \leq a \LLimp (\LLdistFormS)$ and we conclude
%$\LLsrcFormS \leq \LLdistFormS)$
%\end{proof}
%
%
%
%
%
%
