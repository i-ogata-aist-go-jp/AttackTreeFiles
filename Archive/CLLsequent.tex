\subsection{Traditional Interpretation}

Traditionally, attack trees are seen as representation of AND ($\wedge$)-OR ($\vee$) formulae. 
Its algebraic semantics is the two-element (TRUE/FALSE) boolean algebra --- every leaf-node is assigned either TRUE or FALSE. TRUE means a success of attack, and FALSE means a failure. The value of each AND-OR non-leaf node is caluculated according to truth tables. If the root node is calculated as TRUE,  the set of leaf-nodes that are assigned to be TRUE is sufficient to to mount the root. Such collection of possible attacks are called as  {\it intrusion scenario}. 

The main difference can be explained in two points.

\begin{itemize}
\item idempotency, i.e., $A \vee A = A$,  $A \wedge A = A$
\item distributivity $\vee$ over $\wedge$, i.e.,  $A \vee ( B \wedge C) =  (A \vee B) \wedge (A \vee C) $
\end{itemize}

\def\fCenter{\LLvdash}

\begin{table} 
%%
%% Axiom and Cut 
\begin{center}
\begin{prooftree}
\AxiomC{} \RightLabel{Ax}\UnaryInfC {$A \fCenter A$}
%
\DisplayProof \hskip 48pt
%
\AxiomC {$\Gamma_0,A \fCenter D$}
\AxiomC {$\Gamma_1 \fCenter A$} 
\RightLabel{cut} \BinaryInfC  {$\Gamma_0,\Gamma_1 \fCenter D$}
\end{prooftree}
\end{center}
%%
%%  additives
%%
\begin{center}
\begin{prooftree}
\AxiomC{} \RightLabel{L$\LLzero$}\UnaryInfC {$\Gamma,\LLzero \fCenter D$}
\DisplayProof \hskip 96pt
\AxiomC{} \RightLabel{R$\LLtop$}\UnaryInfC {$\Gamma \fCenter \LLtop$}
\end{prooftree}
\end{center}
%
\begin{center}
\begin{prooftree}
\AxiomC {$\Gamma,A_i \fCenter  D$}
\RightLabel{$\with$L $(i = 1,2)$} \UnaryInfC  {$\Gamma,A_1 \with A_2 \fCenter D$}
%
\DisplayProof  \hskip 48pt
%
\AxiomC {$\Gamma \fCenter A$} 
\AxiomC {$\Gamma \fCenter B$}
\RightLabel{$\with$R} \BinaryInfC  {$\Gamma \fCenter A \with B$}
%
\end{prooftree}
\end{center}
%%
\begin{center}
\begin{prooftree}
\AxiomC {$\Gamma,B \fCenter D$} 
\AxiomC {$\Gamma,C \fCenter D$}
\RightLabel{$\oplus$L} \BinaryInfC  {$\Gamma, B \oplus C  \fCenter D$}
%
\DisplayProof \hskip 48pt
%
\AxiomC {$\Gamma \fCenter A_i$}
\RightLabel{$\oplus$R $(i = 1,2)$} \UnaryInfC  {$\Gamma \fCenter A_1 \oplus A_2$}
%
\end{prooftree}
\end{center}
%
% multiplicatives
%
\begin{center}
\begin{prooftree}
\AxiomC{$\Gamma \fCenter D$} \RightLabel{L$\LLone$}
\UnaryInfC {$\Gamma,\LLone \fCenter D$}
%
\DisplayProof \hskip 96pt
%
\AxiomC{} \RightLabel{R$\LLone$} \UnaryInfC {$\fCenter \LLone$}
\end{prooftree}
\end{center}
%
\begin{center}
\begin{prooftree}
\AxiomC {$\Gamma,A,B \fCenter D$}
\RightLabel{$\LLotimes$L} \UnaryInfC  {$\Gamma,A \LLotimes B \fCenter D$}
%
\DisplayProof \hskip 32pt
%
\AxiomC {$\Gamma_0 \fCenter A$} 
\AxiomC {$\Gamma_1 \fCenter B$}
\RightLabel{$\LLotimes$R} \BinaryInfC  {$\Gamma_0,\Gamma_1 \fCenter A\LLotimes B$}
\end{prooftree}
\end{center}
%%

\begin{center}
\begin{prooftree}
\AxiomC {$\Gamma_0 \fCenter A$} 
\AxiomC {$\Gamma_1,B \fCenter D$}
\RightLabel{$\LLimp$L} \BinaryInfC  {$\Gamma_0,\Gamma_1, A \LLimp B \fCenter D$}

%
\DisplayProof \hskip 32pt
%
\AxiomC {$\Gamma,A \fCenter B$}
\RightLabel{$\LLimp$R} \UnaryInfC  {$\Gamma \fCenter A \LLimp B$}
\end{prooftree}
\end{center}



\begin{minipage}{\textwidth}      
\begin{center}
\begin{prooftree}
\AxiomC {$\Gamma \fCenter $ A}
\RightLabel{$\lnegArg$L} \UnaryInfC  {$\Gamma,\lnegA \fCenter \LLbot$}
%
\DisplayProof  \hskip 36pt
%
\AxiomC {$\Gamma,A \fCenter \LLbot $}
\RightLabel{$\lnegArg$R} \UnaryInfC  {$\Gamma \fCenter \lnegA$}
%
\DisplayProof  \hskip 36pt
%
\AxiomC{} \RightLabel{DNE~\footnote{DNS = double negation elimination}} 
\UnaryInfC {$\dlnegA \fCenter A$}
%
\end{prooftree}
\end{center}
\end{minipage}
\caption[Example] 
 { $\with$$\oplus$$\otimes$$\lnegArg$-fragment of Classical Linear Logic}
\label{table:inferenceRules}
\end{table}


\subsection{Classical Linear Logic (CLL) without modality}

Classical Linear logic is usually represented by using Gentzen's sequent-style calculus. 
Gentzen's sequent calculus is a style of formal logical argumentation where every line of a proof is a conditional tautology. 
Each sequent is inferred from other sequent on earlier lines according to inference rules.
This concept directly applies to the concepts of attack trees. 

The syntax  is exactly a  Girard's classical linear logic without modality. 
Hence we start to explain its syntax. 
%%
\begin{definition} [formula]
The formula of linear logic are defined as follows:  
starting from a set of the variables, named $ \set{\alpha,\beta,\gamma,\bnfsk}$, 
they are freely build using the binary connectives $\set{\otimes,\oplus}$.
Formally, the set of formulae is described by the following BNF grammar.
%
\begin{bnf*}
 \bnfprod{variable}{ \alpha,\beta,\gamma,\ldots }  \\
 \bnfprod{formula}{ \bnfpn{variable} 
	\bnfor \bnfpn{multiplicative} 
	\bnfor \bnfpn{additive}}\\
 \bnfprod{multiplicative}{ {\LLzero} 
	\bnfor {\LLtop} 
	\bnfor \bnfpn{formula} \LLoplus \bnfpn{formula}
	\bnfor \bnfpn{formula} \with \bnfpn{formula}}  \\
 \bnfprod{additive}{ {\LLbot} 
	\bnfor {\LLone}
	\bnfor \bnfpn{formula} \LLotimes \bnfpn{formula}  
	\bnfor \bnfpn{formula} \LLimp \bnfpn{formula} 
	\bnfor {\lneg{\bnfpn{formula}}}}
\end{bnf*}
\end{definition}

\begin{definition} [sequent] 
 {\it Sequents}  are of the form:
%
\[  \Gamma \LLvdash B \mbox{,} \]
%
where $B$ is a formula and  $\Gamma$  is a multiset of formulas.
$\Gamma$ may be empty.
If $\Gamma$ is not empty, it is represented  as comma-separated formulas, 
i.e., $\Gamma = A_1, \ldots , A_n$. 
A comma to the left of the turnstile should be thought of as an $\otimes$.
%
Mutliset is a set-like object in which order is ignored, but multiplicity is explicitly significant. 
Therefore, contexts $A,B,C$ and $B,C,A$ are equivalent, 
but $A,A,B,C$ and $A,B,C$ differ.
\end{definition}

\begin{definition} [proof]
A (formal) {\it proof} is a sequence of sequents,
where each of the sequents is derivable from sequents appearing earlier in the sequence
 by using inference rules described in Table~\ref{table:inferenceRules}. 
Otherwise said, a proof is arranged in tree form: 
the tree always starts from Axiom, followed by other inference rules. 
%
A sequent is {\it provable} if it has a formal proof, i.e., can be derived by the method above. 
\end{definition}

\subsection{A Proof-theoretical account }

\begin{definition} [attack scenario and attack tree]
We interpret in classical linear logic.
\begin{enumerate}
\item An {\it attack scenario} is a context $\Gamma = A_1, \ldots , A_n$,
 identified with a formula $C = A_1 \otimes, \ldots, \otimes A_n$ . 
\item An {\it attack tree} is a formula $B$.
\end{enumerate}
\end{definition}

\begin{theorem} [valid attack scenario]
$\Gamma$ is a successful attack scenario for an attack tree $B$ 
if the sequent $\Gamma \LLvdash B$ is provable.  
\end{theorem}

\begin{proof} 
By induction of proofs. 
\begin{description}
\item[Axiom]
Trivial. $A$ is an attack scenario of an attack tree $A$. 
\item[$\otimes$R]
If $  \Gamma_0$  is an attack scenario of $A$ and $\Gamma_1$ is an attack scenario of $B$, 
then $\Gamma_0,\Gamma_1$ is an attack scenaio of $A\otimes B$. 
This means that attacker must try BOTH $\Gamma_0$ AND $\Gamma_1$. 
Since contexts are multisets, multiplicity is essential: 
an attack (formula) may occur more than once in an attack scenario (multiset of formulas). 
\item[$\otimes$L]
If $  \Gamma,A,B$  is an attack scenario of $D$,  
then $\Gamma,A\otimes B$ is an attack scenario of $D$. 
This exactly means that a comma in $\Gamma$ is identified with $\otimes$. 
Specifically a context $A,A$ is equivalent to $A \otimes A$,  
however,  it differs a single occurrence of $A$. 
%
\item[$\oplus$L]
If both $\Gamma,A$ and $\Gamma,B$ is an attack scenario of $D$, then $\Gamma,A\oplus B$ is an attack scenario of $D$.  
This rule says,
if an attacker can choose EITHER $C \otimes A$ OR $C \otimes B$  in order to attack  an attack tree $D$,
then $C \otimes (A \oplus B)$ is also an attack scenario for $D$,
 i.e.,  distributivity of $\otimes$ over $\oplus$ holds. 
Where we identify $\Gamma = A_1,\ldots, A_n$ with $C = A_1 \otimes,\ldots, \otimes A_n$. 
%
\item[$\oplus$R]
If $  \Gamma$  is an attack scenario of $A_i$ , then $\Gamma$ is an attack scenario of $A_1 \oplus A_2$ ($i = 1,2)$.  An attacker can choose EITHER $A_1$ OR $A_2$ to attack $A_1 \oplus A_2$. 
\end{description}
\end{proof}

