%
% Rules
%
\begin{table}
\begin{prooftree}
\RightLabel{Ax}  \AxiomC{}  \UnaryInfC {$A \fCenter A$}
\end{prooftree}
\begin{prooftree}
\AxiomC	{$\Gamma \fCenter A$} 
\AxiomC{$\Delta\fCenter A \RLimp B$}  
\RightLabel{M.P.}
\BinaryInfC{$\Gamma,\Delta \fCenter B$}
	\DisplayProof  \hskip 32pt
\AxiomC{$\Gamma \fCenter A$} 
\AxiomC{$\Gamma \fCenter B$}  
\RightLabel{Conjunction}
\BinaryInfC{$\Gamma \fCenter A \with B$}
\end{prooftree}
\caption{Inference Rules: Deduction from Assumption}
\end{table}

\newcommand{\DTwea} { \alwaysNoLine \AxiomC{$\Pi_1$}  
	\UnaryInfC{$\Gamma \fCenter B$}}

\newcommand{\DTcon} { \alwaysNoLine \AxiomC{$\Pi_1$}  
	\UnaryInfC{$\Gamma,A,A  \fCenter B$}}


\newcommand{\DTmpaONE}{ 	\alwaysNoLine 		
	\AxiomC{$\Pi_1$} 		\UnaryInfC{$ \Gamma,A \fCenter B$} 	}
\newcommand{\DTmpaTWO}{  \alwaysNoLine 
	\AxiomC{$\Pi_2$}{ \UnaryInfC{$ \Delta \fCenter \LTCimpBC$}}}

\newcommand{\DTmpbONE}{ 	\alwaysNoLine 		
	\AxiomC{$\Pi_1$} 		\UnaryInfC{$ \Gamma \fCenter B$} 	}
\newcommand{\DTmpbTWO}{  \alwaysNoLine 
	\AxiomC{$\Pi_2$}{ \UnaryInfC{$ \Delta,A  \fCenter \LTCimpBC$}}}

\newcommand{\DTadjONE} { \alwaysNoLine \AxiomC{$\Pi_1$}  
	\UnaryInfC{$\Gamma, C \fCenter A$}}
\newcommand{\DTadjTWO} { \alwaysNoLine \AxiomC{$\Pi_2$}
	\UnaryInfC{$\Gamma, C \fCenter B$}}

\newcommand{\HBTleadsto}
	{\hskip 6pt \raisebox{-16pt}{\LARGE $\leadsto$}  \hskip 6pt }
\newcommand{\HBTleadstoAx}
	{\hskip 24pt \raisebox{0pt}{\LARGE $\leadsto$}  \hskip 24pt }
\newcommand{\HBTleadstoWea}
	{\hskip 24pt \raisebox{-6pt}{\LARGE $\leadsto$}  \hskip 24pt }
\newcommand{\HBTleadstoCon}
	{\hskip 24pt \raisebox{-12pt}{\LARGE $\leadsto$}  \hskip 24pt }

%%%

We now advance to a theory of Deduction from assumption. 

\subsubsection{Internal and External deductions}

In sequent calculi, we can distinguish several notions of deduction from hypotheses. 

\begin{enumerate}
\item $B$ is an internal consequence of $\Gamma$  iff 
the sequent $ \Gamma \LLvdash B$ is a conditional tautology. 
%
\item $B$ is an external consequence of $\Gamma$  iff  
the sequent $\LLvdash B$ is derivable from hypotheses: 
$\LLvdash A_1, \ldots, \LLvdash A_n$.  We write 
\begin{prooftree}
\AxiomC{$\LLvdash A_1$}
\AxiomC{$\ldots$}
\AxiomC{$\LLvdash A_n$}
\TrinaryInfC{$\LLvdash B$}
\end{prooftree}
\end{enumerate}
%
Our important motivation stems from the search for the notation of deduction where $B$ is said to follow an assumption $\Gamma$ in an explicit manner.  Suppose $\Gamma$ is a context( i.e., multiset of formulas). 
%
In this case, we say $B$ is an internal consequence of $\Gamma$ and write $\Gamma \fCenter B$. 
%
Thus, informally, ``a sequent $\Gamma \fCenter B$ is derivable'' means that $B$ is provable assuming all the formulas in $\Gamma$. 

Another notion of deduction from hypotheses is an external consequence;  a sequent $S$ is externally deducible (i.e., derivable)  from a multiset $X$ of sequents if $S$ can be derived with the element of $X$.



\begin{lemma}[Deduction Theorem]
The deduction theorem holds, i.e.,
\[
 \Gamma, A \fCenter B  \hskip 48pt \mbox{implies} \hskip 48pt
  \Gamma \fCenter \LTCimpAB
\]
\end{lemma}
%
\begin{proof}
By induction on the length of derivations. 
\flushleft{\textbf{\large base case}}
\begin{prooftree}
\RightLabel{Ax}  \AxiomC{}  \UnaryInfC {$A \fCenter A$}
	\DisplayProof \HBTleadstoAx
\RightLabel{Axiom I}  \AxiomC{}  \UnaryInfC {$\fCenter \LTCimpNP{A}{A}$}
\end{prooftree}
%
\flushleft{\textbf{\large induction step}}
% Weakening
\begin{prooftree}
	\DTwea
		\alwaysSingleLine
			\RightLabel{Weakening}
			\UnaryInfC{$\Gamma,A \fCenter B$}
\DisplayProof 	\HBTleadstoWea
	\DTwea
		\alwaysSingleLine
			\RightLabel{K}
			\UnaryInfC{$\Gamma \fCenter \LTCimpAB$}
\end{prooftree}
% Contraction
\begin{prooftree}
	\DTcon
		\alwaysSingleLine
			\RightLabel{Contraction}
			\UnaryInfC{$\Gamma,A \fCenter B$}
\DisplayProof 	\HBTleadstoCon
	\DTcon
		\alwaysSingleLine
			\RightLabel{I.H.}
			\UnaryInfC{$\Gamma \fCenter \LTCimpNP{A}{(\LTCimpAB)}$}
			\RightLabel{W}
			\UnaryInfC{$\Gamma \fCenter \LTCimpAB$}
\end{prooftree}
% Modus Ponens A
\begin{prooftree}
	\DTmpaONE
	\DTmpaTWO
		\alwaysSingleLine
		\RightLabel{M.P.} 
		\BinaryInfC{$\Gamma, \Delta, A \fCenter C$}
\DisplayProof 	\HBTleadsto
	\DTmpaONE
		\alwaysSingleLine
		\RightLabel{I.H.}
		\UnaryInfC{$ \Gamma \fCenter \LTCimpAB$}
%
	\DTmpaTWO
		\alwaysSingleLine
			\RightLabel{B} 
			\BinaryInfC{$\Gamma, \Delta  \fCenter \LTCimpAC$}
\end{prooftree}
% Modus Ponens B
\begin{prooftree}
	\DTmpbONE
	\DTmpbTWO
		\alwaysSingleLine
		\RightLabel{M.P.} 
		\BinaryInfC{$\Gamma, \Delta, A \fCenter C$}
\DisplayProof  \HBTleadsto
	\DTmpbONE
%
	\DTmpbTWO
		\alwaysSingleLine
		\RightLabel{I.H.} 	\UnaryInfC{$ \Gamma \fCenter \LTCimpABC$}
		\RightLabel{C} 		\UnaryInfC{$\Gamma, \Delta  \fCenter \LTCimpBAC$}
	\RightLabel{M.P.} 
	\BinaryInfC{$\Gamma, \Delta  \fCenter \LTCimpAC$}
\end{prooftree}
% Adjunction
\begin{prooftree}
	\DTadjONE    	\DTadjTWO
		\alwaysSingleLine
		\RightLabel{Adj.} 
		\BinaryInfC{$\Gamma,C  \fCenter \LTCmeetAB$}
\DisplayProof
\hskip 12pt \mbox{\LARGE $\leadsto$}  \hskip 12pt 
	\DTadjONE \alwaysSingleLine \RightLabel{I.H.} 
		\UnaryInfC {$\Gamma \fCenter \LTCimpCA$}
	\DTadjTWO \alwaysSingleLine \RightLabel{I.H.}
		\UnaryInfC {$\Gamma \fCenter \LTCimpCB$}
	\RightLabel{Adj.} 		
	\BinaryInfC{$\Gamma, \Delta  \fCenter \LTCmeetNP{(\LTCimpCA)}{(\LTCimpCB)}$}
			\RightLabel{$\LTCmeet$ intro.} 
			\UnaryInfC{$\Gamma, \Delta  \fCenter \LTCimpNP{C}{(\LTCmeetAB)}$}
\end{prooftree}
\end{proof}

\subsubsection{Cut}

