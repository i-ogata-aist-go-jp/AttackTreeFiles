\def\fCenter{\subseteq}

\section{Phase Space}

\begin{definition}[power set] 
Given a set $M$,
the power set of  $M$ is the set of all subsets of $M$
including the $\emptyset$ (empty set) and $M$ itself.
\end{definition}
The power set of $M$ is denoted by $\PSpowerset{M}$.
%%
Recall that 
$(\PSpowerset{M},\cup,\emptyset,\cap,M,{\PSargument}^{\complement}) $ 
is always a complete boolean algebra, 
where $\cup$(set union), $\emptyset$(empty set),
$\cap$(set intersection) and ${\PSargument}^{\complement}$(set complement) 
are the usual set-theoretic operations. It is ordered by $\subseteq$(set inclusion).
%
\begin{definition}
Let $(M,\PScdot,\PSidentity)$ be a commutative monoid. A binary function
%
$ {\PSargument} { \PScdot} {\PSargument} : M \times M \longrightarrow M$ 
%
is naturally extended to binary operations on $\PSpowerset{M}$:
 \begin{eqnarray*}
 Z \otimes X  &=&  \bigcup \set{ z \PScdot x | z \in Z,  x \in X} \\
 X \multimap Y  &=&  \bigcup \set{ z  | \forall x \in X (z \PScdot x \in Y)} 
 \end{eqnarray*}
 for any subset $X$,$Y$ of $M$.
That is, 
$ Z \otimes X$ contains all the possible interactions 
between one element of $z \in Z$ and one element of $x \in X$. 
$ X \multimap Y$ contains all the possible elements $z \in Z$
such that $\forall x \in X (z \PScdot x \in Y)$.
\end{definition}
%%
\begin{proposition}
Let $(M,\cdot,\PSidentity) $ be a commutative monoid.
$(\PSpowerset{M},\cup,\emptyset,\cap,M,\multimap,\otimes,\set{\PSidentity}) $ 
is an IL-algebra.
\end{proposition}
\begin{proof}
\begin{enumerate}
%
\item {\em lattice}: $(\PSpowerset{M},\cup,\emptyset,\cap,M)$ 
is a lattice. 
%
\item {\em commutative monoid}:
The extension of $\PScdot$ to $\PSpowerset{M}$  induces  a commutative monoid $(\PSpowerset{M},\otimes,\set{\PSidentity}) $. 
%
\item {\em residual law}: 
for each pair of elements $(X,Y)$, 
there exists an element $X \QTmultimap Y$  
such that residual law holds: \hskip -7cm
\begin{prooftree}
	\AxiomC{$ Z \otimes X \fCenter Y$}
	\doubleLine
	\UnaryInfC{$ Z \fCenter X \multimap Y$}
\end{prooftree}
\end{enumerate}
\end{proof}

%%%

\begin{definition}[phase space]
A phase space is a quadruple $(\PSpowerset{M},\cdot,\set{\PSidentity},\PSzero)$ 
where
\begin{enumerate}
\item {\em commutative monoid}:$(\PSpowerset{M},{\otimes},\set{\PSidentity}) $ 
is a commutative monoid.
\item {\em multiplicative zero}: ${\PSzero}$ 
is a designated subset of $M$ ($ \PSzero \subseteq M$) 
called {\it multiplicative zero}. 
\end{enumerate}
\end{definition}
%%
The elements of $\PSpowerset{M}$ are called phases. 

%
\begin{definition}[relation]
A relation is any subset of a Cartesian product.  A subset of  $X \times Y$, called a binary relation from $X$ to $Y$,
is a collection of ordered pairs $(x,y)$ with first components from $X$ and second components from $Y$.
\end{definition}
%%
\begin{definition}[linear negation]
We write $\bot$ for the relation  $\set{ (x,y) | x \cdot y \in \PSzero}$.
We define the linear negation operator  $\lneg{\PSargument} : X \longrightarrow Y$ as  $y \in \lneg{X}  \mbox{ iff for every }  x \in X \mbox{  we have  }  (x,y) \in \bot $.
\end{definition}
%
%\begin{definition}[linear negation]
%We define the linear negation operator  
%$\lneg{\PSargument} : X \longrightarrow Y$ as follows:
%
% \[ y \in \lneg{X}  \mbox{ iff for every }  x \in X \mbox{  we have  }  (x,y) \in \bot \]
%
%\end{definition}
An easy consequence of the above definition is that 
$\lneg{\PSemptyset} = M$,
$\lneg{\set{\PSidentity}} = \PSzero$
  and $X \otimes \lneg{X} \subseteq \PSzero$. 
%
\begin{proposition} For any subset $X,Y,Z$ of $M$, we have the following property. 
\begin{enumerate}
\item $Z \subseteq X$  implies  $Z \otimes \lneg{X} \subseteq \PSzero$. 
\item $X \otimes Y \subseteq \PSzero$  iff $Y \subseteq \lneg{X}$.
\end{enumerate}
\end{proposition}
\begin{proof}
(1) For every $z \in Z \subseteq X$ and $y \in \lneg{X}$ we have $(z,y) \in \bot$. 
(2) From assumption, for every $x \in X$  and $y \in Y$  we have $(x,y) \in \bot$. Hence $Y \subseteq \lneg{X}$ by definition. Conversely, if $ y \in Y \subseteq \lneg{X}$, we have $(x,y) \in \bot$ for every $x \in X$ and $y \in Y$. 
\end{proof}
%
%
\begin{proposition} [closure operator]
For any subset $X,Y$ of $M$, $\dlneg{\PSargument}$ is the closure operator on $X,Y$. 
Explicitly, $\dlneg{\PSargument} : X \longrightarrow X$ is a closure operator if the following conditions hold:
%
\begin{enumerate}
\item $X \subseteq \dlneg{X}$.
\item If $X \subseteq Y$ then $\dlneg{X} \subseteq \dlneg{Y}$.
\item $\dlneg{(\dlneg{X})} \subseteq \dlneg{X}$.
\item $\dlneg{X} \otimes \dlneg{Y} \subseteq \dlneg{(X \otimes Y)}$.
\end{enumerate}
\end{proposition}
%
\begin{proof}
\begin{prooftree}
\AxiomC {$ X \fCenter X$}
\RightLabel{(1)} \UnaryInfC  {$X \otimes \lneg{X} \fCenter \PSzero$}
\RightLabel{(2)} \UnaryInfC  {$X \fCenter \dlneg{X}$}
\DisplayProof \hskip 24pt
%
\AxiomC {$ X \fCenter Y $}
\RightLabel{(1)} \UnaryInfC  {$X \otimes \lneg{Y} \fCenter \PSzero$}
\RightLabel{(2)} \UnaryInfC  {$\lneg{Y} \fCenter \lneg{X}$}
\RightLabel{(1)(2)} \UnaryInfC  {$\dlneg{X} \fCenter \dlneg{Y}$}
\DisplayProof \hskip 24pt
%
\AxiomC {$\lneg{X} \fCenter \dlneg{(\lneg{X})}$}
\RightLabel{(1)(2)} \UnaryInfC  {$\dlneg{(\dlneg{X})} \fCenter \dlneg{X}$}
\end{prooftree}
%
\begin{prooftree}
\AxiomC {$ X \otimes Y \fCenter X \otimes Y$}
\RightLabel{(1)}
 \UnaryInfC {$ X \otimes Y \otimes \lneg{( X \otimes Y)} \fCenter \PSzero$}
\RightLabel{(2)(1)(2)(1)}
\UnaryInfC {$ \dlneg{X} \otimes \dlneg{Y}  \otimes \lneg{( X \otimes Y)} 
                \fCenter \PSzero$}
\RightLabel{(2)}
\UnaryInfC  {$\dlneg{X} \otimes \dlneg{Y} \fCenter \dlneg{(X \otimes Y)}$}
\end{prooftree}
\end{proof}

\begin{definition} [fact] 
A fact in a phase space is a fixed point for the closure operator. Explicitly, for any subset $X$ of $M$, $X$ is a fact if $X = \dlneg{X}$. 
\end{definition}

\begin{proposition}  \label{propOfFact}
We have the following property about facts:
\begin{enumerate}
\item $\lneg{X}$ is a fact. 
\item $\dlneg{X}$ is the smallest fact that includes $X$. 
\end{enumerate}
\end{proposition}
%
\begin{proof}
(1) $\lneg{X}$ is a fact because of $\lneg{X} = \dlneg{(\lneg{X})}$. 
(2) Suppose $Y$ is any fact that includes $X$  (i.e., $Y = \dlneg{Y}$ and $X \subseteq Y$) .
Then $X \subseteq Y$ implies $\dlneg{X} \subseteq \dlneg{Y} = Y$.
It directly follows that $\dlneg{X}$ is the smallest among those facts. 

\end{proof}
%
\begin{proposition}
The set of facts of a phase space is a complete lattice. 
\end{proposition}
\begin{proof}
We define phase space operators, 
$\PSlub$ (least upper bound, lub) and $\PSglb$ (greatest lower bound, glb),  
for $(X_i)_{i \in I} \subseteq \PSpowerset{M}$, a family of subsets of $M$,  as follows:
%
\[
\PSlub = \dlnegP{\bigcup_{ i \in I}  X_i} 
\hskip 24pt \mbox{and} \hskip 24pt
\PSglb = \bigcap_{ i \in I}  X_i \mbox{.}
\]
%
$\PSlub$ is obviously a fact, it suffices to prove it is indeed a lub.
%%
$\PSlub$ is an upper bound 
because $X_i \subseteq \bigcup X_i$ implies $X_i = \dlneg{X_i} \subseteq \PSlub$.
%%
If $Y$ is an upper bound, i.e., $X_i \subseteq Y$ for all $i$, 
then $\cup X_i \subseteq Y$,  $\PSlub \subseteq \dlneg{Y} = Y $, 
so $\PSlub$ is indeed a lub within the set of facts. \\
%%
$\PSglb$ is obviously a glb, it suffices to prove it is indeed a fact. 
$\PSglb \subseteq X_i$ implies $\dlnegP{\PSglb} \subseteq \dlneg{X_i} = X_i$ for each $i$, 
hence $\dlnegP{\PSglb} \subseteq \PSglb$. \\
\end{proof}

%%%
\subsection{phase space semantics}

\begin{proposition} [phase space units]
According to the corresponding logical units of linear logic, we designate four facts as follows:
\begin{itemize}
\item $\PSzero$ (multiplicative zero) and  $\PSunit = \lnegP{\PSzero} $ (multiplicative unit).
\item $ \PStrue = M$ (additive true) and $\PSfalse = \lneg{M}$ (additive false). 
\end{itemize}
\end{proposition}
\begin{proof}
$\PStrue = M = \lneg{\PSemptyset}$ and $\PSzero = \lneg{\set{\PSidentity}}$ are facts from proposition~\ref{propOfFact}.
\end{proof}
%
Notice also that $\PSfalse = \dlneg{\PSemptyset}$ is the smallest fact 
out of all facts in a phase space.
$\PSunit = \dlneg{\set{\PSidentity}}$ is the smallest fact containing $\PSidentity$. 

\begin{definition} [phase space operators]
We define the phase space operators as follows:
\begin{eqnarray*}
X \with Y =  X \cap Y,
 &\phantom{XXXXX}&  X \oplus  Y  = \dlnegP{X \cup Y}, \\ 
X \parr  Y  = \lnegP{\lneg{X} \otimes \lneg{Y}} ,
 &\phantom{XXXXX}& X \otimes Y = \dlnegP{X \otimes Y},\\
\wn X = \lnegP{\lneg{X} \cap \PSexpStandard},
 &\phantom{XXXXX}& \oc X = \dlnegP{X \cap \PSexpStandard},
\end{eqnarray*}
where $\PSexpStandard = \set{ x \in \PSunit | x = x^2 }$. 
\end{definition}
%%%
\begin{proposition}
$\otimes$ distributes over $\bigoplus$. 
\end{proposition}
\begin{proof}
$ Y \cdot X_i \subseteq Y \cdot (\PSlub)$ implies 
$Y \otimes X_i \subseteq \PSdist$ for each $i$, 
and therefore $\bigcup (Y \otimes X_i) \subseteq \PSdist$. 
This implies $ \PSsrc  \subseteq \PSdist $.
Conversely,  $\PSdist = \dlnegP{\dlneg{Y} \cdot \dlnegP{\bigcup X_i}}
= \dlnegP{Y \cdot \bigcup X_i} = \dlnegP{ \bigcup (Y \cdot X_i)} =  \PSpre$. 
But $\PSpre \subseteq \PSsrc$, this completes the proof. 
\end{proof}
%%
\begin{definition} [phase space model]
Given a formula $A$ of linear logic and an assignation that associate a fact $\alpha^{M}$ to any variable $\alpha$. 
A phase space model is a phase space together with a fact  for each (propositional) variables. 
We define a interpretation $\interpret{\PSargument}$ of a formula  $A$ in a phase space$(M,\cdot,\PSidentity,\PSzero)$  by structural induction as follows:
\begin{itemize}
\item $\interpretP{X \otimes Y} = \interpret{X} \otimes \interpret{Y}$.
\item $\interpretP{X \oplus  Y}   = \interpret{X} \oplus \interpret{Y}$.
\end{itemize}
$\interpret{\PSargument}$ is lifted to contexts (multiset of formulas)  as  
%
$\interpretP{A_1,\ldots,A_n} 
   = \interpretP{A_1} \otimes \ldots \otimes \interpretP{A_n}$.
%
\end{definition}


\begin{theorem}[Soundness]
Let $\Gamma \LLvdash B$ be a provable sequent in linear logic.
Then $ \interpret{\Gamma} \subseteq \interpret{B}$ for any assignment for variables. 
\end{theorem}
\begin{proof}
By induction  on proofs.
\end{proof}


