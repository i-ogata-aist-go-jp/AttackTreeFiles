


\subsection{Quotient Algebra}

\begin{definition}[equivalence relation]
Let $A$ be a set. 
A {\em equivalence relation} on $A$ is a binary relation $\sim$ which is
\begin{enumerate}
\item {\em reflexive}: for all $a \in A$,  $a \sim a$,
\item {\em transitive}:  if $a \sim b$ and $b \sim c$, then $a \sim c$, and
\item {\em symmetric}:  $a \sim b$ iff  $b \sim a$.
\end{enumerate}
\end{definition}


\begin{definition}[equivalence class]
Let $A$ be a set with an equivalence relation $\sim$ on $A$.
The {\em equivalence class}, denoted by $[a]$ of an element $a \in A$ 
is the set  of elements which are equivalent to a. 
We call $a$ the {\em representative} of $[a]$.
Formally,
\[ 
[a] = \set{ x \in A | x \sim a}
\]
\end{definition}

\begin{proposition}
Let $A$ be a set and $\sim$ an equivalence relation on $A$. 
For any $a \in A$, $a \in [a]$.
\end{proposition}

\begin{proof}
\begin{prooftree}
\AxiomC{$ a \sim a$}
\UnaryInfC{$ a \in [a] $}
\end{prooftree}
\end{proof}

\begin{lemma}
Let $A$ be a set and $\sim$ an equivalence relation on $A$. 
These statements are equivalent:
\begin{enumerate}
\item $[a] \cap [b] \neq \emptyset$
\item $a \sim b$
\item $ [a] = [b]$
\end{enumerate}
\end{lemma}

\begin{proof}
(1)  $\Rightarrow$ (2)
There exists an $x \in A$ such that $x \in [a]$ and $x \in [b]$. 
This means $x \sim a$ and $x \sim b$, implying $a \sim b$ by transitivity. \\
%%
(2) $\Rightarrow$ (3)
Let $a \sim b$. 
For any $x \in [a]$ we have $x \sim a$. 
From $x \sim a$ and $a \sim b$, we have $x \sim b$ by transitivity,
i.e. $x \in [b]$. 
Hence $[a] \subseteq [b]$. 
Similarly we have $[b] \subseteq [a]$  because $b \sim a$ by symmetricity.
Hence [a]=[b]. \\
%%
(3) $\Rightarrow$ (1)
If $[a] = [b]$ then $[a] \cap [b] = [a] \neq \emptyset$. 
\begin{prooftree}
\AxiomC{$ x \in [a] $}
\UnaryInfC{$ x \sim  a $}
\AxiomC{$ x \in [b] $}
\UnaryInfC{$ x \sim  b $}
\BinaryInfC{$a \sim b$}
	\DisplayProof  \hskip 48pt
\AxiomC{$ x \in [a] $}
\UnaryInfC{$ x \sim  a $}
\AxiomC{$ a \sim b$}
\BinaryInfC{$ x \sim b$}
\UnaryInfC{$x \in [b] $}
\end{prooftree}
\end{proof}




\begin{proposition}
Let $A$ be a set and $\sim$ an equivalence relation on $A$. 
For an equivalence class $S$, 
any element $a \in S$ can be used as a representative of S, 
guaranteeing $[a] = S$.
\end{proposition}

In other words, the equivalence classes form a partition of $A$. 
This partition (i.e., the set of equivalence classes)
is called "the quotient set of $A$ by $\sim$"
and is denoted by $A / \sim$. 

\begin{definition}[compatible relation]
Let $A$ be a set with an equivalence relation $\sim$ on $A$.
The equivalence relation $\sim$ is said to be compatible with 
(or have the substitution property with respect to)
\item a binary operation $\QTargument \cdot \QTargument$
if for all $a_1,a_2,b_1,b_2 \in A$ 
whenever $a_1 \sim b_1$ and $a_2 \sim b_2$ implies 
$a_1 \cdot b_1 \sim  a_2 \cdot b_2$.
\end{definition}

\begin{definition}[congruence]
A {\em congruence} is an equivalence relation on an algebraic structure 
that is compatible with all the operations of an algebra.
\end{definition}

A {\em quotient algebra} is obtained 
by partitioning the elements of an algebra into equivalence classes 
given by a congruence relation.
It is straightforward to define the operations induced on $A/\sim$.

\begin{definition} [quotient algebra]
Let $(A,\cdot)$ be an algebra, where $\cdot$ are each operations in $A$.
Let $\sim$ be a a congruence relation on $A$. 
Define an operation on $A / \sim$ such that
\[ [A] \cdot [B]  = [A \cdot B] \]
Then  $ (A / \sim, \cdot)$ is called the {\em quotient algebra}. 
\end{definition}

\subsection{Soundness}

\begin{definition} [CRL mode]
A {\em CRL model} consists of a CRL 
with an interpretation $\interpret{\QTargument}$.
\end{definition}



\begin{definition} [interpretation]
Let $(M,
\interpret{\LLoplus},\interpret{\LLzero},
\interpret{\with},\interpret{\LLtop},
\interpret{\LLimp},\interpret{\LLotimes},\interpret{\LLone})$ be a CRL. 
%%
The base of an interpretation $\interpret{\QTargument}$  
is an assignation $\sigma$ 
that associate an element of CRL to any propositional variable.
%%
$\interpret{\QTargument}$ is extended to arbitrary formulas by
\[ \interpretP{A \cdot B}  =  \interpret{A} \interpret{\cdot} \interpret{B} \] 
%%
$\interpret{\QTargument}$ is extended to multisets by
\[
\interpret{\Lambda} = \interpret{\LLone}
\]
\[
\interpret{\Gamma} = 
\interpret{A_1} \interpret{\LLotimes}, \ldots, \interpret{\LLotimes}
\interpret{A_n} 
\]
\end{definition}


\begin{theorem} [soundness]
If $\Gamma \entails D$ is provable in {\MALLsystem}, 
then $\interpret{\Gamma} \leq \interpret{D}$. 
\end{theorem}
\begin{proof}
By induction on the length of derivations. 
\end{proof}

\begin{theorem} [completeness]
Let $F$ be a set of formulas of {\MALLsystem}. 
Let $\sim$ be a congruence relation defined as follows:
\[  A \sim B  
\hskip 12pt \text{iff} 
\hskip 12pt \text{both} \hskip 12pt A \entails B 
\hskip 12pt \text{and}  \hskip 12pt B \entails A 
\hskip 12pt \text{are provable in {\MALLsystem}.} \]

Let $ F / \sim $ be a quotient algebra. 
Then $ F / \sim $ is a CRL.

$ [A] \leq [B]$  is defined by  $A \with B \sim A$.

\[ [A] \with [B] = [ A \with B ] =  [A] 
\hskip 12pt \text{iff} \hskip 12pt A \with B \sim A\]

Then
\[
A \entails B  \hskip 12pt \text{iff}  \hskip 12pt  [A]  \leq [B] 
\]
\end{theorem}



