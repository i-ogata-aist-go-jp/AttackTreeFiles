The Lindenbaum–Tarski algebra of a theory in propositional logic is an algebraic (and order-theoretic) structure built out of its formula?s (modulo provable equivalence) and connectives. It thus carries an algebraic structure that corresponds to the logic in question and is generally the free such structure generated by the axioms of the theory.

Examples

The Lindenbaum–Tarski algebra of a theory in intuitionistic propositional logic is a Heyting algebra.

The Lindenbaum–Tarski algebra of a theory in classical propositional logic is a Boolean algebra.

%  Linear logics are, in fact, relevant logics 
% that lack contraction and the distribution of conjunction over disjunction


%  def  "provably equivalent" 

If two formulas $A$ and $B$ have the property that each implies the other, i.e., $A \vdash B$ and $B \vdash A$, 
then then $A$ and $B$ are said to be {\it congruent} 
and denoted by $A \cong B$. 

%% lemma   provably equivalent is  A \cong B  is a congruent relation 

\begin{enumerate}
\item {\it determination} \cite{Lincoln90simple,Lincoln94ll,Lincoln94tcs}
\item {\it reflexive}
\item {\it symmetric}
\item {\it transitive}
\item {\it structure}
\end{enumerate}


\newcommand{\LTCapprox}{\approx}

To discuss about  ``equality''

Lindenbaum-Tarski algebra 

\subsection{Lindenbaum–Tarski algebra}

The most basic equalities on formulas are

equivalence class corresponding a formula $P$ in the 

% What is Lindenbaum-Tarski algebra 

The Lindenbaum–Tarski algebra of a proof system consists of the equivalence classes of formulas. 

% Equivalence class

An equivalence class is defined as a subset of the form $\set{x \in P | x \LTCapprox A}$  where $A$ is an element (i.e., formula) of $P$ and the notation $x \LTCapprox y$  is used to mean that there is an equivalence relation between $x$ and $y$. 

% congruence 

A congruence is an equivalence relation on an algebraic object that is compatible with the algebraic structure, in the sense that the operations are well-defined on the equivalence classes. 
Every congruence relation has a corresponding quotient structure, whose elements are the equivalence classes.

The Lindenbaum–Tarski algebra is thus the quotient algebra obtained by factoring the algebra of formulas by this congruence relation.

We write $ A / \LTCapprox$  for the equivalence class corresponding to a prime formula  $A$ in the Lindenbaum-Tarski algebra. 
For example, $\LTCtop / \LTCapprox$  is an equivalence class of tautology. 

% equivalence relation 

We define an equivalence relation $\LTCapprox$  on the set of formulas  as follows:
\[  A \LTCapprox  B  \phantom{XXXXX}\mbox{if and only if}   \phantom{XXXXX}
\LTCvdash A \LTCimp  B \phantom{XXXXX}\mbox{and} \phantom{XXXXX}
  \LTCvdash B \LTCimp  A \]

% well-definedness

The operations in a Lindenbaum–Tarski algebra  are inherited from those in the underlying proof system.  They  must be well-defined on the equivalence classes  --- 
i.e., $A \LTCapprox B$ and $C \LTCapprox D$ implies:
\begin{enumerate}
\item $\LTCneg A \LTCapprox \LTCneg B$
\item  $A \LTCmeet B \LTCapprox C \LTCmeet D$ 
\item  $A \LTCjoin B \LTCapprox C \LTCjoin D$ 
\item  $A \LTCimp B \LTCapprox C \LTCimp D$.
\end{enumerate}

%  implication is a partial order on P

Then  $\LTCimp$ is a partial order $\leq$ on $P$ as follows:
\begin{enumerate}
\item The axiom I implies reflexivity  --- $A \leq A$.
\item  The axiom B implies transitivity --- if  $A \leq B$  and $B \leq C$, then $A \leq C$. 
\item  The congruence relation $\LTCapprox$ implies antisymmetry --- if $A \leq B$ and $B \leq A$, then $A \LTCapprox B$. 
\end{enumerate}


\subsection{Equality}


\newcommand{\LLdistLeaf}[1]{
\LLaxiom{A} \LLaxiom{#1}
\RightLabel{$\otimes$R}
\BinaryInfC{ $A,#1\fCenter A \otimes #1 $ }
\RightLabel{$\oplus$R}
\UnaryInfC{$A,#1\fCenter\LLdistForm$} 
}

\newcommand{\LLsrcLeaf}[1]{
\LLaxiom{A}  \LLaxiom{#1} 
\RightLabel{$\oplus$R}  \UnaryInfC{$#1 \fCenter B \oplus C$} 
\RightLabel{$\otimes$R}  \BinaryInfC{$A , #1 \fCenter \LLsrcForm$}
\RightLabel{$\otimes$L}  \UnaryInfC{$A \otimes #1 \fCenter \LLsrcForm$}
}

\begin{table}[t]  
\begin{prooftree}
\LLdistLeaf{B}  \LLdistLeaf{C}
\RightLabel{$\oplus$L} \BinaryInfC{ $A, B\oplus C\fCenter  (\LLdistForm)$}
\RightLabel{$\otimes$L} \UnaryInfC{ $\LLsrcForm \fCenter \LLdistForm$}
\end{prooftree}
%
\begin{prooftree}
\LLsrcLeaf{B}  \LLsrcLeaf{C}
\RightLabel{$\oplus$L} \BinaryInfC{ $\LLdistForm \fCenter \LLsrcForm$}
\end{prooftree}
\caption{Distributibity in Linear Logic}
\label{table:DL}
\end{table}

%%%
\subsection{Distributivity in Linear Logic}
%%%

\begin{proposition} 
Distributibity of $\otimes$ over $\otimes$ holds:

\[  \LLsrcForm =  \LLdistForm  \]
\end{proposition}
\begin{proof}
The proof is shown in Table~\ref{table:DL}
\end{proof}

% Distributibity of $\parr$ over $\with$.    \[  A \parr (B \with C) = (A\parr B)\with(A\parr C)  \]


We define XX  as a binary relation between formulas (in X) by

xxxxx


IF  A modulo B,  we say that A and B are equivalent modulo xxx.


XXX  is a congruence relation of free algebras. 

We have to demonstrate it is comptible under operators. 

L-T algebra for this logic. 

We simply define it to be the quotient algebra of the free algebra over the congruence relation . 








