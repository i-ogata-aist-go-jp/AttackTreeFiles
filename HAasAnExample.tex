\section{Heyting Algebras, Frames and Booean Algebras}



\subsection{Heyting Algebras and Frames}

The following proposition says that a Heyting algebra is a distributive lattice. 
%
\begin{proposition} [A Heyting algebra is distributive]
\label{HeytingPreservation}
A function $ c \LTCmeet \QTargument$ preserves joins.
Similarly, a function $ c \LTCimp \QTargument$ preserves meets.
\[ a \wedge  (b \vee c) =  (a \wedge b) \vee (a \wedge c)
\hskip 2cm 
 c \LTCimp (a \wedge b) =  (\LTCimpca) \wedge (\LTCimpcb) \]
\end{proposition}
\begin{proof}
Pictorially
\begin{prooftree}
		\AxiomC{$c \fCenter c$}
		\AxiomC{$a \fCenter \LTCjoinab$}
	\BinaryInfC{$\LTCmeetca \fCenter c \LTCmeet (\LTCjoinab)$}
		\AxiomC{$c \fCenter c$}
		\AxiomC{$b \fCenter \LTCjoinab$}
	\BinaryInfC{$\LTCmeetcb \fCenter c \LTCmeet (\LTCjoinab)$}
	\BinaryInfC{$\LTCjoincacb \fCenter c \LTCmeet (\LTCjoinab)$}	
\end{prooftree}
\begin{prooftree}
\AxiomC{$ \LTCmeetca \fCenter \LTCjoincacb$}
\UnaryInfC{$ a \fCenter c \LTCimp (\LTCjoincacb)$}
\AxiomC{$ \LTCmeetcb \fCenter \LTCjoincacb$}
\UnaryInfC{$ b \fCenter c \LTCimp (\LTCjoincacb)$}
\BinaryInfC{$ \LTCjoinab \fCenter c \LTCimp (\LTCjoincacb)$}
\UnaryInfC{$ c \LTCmeet (\LTCjoinab) \fCenter \LTCjoincacb$}
\end{prooftree}
%
\begin{prooftree}
		\AxiomC{$c \fCenter c$}  \AxiomC{$ \LTCmeetab \fCenter a$}
    \BinaryInfC{$c \LTCimp (\LTCmeetab) \fCenter c \LTCimp a$}
    	\AxiomC{$c \fCenter c$}  \AxiomC{$ \LTCmeetab \fCenter b$}
    \BinaryInfC{$c \LTCimp (\LTCmeetab) \fCenter c \LTCimp b$}
   	\BinaryInfC{$c \LTCimp (\LTCmeetab) \fCenter \LTCimpcacb$}
\end{prooftree}
\begin{prooftree}
		\AxiomC{$\LTCimpcacb \fCenter \LTCimpca$}
	\UnaryInfC{$c \LTCmeet \LTCimpcacb \fCenter a$}
		\AxiomC{$\LTCimpcacb \fCenter \LTCimpcb$}
	\UnaryInfC{$c \LTCmeet \LTCimpcacb \fCenter b$}
	\BinaryInfC{$c \LTCmeet \LTCimpcacb \fCenter \LTCmeetab$}
	\UnaryInfC{$\LTCimpcacb \fCenter c \LTCimp (\LTCmeetab)$}
\end{prooftree}
\end{proof}

%
%  infinite case
%
The Adjoint functor theorem tells us that 
the right adjoint $a \LTCimp \QTargument$ preserves all meets which exists in $H$
as well as the left adjoint $ a \LTCmeet \QTargument$ preserves all joins. 

\newcommand{\LTCbigwedge}{\bigvee ( a \LTCmeet B) }
\newcommand{\LTCbigimp}{\bigwedge ( a \LTCimp B )}

We use the notation to represent the subsets in lattice $L$
for all $a \in L$ and $B \subseteq L$. 

 \[ \LTCmeetaB = \set{a \LTCmeet b |  b \in B}
 \hskip 2cm 
     \LTCimpaB = \set{a \LTCimp b |  b \in B}
\]

\begin{proposition} [Adjoint Functor Theorem]\label{HeytingAFT}
Let $H$ be a Heyting algebra. 
\begin{enumerate}
\item $a \wedge \QTargument$ preserves all joins which exist in $H$. 
\item $a \LTCimp \QTargument$ preserves all meets which exist in $H$. 
\end{enumerate}
%
\[ 
a \LTCmeet \bigvee B  = \LTCbigwedge
\hskip 3cm
a \LTCimp  \bigwedge B = \LTCbigimp
\]
%
\end{proposition}

%
\begin{lemma} [an equational representation of a Heyting algebra]
Let $H$ be a lattice, $\LTCimp$ a binary operation on $H$. 
Then $\LTCimp$ makes $H$ into a Heyting algebra
iff the following equations hold for all $a,b,c \in H$. 
\begin{eqnarray}
a\LTCimp a &=& \top \\
a\wedge(a\LTCimp b) &=& a\wedge b \\
b\wedge(a\LTCimp b) &=& b \\
c\LTCimp (a\wedge b) &=& (c\LTCimp a)\wedge (c\LTCimp b) 
\end{eqnarray}
\end{lemma}
%
\begin{proof}
$(\Leftarrow)$ Suppose the equations hold. 
Recall that the last equation says that 
$c \LTCimp \QTargument$ preserves meets (i.e., be an endomorphism), 
hence order preserving. 
\begin{prooftree}
	\AxiomC{$ c \fCenter \LTCimpab$}
    \UnaryInfC{$ \LTCmeetac \fCenter a \wedge (\LTCimpab)$}
    \AxiomC{$a \wedge (\LTCimpab) = \LTCmeetab$}
    \AxiomC{$\LTCmeetab \fCenter b$}
    \TrinaryInfC{$\LTCmeetca \fCenter b$}
 \end{prooftree}
\begin{prooftree}
	\AxiomC{$\LTCimpaa = \top$}
    	\UnaryInfC{$ \top \fCenter \LTCimpaa$}
    \AxiomC{$c \LTCmeet (\LTCimpac) = c$}
    	\UnaryInfC{$c \fCenter \LTCimpac$} 
    \BinaryInfC{$c \fCenter (\LTCimpaa) \LTCmeet (\LTCimpac)$}
    \AxiomC{$(\LTCimpaa) \LTCmeet (\LTCimpac) = a \LTCimp (\LTCmeetac)$}
    \AxiomC{$\LTCmeetca \fCenter b$}
    	\UnaryInfC{$a \LTCimp (\LTCmeetac) \fCenter \LTCimpab$}
    \TrinaryInfC{$c \fCenter \LTCimpab$}
\end{prooftree}
%
$(\Rightarrow)$ Suppose $H$ is a Heyting algebra.
\begin{enumerate}
\item \hskip -5cm
\begin{prooftree}
	\AxiomC{$ a \wedge \top \fCenter a$}
    \UnaryInfC{$ \top \fCenter a \LTCimp a$}
    \AxiomC{$ a \LTCimp a \fCenter \top$}
    \BinaryInfC{$ a \LTCimp a = \top$}
\end{prooftree}
\item \hskip -5cm
\begin{prooftree}
	\LTCaxiom{a}
    \AxiomC{$ b \wedge a \fCenter b$}
	\UnaryInfC{$ b \fCenter \LTCimpab$}
    \BinaryInfC{$a \wedge b \fCenter  a \wedge (\LTCimpab)$}
    \DisplayProof \hskip 2cm
    \AxiomC{$ a \wedge (\LTCimpab) \fCenter a$}
    \LTCaxiom{\LTCimpab}
    	\UnaryInfC{$a \wedge (\LTCimpab) \fCenter b$}
    \BinaryInfC{$ a \wedge (\LTCimpab) \fCenter \LTCmeetab $}
\end{prooftree}
\item \hskip -5cm
\begin{prooftree}
	\AxiomC{$ b \wedge a \fCenter b$}
    \UnaryInfC{$b \fCenter \LTCimpab$}
    \UnaryInfC{$b \wedge (\LTCimpab) = b$}
\end{prooftree}
\item Proposition~\ref{HeytingPreservation}.
\end{enumerate}
\end{proof}

 % frames
\begin{definition} [frames]
An {\em Frame} is a double $(L,\bigvee)$.
The following conditions hold:
\begin{enumerate}
\item {\em join-semilattice }: $(L,\bigvee)$ is a join-semilattice,
\item {\em infinite distributive law}:  
$a \LTCmeet \QTargument$ preserves all joins in $L$. 
\[ a \LTCmeet \bigvee C = \bigvee (a \LTCmeet C)\]
\end{enumerate}
\end{definition}
%%
\begin{proposition} [Adjoint Functor Theorem]
If $L$ has all joins and $ a \LTCmeet \QTargument$ preserves them, 
then it has a right adjoint, say $a \LTCimp \QTargument$. 
\end{proposition}
%%
\begin{corollary}
A frame is a complete Heyting algebra.
Specifically, a finite distributive lattice is a (finite) Heyting algebra. 
\end{corollary}
%%
%%  Boolean algebras
%%
\subsection{Complements, Negations and Boolean Algebras}

\begin{definition}[complement]\label{complement}
Let $L$ be a lattice. A {\em complement}  of $a$ ($a \in L$) is an element $ x \in A$ satisfying the following conditions.
\begin{enumerate}
\item $a \vee x = \top$
\item $a \wedge x = \bot$ 
\end{enumerate}
\end{definition}

\begin{proposition}[uniqueness of complements] \label{complementIsUnique}
In disributive lattices, complements are unique when they exists.
\end{proposition}
\begin{proof}
Let $L$ be a distributive lattice. 
Suppose $x \in L$ and $y \in L$ both satisfy the conditions. Then
%
\[
x = x \wedge \top
= x \wedge ( y \vee a )
= (x \wedge y)  \vee ( x \wedge a)
= (x \wedge y) \vee \bot
=  x \wedge y  \mbox{,}
\]
%
\[ x = x \vee \bot 
= x \vee ( y \wedge a) = (x \vee y) \wedge (x \vee a) 
= (x \vee y) \wedge \top
= x \vee y \mbox{,}
\]
showing $x \leq y$ and $y \leq x$, concluding $x = y$. 
\end{proof}

% Boolean Algebras
\begin{definition} [Boolean Algebra]
A {\em Boolean algebra} is a sextuple  
$(L,\LTCjoin,\bot,\LTCmeet,\top,\LTCneg)$.
The following conditions hold:
\begin{enumerate}
\item {\em distributive lattice}: 
$(L,\LTCjoin,\bot,\LTCmeet,\top)$ is a  distributive lattice
 (Definition~\ref{distributiveLattice}),
\item {\em complement}:
 $\LTCneg{a}$ is a complement of $a$ (Definition~\ref{complementIsUnique}).
\end{enumerate}
\end{definition}

\begin{proposition} [Every Boolean algebra is a Heyting algebra]
Every Boolean algebra is a Heyting algebra.
\end{proposition}
\begin{proof}
Let  $L$ be a boolean algebra and $a,b \in L$. Set $a \LTCimp b = \LTCneg a \LTCjoin b$. Then residual law holds.
\begin{prooftree}
	\AxiomC{$c  \fCenter \LTCneg a \LTCjoin b$}
    \UnaryInfC{$a \LTCmeet c \fCenter a \LTCmeet (\LTCneg a \LTCjoin b) $}
    \AxiomC{$ a \LTCmeet (\LTCneg a \LTCjoin b)
    = (a \LTCmeet \LTCneg a) \LTCjoin (a \LTCmeet b)
    =  \bot \LTCjoin  (a \LTCmeet b)
    = a \LTCmeet b 
    \fCenter b$}
    \BinaryInfC{ $ c \LTCmeet a \fCenter b$}
\end{prooftree}
\begin{prooftree}
    \AxiomC{$ c \fCenter \LTCneg a \LTCjoin c
    = (\LTCneg a \LTCjoin c) \LTCmeet \top
    = (\LTCneg a \LTCjoin c) \LTCmeet (\LTCneg a \LTCjoin a)
    = \LTCneg a \LTCjoin (c \LTCmeet a) $}
    \AxiomC{$c  \LTCmeet a \fCenter  b$}
      \UnaryInfC{$ \LTCneg a \LTCjoin (\LTCmeetca) \fCenter \LTCneg a \LTCjoin b$}
    \BinaryInfC{$c \fCenter \LTCneg a \LTCjoin b$}
\end{prooftree}
\end{proof}

\begin{proposition}\label{doubleNegationElimination}
Let $L$ be a Boolean algebra. Then every $a \in L$ we have $a = \LTCneg\LTCneg a$. 
\end{proposition}
\begin{proof}
Since $\LTCneg a$ is a complement of $a$, 
$a \LTCmeet \LTCneg a = \LTCbot$ and $a \LTCjoin \LTCneg a = \LTCtop$ hold. 
Simlarly  $\LTCneg\LTCneg a$ is a complement of $\LTCneg a$, 
$\LTCneg\LTCneg a \LTCmeet \LTCneg a = \LTCbot$ and $\LTCneg\LTCneg a \LTCjoin \LTCneg a = \LTCtop$ hold.
From the uniqueness of complement (Proposition~\ref{complement}) we conclude $ a = \LTCneg\LTCneg a$. 
\end{proof}

Since $\LTCneg a = \LTCneg a \LTCjoin \LTCbot = a \LTCimp \LTCbot$, we can recover the unary operation $\LTCneg$ from the binary operation $\LTCimp$.
In a general Heyting algebra, we use this as the definition of $\LTCneg$.

\begin{definition} [negation] \label{HeytingNegation}
Let $H$ be a Heyting algebra. 
We set 
\[ \LTCneg a \defeq a \LTCimp \LTCbot  \] 
to produce an operation, called {\em negation} 
$ \LTCneg{\QTargument}  : H  \rightarrow H$.
\end{definition}

\begin{proposition} \label{HeytingComplement}
In a Heyting algebra, $\LTCneg a$ is the unique largest element $x$ satisfying 
$a \LTCmeet x = \LTCbot$.
\end{proposition}
\begin{proof} \hskip -7cm
\begin{prooftree}
\AxiomC{$\LTCneg{a} \fCenter \LTCneg{a}$}
\UnaryInfC{$a \LTCmeet \LTCneg{a} \fCenter \LTCbot$}
\DisplayProof \hskip 3cm 
\AxiomC{$a \LTCmeet x \fCenter \LTCbot$}
\UnaryInfC{$x \fCenter \LTCneg{a}$}
\end{prooftree}
\end{proof}


\newcommand{\LTCdeMorganMeet}{\LTCnegP{\LTCnega \LTCmeet \LTCnegb}}
\newcommand{\LTCdeMorganJoin}{\LTCnegP{\LTCnega \LTCjoin \LTCnegb}}
\newcommand{\LTCdeMorganJoinD}{\LTCneg\LTCnegP{\LTCnega \LTCjoin \LTCnegb}}


\begin{proposition}
The law $\LTCneg{\LTCbot} = \LTCtop$ and $\LTCneg(\LTCjoinab) = \LTCnega \LTCmeet \LTCnegb$ holds in any Heyting algebra.
\end{proposition}
\begin{proof}
\hskip -4cm
\begin{prooftree}
\AxiomC{$\LTCtop \LTCmeet \LTCbot =   \LTCbot  \fCenter \LTCbot$}
\UnaryInfC{$\LTCtop \fCenter \LTCneg\LTCbot$}
\end{prooftree}
%
\begin{prooftree}
\AxiomC{$a \fCenter a \LTCjoin b$} 
 \UnaryInfC {$\LTCnegP{a \LTCjoin b} \fCenter \LTCnega $} 
\AxiomC{$b \fCenter a \LTCjoin b$} 
 \UnaryInfC {$\LTCnegP{a \LTCjoin b} \fCenter \LTCnegb $} 
  \BinaryInfC  {$\LTCnegP{a \LTCjoin b} \fCenter \LTCnega \LTCmeet \LTCnegb $}
%
\DisplayProof \hskip 12pt
%
\AxiomC {$\LTCnega \LTCmeet \LTCnegb  \fCenter  \LTCnega$}
 \UnaryInfC{$a \fCenter \LTCdeMorganMeet$}
\AxiomC {$\LTCnega \LTCmeet \LTCnegb  \fCenter  \LTCnegb$}
 \UnaryInfC{$b \fCenter \LTCdeMorganMeet$}
 \BinaryInfC  {$a \LTCjoin b\fCenter \LTCdeMorganMeet$}
 \UnaryInfC {$ \LTCnega \LTCmeet \LTCnegb \fCenter \LTCnegP{a \LTCjoin b} $}
\end{prooftree}

\end{proof}

Howerver we do not have $a \LTCjoin \LTCnega = \LTCtop$ , $\LTCneg\LTCtop  = \LTCbot$ and $\LTCneg(\LTCmeetab) = \LTCnega \LTCjoin \LTCnegb$ in a Heyting algebra in general. 

\begin{proposition}
The law $\LTCdneg\LTCtop = \LTCtop$ and $\LTCdneg(\LTCmeetab) = \LTCneg\LTCnega \LTCmeet \LTCneg\LTCnegb$ holds in any Heyting algebra.
\end{proposition}

\begin{proof}
\begin{prooftree}
\AxiomC{$ \LTCtop \fCenter \LTCtop$}
\UnaryInfC{$\LTCtop \fCenter \LTCdneg\LTCtop$}
\end{prooftree}
% \hskip -4cm
\begin{prooftree}
\AxiomC{$ \LTCmeetab \leq a$}
\UnaryInfC{$ \LTCdneg(\LTCmeetab)  \fCenter \LTCdnega$}
\AxiomC{$ \LTCmeetab \leq b$}
\UnaryInfC{$ \LTCdneg(\LTCmeetab)  \fCenter \LTCdnegb$}
\BinaryInfC{$ \LTCdneg(\LTCmeetab) \fCenter \LTCdnega \LTCmeet \LTCdnegb$}
\DisplayProof \hskip  12pt 
\AxiomC{$ \LTCneg(\LTCmeetab) \fCenter \LTCneg(\LTCmeetab)$}
\UnaryInfC{$ a \LTCmeet b \LTCmeet \LTCneg(\LTCmeetab) \fCenter \LTCbot$}
\UnaryInfC{$ a \LTCmeet \LTCneg(\LTCmeetab) \fCenter \LTCnegb$}
\UnaryInfC{$ a \LTCmeet \LTCdnegb \LTCmeet \LTCneg(\LTCmeetab) \fCenter \LTCbot$}
\UnaryInfC{$ \LTCdnegb \LTCmeet \LTCneg(\LTCmeetab) \fCenter \LTCnega$}
\UnaryInfC{$ \LTCdnega \LTCmeet \LTCdnegb \LTCmeet \LTCneg(\LTCmeetab) \fCenter \LTCbot$}
\UnaryInfC{$ \LTCdnega \LTCmeet \LTCdnegb  \fCenter \LTCdneg(\LTCmeetab)$}
\end{prooftree}
\end{proof}


\newcommand{\LTCcomplementJ}{a \LTCjoin \LTCneg a}

\begin{proposition}
A Heyting algebra $H$ is a Boolean algebra if $\LTCneg\LTCneg a = a$ for all $a \in H$. 
\end{proposition}
\begin{proof}
We know that a Heyting algebra is distributive (Proposition~\ref{HeytingPreservation}).
We also know $a \LTCmeet \LTCnega = \LTCbot$ holds  (Proposition~\ref{HeytingComplement}).
Hence we need only verify the identity $\LTCcomplementJ = \LTCtop$. 
\begin{prooftree}
\AxiomC{$\LTCneg a \fCenter \LTCcomplementJ$}
	\UnaryInfC{$\LTCneg(\LTCcomplementJ) \fCenter \LTCneg\LTCneg a = a$}
\AxiomC{$a \fCenter \LTCcomplementJ$}
	\UnaryInfC{$\LTCneg(\LTCcomplementJ) \fCenter \LTCneg a$}
\BinaryInfC{$\LTCneg(\LTCcomplementJ) \fCenter a \LTCmeet \LTCneg a = \LTCbot$}
\UnaryInfC{$\LTCtop \fCenter \LTCneg\LTCneg(\LTCcomplementJ) = \LTCcomplementJ$}
\end{prooftree}
\end{proof}

%  A bijection \theta between lattices is an isomorphism 
%  iff both \theta and \theta^{-1} are order preserving; (= order-embedding)
% then \theta^{-1} is also an isomorphism.   (Abstract Algebra  P.542  Pierre Antoine Grillet)

%  order isomorphism = bijective + order-embedding 
%  1) order-embedding = order-preserving + order-reflecting
%  2)  bijective =  injective (one-to-one) + surjective (onto)

%
%  lattice isomorphism = bijective + lattice homomorphism 
%

%
%  order-isomorphism  iff   lattice isomorphism 




%
\begin{proposition} [De Morgan Law]
De Morgan Law holds in Boolean algebras. Note that it exactly says that $\LTCneg$ is a lattice isomorphisim. 
\[ 
 \LTCneg\LTCbot = \LTCtop \hskip 48pt
  \LTCnegP{a \LTCjoin b} =  \LTCneg a \LTCmeet \LTCneg b  
 \]
  \[
    \LTCneg\LTCtop = \LTCbot
  \hskip  48pt 
   \LTCnegP{a \LTCmeet b} =  \LTCneg a \LTCjoin \LTCneg b
   \] 
\end{proposition}
\begin{proof}
The condition $\LTCneg\LTCneg a = a$ (Proiposition~\ref{doubleNegationElimination}) implies 
that $\LTCneg$ is a bijection. 
Moreover, both $\LTCneg \QTargument : A \longrightarrow A^{\sf op}$ and 
$\LTCneg \QTargument : A^{\sf op} \longrightarrow A$ are order-preserving maps,
because $\LTCneg a = \LTCneg a \LTCjoin \LTCbot = a \LTCimp \LTCbot$ is order reversing.
%
This means $\LTCneg$ is an order-isomorphism thus a lattice isomorphism(Proposition~\ref{LTC:latticeIsomorphism}).
%
Concrete proof can also be shown. 
\[ \LTCneg\LTCtop = \LTCneg\LTCneg\LTCbot = \LTCbot \]
%
\begin{prooftree}
\AxiomC{$a \LTCmeet b  \fCenter a$} 
 \UnaryInfC {$ \LTCnega \fCenter \LTCnegP{a \LTCmeet b} $} 
\AxiomC{$a \LTCmeet b \fCenter b$} 
  \UnaryInfC {$ \LTCnegb \fCenter \LTCnegP{a \LTCmeet b} $} 
  \BinaryInfC  {$\LTCneg a \LTCjoin \LTCnegb \fCenter \LTCnegP{a \LTCmeet b} $}
%
\DisplayProof \hskip 6pt
%
\AxiomC {$\LTCnega  \fCenter  \LTCnega \LTCjoin \LTCnegb $}
 \UnaryInfC{$\LTCdeMorganJoin \fCenter \LTCdnega $}
 \UnaryInfC{$\LTCdeMorganJoin \fCenter a $}
\AxiomC {$\LTCnegb  \fCenter  \LTCnega \LTCjoin \LTCnegb $}
  \UnaryInfC{$\LTCdeMorganJoin \fCenter \LTCdnegb $}
  \UnaryInfC{$\LTCdeMorganJoin \fCenter b $}
 \BinaryInfC  {$\LTCdeMorganJoin \fCenter a \LTCmeet b$}
 \UnaryInfC {$ \LTCnegP{ a \LTCmeet b }  \fCenter \LTCdeMorganJoinD$}
 \UnaryInfC {$ \LTCnegP{ a \LTCmeet b }  \fCenter \LTCnega \LTCjoin \LTCnegb$}
\end{prooftree}
\end{proof}

\subsection{De Morgan Algebra}

\begin{definition} [De Morgan Algebra]
A {\em De Morgan algebra} is a sextuple  
$(L,\LTCjoin,\bot,\LTCmeet,\top,\LTCneg)$.
The following conditions hold:
\begin{enumerate}
\item {\em distributive lattice:}
 $(L,\LTCjoin,\bot,\LTCmeet,\top)$ is a  distributive lattice (Definition~\ref{distributiveLattice}),
\item {\em De Morgan Laws:}
$\LTCneg(a \LTCmeet b) = \LTCneg a \LTCjoin \LTCneg b$,
\item {\em involution:}  $\LTCneg\LTCneg a = a$. 
\end{enumerate}
\end{definition}

\begin{proposition}
Let $(A,\leq)$ be a poset and  $(A^{\sf op},\geq)$ its opposite. 
Then $ \LTCneg\QTargument : A \longrightarrow A^{\sf op}$ 
is a bijective lattice homomorphism,
thus order isomorphisms(Proposition~\ref{LTC:latticeIsomorphism}).
\end{proposition}
\begin{proof}

\[  \LTCneg ( a \LTCjoin b) 
= \LTCneg ( \LTCdneg a \LTCjoin \LTCdneg b)
= \LTCdneg ( \LTCneg a \LTCmeet \LTCneg b)
= \LTCneg a \LTCmeet \LTCneg b
\]

\[ \LTCneg\LTCtop
= \LTCneg\LTCtop \LTCjoin \LTCbot
= \LTCneg\LTCtop \LTCjoin \LTCdneg\LTCbot
= \LTCneg (\LTCtop \LTCmeet \LTCneg\LTCbot)
= \LTCdneg\LTCbot
= \LTCbot
\]

\[ \LTCneg\LTCbot
= \LTCneg\LTCbot \LTCmeet \LTCtop
= \LTCneg\LTCbot \LTCmeet \LTCdneg\LTCtop
= \LTCneg (\LTCbot \LTCjoin \LTCneg\LTCtop)
= \LTCdneg\LTCtop
= \LTCtop
\]

\end{proof}

