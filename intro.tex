% Attack trees are useful

%  structural tree notation can be identified with formulas of  formal logic.

% What is attack tree. 

An attack tree is a tree which the nodes represent goals.
The root node  of the tree  is the global goal.
Children of a  node are refinements of  a goal to sub-goals.
A refinement can be  conjunctive or disjunctive.
Disjunctive nodes represent different ways to achieving the same goal.
Conjunctive nodes represent different steps in achieving the goal.

The leafs,  therefore,  attacks that can no longer be refined. 

%  as a standard notation  for  threat analysis documents.

%  


%  in general
This article describes how purely mathematical framework can be used
in order to evaluate the security of complex systems. 
%  syntax and semantics
Attack Tree models are 
analyze how target might be attacked or defensed. 
% syntax is a propositional classical linear logic without modality
% it is  contraction and weakening free logic 
% semantics   CLL algebra or Quantale
% CLL algebra includes  two-element (TRUE/FALSE) boolean algebra
% CLL algebra includes  finite totally ordered sets. 

the notion of algebraizable logic \cite{sep-consequence-algebraic}


% back ground - attack trees

Attack trees (the term is introduced by Schneier in \cite{Schneier:1999:AT}) 

There are Good Surveys \cite{DBLP:journals/corr/abs-1303-7397}

% formal methods of attack trees

This is attack defence tree \cite{KoMaRaSc2}.  Its formal semantics is introduced in  \cite{DBLP:conf/ifip1-7/KordyMRS10}

% back ground - linear logic and phase semantics

This is Girard's linear logic paper \cite{Girard88book}.
There is a good text for linear logic  in the book\cite{Troelstra92}.

% back ground phase semantics 

Phase space semantics \cite{Girard87tcs} which is improved in  \cite{Girard95a}.

% methods

% results and conclusions

